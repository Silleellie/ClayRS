%! Author = DIEGO
%! Date = 19/02/2024

###
% Created by Diego Miccoli

% Preamble
\documentclass[11pt]{article}
% [twocolumn]

\usepackage{graphicx} % Required for inserting images
\usepackage{amsmath}
\usepackage{verbatim}
\usepackage{hyperref}
\usepackage{comment}
\usepackage{caption}
\usepackage{url}
\usepackage{listings}
\usepackage{float}
\usepackage{array}
\usepackage{booktabs}
\usepackage{multirow}
\usepackage{makecell}
\usepackage{geometry}
\usepackage{changepage}
\usepackage{longtable}
\usepackage{soul}

\usepackage{background}

\backgroundsetup{
    scale=1,
    angle=0,
    opacity=1,
    color=black,
    position=current page.north east,
    vshift=-0.3cm, % Regola questa dimensione per posizionare il testo verticalmente
    hshift=-3.5cm, % Regola questa dimensione per posizionare il testo orizzontalmente
    contents={%
        Aldo Moro University of studies of Bari
    }
}

\geometry{top=1cm, bottom=1cm}

\title{\textbf{ !! }\\ [1cm] Department of Computer Science}
\author{ ?? \thanks{Experiment conduct by Z}}
\date{X}

\begin{document}

\maketitle

\section{Introduction}\label{sec:intro}
Recommender Systems (RS) are designed to assist users in various decision-making tasks by acquiring
information about their needs, interests, and preferences in order to personalize the user experience
based on such data.
These systems are based on paradigms that allow managing user information and processing
it to provide decision support.
In particular, two successful paradigms are Collaborative Filtering and Content-based Filtering.
In this document, we will present the results obtained using a recommendation system called \textbf{ClayRS},
a framework developed by the SWAP research group at the Department of Computer Science of the University of Bari.\\
\hfill\break

\textit{\ul{The following report has been automatically generated from YAML configuration files.}}

\hfill\break

###



\begin{comment}
Author = DIEGO MICCOLI
Alias = Kozen88
Organization = SWAP Research Group UniBa
Date = 27-12-2023

This mini template is not working by itself because there are latex command missing needed
to compile the file and give as output a pdf file, in addition it has been added jinja
statement in order to control the rendering of the latex file with the jinja library, for these
reasons it needs to be used with the other mini chunks in conjunction.
\end{comment}