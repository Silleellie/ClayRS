%! Author = DIEGO MICCOLI
%! Date = 27/12/2023

\usepackage{comment}
\usepackage{hyperref}
\usepackage{amsmath}


###

\BLOCK{if my_dict['metrics'] is defined or my_dict['sys_results'] is defined}
% -------------------------------------- OPENING THE EVALUATION MODULE SECTION ---------------------------------------
\section{Evaluation Module}\label{sec:eva-module}
The \textbf{EvalModel} which is the abbreviation for Evaluation Model has the task of evaluating a recommender system,
using several state-of-the-art metrics, this allows to compare different recommender system and different algorithm of
recommendation and find out which are the strength points and which the weak ones.

\BLOCK{if my_dict['metrics'] is defined}
\subsection{Metrics}\label{subsec:metrics}
% --- Metrics ---
During the experiment a bunch of formal metrics have been performed on the recommendation produced in order to evaluate
the performance of the system.
The metrics used are the followings:
\hfill\break
\hfill\break

% ---------------------------------------- CLASSIFICATION METRICS STARTED --------------------------------------------
\BLOCK{if my_dict['metrics'] is defined and
          my_dict['metrics']['Precision'] is defined}
% Precision report start
\subsubsection{Precision}\label{subsubsec:precision}
In ClayRS, the Precision metric is calculated as such for the \textbf{single user}:
\hfill\break
\hfill\break
    \[
        \begin{gathered}
         Precision_u = \frac{tp_u}{tp_u + fp_u}
        \end{gathered}
    \]
\hfill\break
\hfill\break
    Where:
\begin{itemize}
    \item $tp_u$ is the number of items which are in the recommendation list of the user and have a
       $\geq$ \BLOCK{ if my_dict.get('metrics', {}).get('Precision', {}).get('relevant_threshold') is not none}
        \textbf{\VAR{my_dict['metrics']['Precision']['relevant_threshold']|truncate}}.
        \BLOCK{else}
        \textbf{\VAR{my_dict['interactions']['mean_score']|default('no relevant threshold used')|truncate}}.
        \BLOCK{endif}
    \item $fp_u$ is the number of items which are in the recommendation list of the user and have a
      rating $<$ \BLOCK{if my_dict.get('metrics', {}).get('Precision', {}).get('relevant_threshold') is not none}
        \textbf{\VAR{my_dict['metrics']['Precision']['relevant_threshold']|truncate}}.
        \BLOCK{else}
        \textbf{\VAR{dict['interactions']['mean_score']|default('no relevant threshold used')|truncate}}.
        \BLOCK{endif}
\end{itemize}
\hfill\break
\hfill\break
In ClayRS, Precision needs those parameters:
the \textbf{relevant\_threshold}, is a parameter needed to discern relevant items and non-relevant items for every user.
If not specified, the mean rating score of every user will be used, in this experiment it has been set to
\textbf{\VAR{my_dict['metrics']['Precision']['relevant_threshold']|truncate|safe_text}}.
\hfill\break
\hfill\break
% precision report ended___
\BLOCK{endif}

\BLOCK{if my_dict['metrics'] is defined and
          my_dict['metrics']['Recall'] is defined}
% Recall report start
\subsubsection{Recall}\label{subsubsec:recall}
The Recall metric is calculated as such for the \textbf{single user}:
\hfill\break
\hfill\break
    \[
       \begin{gathered}
           Recall_u = \frac{tp_u}{tp_u + fn_u}
       \end{gathered}
    \]
\hfill\break
\hfill\break
    Where:
\begin{itemize}
    \item $tp_u$ is the number of items which are in the recommendation list of the user and have a
      rating $>=$ \BLOCK{ if my_dict.get('metrics', {}).get('Recall', {}).get('relevant_threshold') is not none}
        \textbf{\VAR{my_dict['metrics']['Recall']['relevant_threshold']|truncate}}.
        \BLOCK{else}
        \textbf{\VAR{my_dict['interactions']['mean_score']|default('no relevant threshold used')|truncate}}.
        \BLOCK{endif}
    \item $fn_u$ is the number of items which are not in the recommendation list of the user and have a
      rating $>=$ \BLOCK{if my_dict.get('metrics', {}).get('Recall', {}).get('relevant_threshold') is not none}
        \textbf{\VAR{my_dict['metrics']['Recall']['relevant_threshold']|truncate}}.
        \BLOCK{else}
        \textbf{\VAR{dict['interactions']['mean_score']|default('no relevant threshold used')|truncate}}.
        \BLOCK{endif}
\end{itemize}
\hfill\break
\hfill\break
In ClayRS, Recall needs those parameters:
the \textbf{relevant\_threshold}, is a parameter needed to discern relevant items and non-relevant items for every user.
If not specified, the mean rating score of every user will be used, in this experiment it has been set to
\textbf{\VAR{my_dict['metrics']['Recall']['relevant_threshold']|default('no relevant threshold used')|truncate|safe_text}}.
\hfill\break
\hfill\break
% recall report end___
\BLOCK{endif}

\BLOCK{if my_dict['metrics'] is defined and
          my_dict['metrics']['FMeasure'] is defined}
%FMeasure report start
\subsubsection{FMeasure}\label{subsubsec:f-meas}
The FMeasure metric combines Precision and Recall into a single metric.
It is calculated as such for the \textbf{single user}:
\hfill\break
\hfill\break
    \[
       \begin{gathered}
           FMeasure_u = (1 + \beta^2) \cdot \frac{P_u \cdot R_u}{(\beta^2 \cdot P_u) + R_u}
       \end{gathered}
    \]
\hfill\break
\hfill\break
    Where:
\begin{itemize}
    \item $P_u$ is the Precision calculated for the user \textbf{u}.
    \item $R_u$ is the Recall calculated for the user \textbf{u}.
    \item $\beta$ is a real factor which could weight differently Recall or Precision based on its value:
    \begin{itemize}
        \item $\beta = 1$: Equally weight Precision and Recall.
        \item $\beta > 1$: Weight Recall more.
        \item $\beta < 1$: Weight Precision more.
    \end{itemize}
\end{itemize}
\hfill\break
\hfill\break
A famous FMeasure is the F1 Metric, where $\beta = 1$, which basically is the harmonic mean of recall and
precision:
\hfill\break
\hfill\break
    \[
       \begin{gathered}
           F1_u = \frac{2 \cdot P_u \cdot R_u}{P_u + R_u}
       \end:{gathered}
    \]
\hfill\break
\hfill\break
The FMeasure metric is calculated as such for the entire system, depending on if \textbf{macro} average or
\textbf{micro} average has been chosen:
\hfill\break
\hfill\break
    \[
       \begin{gathered}
           FMeasure_{sys} - micro = (1 + \beta^2) \cdot \frac{P_u \cdot R_u}{(\beta^2 \cdot P_u) + R_u}
       \end{gathered}
    \]
\hfill\break
\hfill\break
    \[
      \begin{gathered}
          FMeasure_{sys} - macro = \frac{\sum_{u \in U} FMeasure_u}{|U|}
      \end{gathered}
    \]
\hfill\break
\hfill\break
During the experiment the FMeasure has been calculated with $\beta = $
\VAR{my_dict['metrics']['FMeasure']['beta']|safe_text} and the relevant threshold is
\textbf{\VAR{my_dict['metrics']['FMeasure']['relevant_threshold']|default('no relevant threshold used')|safe_text}}.
\hfill\break
\hfill\break
% FMeasure report end___
\BLOCK{endif}

\BLOCK{if my_dict['metrics'] is defined and
          my_dict['metrics']['PrecisionAtK'] is defined}
% PRECISION @K REPORT start
\subsubsection{Precision@K}\label{subsubsec:prec-k}
The Precision@K metric is calculated as such for the \textbf{single user}:
\hfill\break
\hfill\break
    \[
      \begin{gathered}
          Precision@K_u = \frac{tp@K_u}{tp@K_u + fp@K_u}
      \end{gathered}
    \]
\hfill\break
\hfill\break
    Where:
\begin{itemize}
    \item $tp@K_u$ is the number of items which are in the recommendation list of the user
      cutoff to the first K items and have a rating $>=$ relevant threshold in its ground truth.
    \item $tp@K_u$ is the number of items which are in the recommendation list of the user
      cutoff to the first K items** and have a rating $<$ relevant threshold in its ground truth.
\end{itemize}
\hfill\break
\hfill\break
And it is calculated as such for the entire system, depending on if \textbf{macro} average or \textbf{micro} average
has been chosen:
\hfill\break
\hfill\break
   \[
      \begin{gathered}
          Precision@K_{sys} - micro = \frac{\sum_{u \in U} tp@K_u}{\sum_{u \in U} tp@K_u + \sum_{u \in U} fp@K_u}
      \end{gathered}
   \]
\hfill\break
\hfill\break
    \[
      \begin{gathered}
          Precision@K_{sys} - macro = \frac{\sum_{u \in U} Precision@K_u}{|U|}
      \end{gathered}
    \]
\hfill\break
\hfill\break
During the experiment Precision@K has been used with the following settings:
\begin{itemize}
    \item \textbf{K: \VAR{my_dict['metrics']['PrecisionAtK']['k']|safe_text} }
    \item \textbf{relevant threshold:\VAR{my_dict['metrics']['PrecisionAtK']['relevant_threshold']|default('no relevant threshold used')|safe_text} }
    \item \textbf{sys average: \VAR{my_dict['metrics']['PrecisionAtK']['sys_average']|safe_text} }
\end{itemize}
\hfill\break
\hfill\break
% precision@k report end___
\BLOCK{endif}

\BLOCK{if my_dict['metrics'] is defined and
          my_dict['metrics']['RecallAtK'] is defined}
% RECALL @K REPORT start
\subsubsection{Recall@K}\label{subsubsec:rec-k}
The Recall@K metric is calculated as such for the \textbf{single user}:
\hfill\break
\hfill\break
    \[
       \begin{gathered}
           Recall@K_u = \frac{tp@K_u}{tp@K_u + fn@K_u}
       \end{gathered}
    \]
\hfill\break
\hfill\break
    Where:
\begin{itemize}
    \item $tp@K_u$ is the number of items which are in the recommendation list of the user
      \textbf{cutoff to the first K items} and have a rating $>=$ relevant threshold in its ground truth
    \item $tp@K_u$ is the number of items which are NOT in the recommendation list of the user
      \textbf{cutoff to the first K items} and have a rating $>=$ relevant threshold in its ground truth
\end{itemize}
\hfill\break
\hfill\break
And it is calculated as such for the entire system, depending on if \textbf{macro} average or \textbf{micro} average
has been chosen:
\hfill\break
\hfill\break
    \[
       \begin{gathered}
           Recall@K_{sys} - micro = \frac{\sum_{u \in U} tp@K_u}{\sum_{u \in U} tp@K_u + \sum_{u \in U} fn@K_u}
       \end{gathered}
    \]
\hfill\break
\hfill\break
    \[
      \begin{gathered}
          ecall@K_{sys} - macro = \frac{\sum_{u \in U} Recall@K_u}{|U|}
      \end{gathered}
    \]
\hfill\break
\hfill\break
During the experiment Recall@K has been used with the following settings:
\begin{itemize}
    \item \textbf{K: \VAR{my_dict['metrics']['RecallAtK']['k']|safe_text} }
    \item \textbf{relevant threshold: \VAR{my_dict['metrics']['RecallAtK']['relevant_threshold']|default('no relevant threshold used')|safe_text} }
    \item \textbf{sys average: \VAR{my_dict['metrics']['RecallAtK']['sys_average']|safe_text} }
\end{itemize}
\hfill\break
\hfill\break
% recall@k report end___
\BLOCK{endif}

\BLOCK{if my_dict['metrics'] is defined and
          my_dict['metrics']['FMeasureAtK'] is defined}
% FMEASURE @K REPORT start
\subsubsection{FMeasure@K}\label{subsubsec:f-meas-k}
The FMeasure@K metric combines Precision@K and Recall@K into a single metric.
It is calculated as such for the \textbf{single user}:
\hfill\break
\hfill\break
    \[
       \begin{gathered}
           FMeasure@K_u = (1 + \beta^2) \cdot \frac{P@K_u \cdot R@K_u}{(\beta^2 \cdot P@K_u) + R@K_u}
       \end{gathered}
    \]
\hfill\break
\hfill\break
    Where:
\begin{itemize}
    \item $P@K_u$ is the Precision at K calculated for the user \textbf{u}.
    \item $R@K_u$ is the Recall at K calculated for the user \textbf{u}.
    \item $\beta$ is a real factor which could weight differently Recall or Precision based on its value:
    \begin{itemize}
        \item $\beta = 1$: Equally weight Precision and Recall.
        \item $\beta > 1$: Weight Recall more.
        \item $\beta < 1$: Weight Precision more.
    \end{itemize}
\end{itemize}
\hfill\break
\hfill\break
A famous FMeasure@K is the F1@K Metric, where $\beta = 1$, which basically is the harmonic mean of recall and
precision:
\hfill\break
\hfill\break
    \[
      \begin{gathered}
          F1@K_u = \frac{2 \cdot P@K_u \cdot R@K_u}{P@K_u + R@K_u}
      \end{gathered}
    \]
\hfill\break
\hfill\break
The FMeasure@K metric is calculated as such for the entire system, depending on if \textbf{macro} average or
\textbf{micro} average has been chosen:
\hfill\break
\hfill\break
    \[
      \begin{gathered}
          FMeasure@K_{sys} - micro = (1 + \beta^2) \cdot \frac{P@K_u \cdot R@K_u}{(\beta^2 \cdot P@K_u) + R@K_u}
      \end{gathered}
    \]
\hfill\break
\hfill\break
    \[
       \begin{gathered}
           FMeasure@K_{sys} - macro = \frac{\sum_{u \in U} FMeasure@K_u}{|U|}
       \end{gathered}
    \]
\hfill\break
\hfill\break
During the experiment FMeasure@K has been used with the following settings:
\begin{itemize}
    \item \textbf{K: \VAR{my_dict['metrics']['FMeasureAtK']['k']|safe_text}}
    \item  \textbf{$\beta$: \VAR{my_dict['metrics']['FMeasureAtK']['beta']|safe_text}}
    \item \textbf{relevant threshold: \VAR{my_dict['metrics']['FMeasureAtK']['relevant_threshold']|default('no relevant threshold used')|safe_text} }
    \item \textbf{sys average: \VAR{my_dict['metrics']['FMeasureAtK']['sys_average']|safe_text} }
\end{itemize}
\hfill\break
\hfill\break
% fmeasure@k report end___
\BLOCK{endif}

\BLOCK{if my_dict['metrics'] is defined and
          my_dict['metrics']['RPrecision'] is defined}
% R-PRECISION REPORT start
\subsubsection{R-Precision}\label{subsubsec:r-prec}
The R-Precision metric is calculated as such for the \textbf{single user}:
\hfill\break
\hfill\break
    \[
       \begin{gathered}
           R-Precision_u = \frac{tp@R_u}{tp@R_u + fp@R_u}
       \end{gathered}
    \]
\hfill\break
\hfill\break
    Where:
\begin{itemize}
    \item $R$ it's the number of relevant items for the user \textbf{u}.
    \item $tp@R_u$ is the number of items in the recommendation list of the user, up to the first   $R$ items,
        that have a rating $\geq$ the relevant threshold in its ground truth.
    \item $tp@R_u$ is the number of items in the recommendation list of the user, up to the first $R$ items,
        that have a rating $<$ \texttt{relevant\_threshold} in their ground truth.
\end{itemize}
\hfill\break
\hfill\break
And it is calculated as such for the entire system, depending on if \textbf{macro} average or \textbf{micro} average
has been chosen:
\hfill\break
\hfill\break
    \[
       \begin{gathered}
           Precision@R_{sys} - micro = \frac{\sum_{u \in U} tp@R_u}{\sum_{u \in U} tp@R_u + \sum_{u \in U} fp@R_u}
       \end{gathered}
    \]
\hfill\break
\hfill\break
    \[
       \begin{gathered}
           Precision@R_{sys} - macro = \frac{\sum_{u \in U} R-Precision_u}{|U|}
       \end{gathered}
    \]
\hfill\break
\hfill\break
During the experiment R-Precision has been used with the following settings:
\begin{itemize}
    \item \textbf{relevant threshold:\VAR{my_dict['metrics']['RPrecision']['relevant_threshold']|default('no relevant threshold used')|safe_text} }
    \item \textbf{sys average: \VAR{my_dict['metrics']['RPrecision']['sys_average']|safe_text} }
\end{itemize}
\hfill\break
\hfill\break
% r-precision report end___
\BLOCK{endif}
% <----- classification metrics end ------->

% -------------------------------------------- ERROR METRICS STARTED ------------------------------------------------
\BLOCK{if my_dict['metrics'] is defined and
          my_dict['metrics']['MSE'] is defined}
% MSE REPORT start
\subsubsection{MSE}\label{subsubsec:mse}
The MSE abbreviation Mean Squared Error metric is calculated as such for the \textbf{single user}:
\hfill\break
\hfill\break
    \[
       \begin{gathered}
           MSE_u = \sum_{i \in T_u} \frac{(r_{u,i} - \hat{r}_{u,i})^2}{|T_u|}
       \end{gathered}
    \]
\hfill\break
\hfill\break
    Where:
\begin{itemize}
    \item $T_u$ is the test set of the user \textbf{u}.
    \item $r_{u, i}$ is the actual score give by user \textbf{u} to item \textbf{i}.
    \item $\hat{r}_{u, i}$ is the predicted score give by user \textbf{u} to item \textbf{i}.
\end{itemize}
\hfill\break
\hfill\break
It is calculated as such for the entire system:
\hfill\break
\hfill\break
    \[
       \begin{gathered}
           MSE_{sys} = \sum_{u \in T} \frac{MSE_u}{|T|}
       \end{gathered}
    \]
\hfill\break
\hfill\break
    Where:
\begin{itemize}
    \item $T$ is the \textbf{test set}.
    \item $MSE_u$ is the MSE calculated for user \textbf{u}.
\end{itemize}
\hfill\break
\hfill\break
There may be cases in which some items of the test set of the user could not be predicted
\textit{e.g. a CBRS was chosen and items were not present locally}.
In those cases, the $MSE_u$ formula becomes:
\hfill\break
\hfill\break
    \[
       \begin{gathered}
           MSE_u = \sum_{i \in T_u} \frac{(r_{u,i} - \hat{r}_{u,i})^2}{|T_u| - unk}
       \end{gathered}
    \]
\hfill\break
\hfill\break
    Where:
\begin{itemize}
    \item $unk$ stay for unknown is the number of items of the user test set that could not be predicted.
\end{itemize}
\hfill\break
\hfill\break
If no items of the user test set have been predicted $|T_u| - unk = 0$, then:
\hfill\break
\hfill\break
    \[
        \begin{gathered}
            MSE_u = NaN
        \end{gathered}
    \]
\hfill\break
\hfill\break
% mse report end___
\BLOCK{endif}

\BLOCK{if my_dict['metrics'] is defined and
          my_dict['metrics']['RMSE'] is defined}
% RMSE REPORT start
\subsubsection{RMSE}\label{subsubsec:rmse}
The RMSE (Root Mean Squared Error) metric is calculated as such for the \textbf{single user}:
\hfill\break
\hfill\break
    \[
       \begin{gathered}
           RMSE_u = \sqrt{\sum_{i \in T_u} \frac{(r_{u,i} - \hat{r}_{u,i})^2}{|T_u|}}
       \end{gathered}
    \]
\hfill\break
\hfill\break
    Where:
\begin{itemize}
    \item $T_u$ is the test set of the user $u$.
    \item $r_{u, i}$ is the actual score give by user $u$ to item $i$.
    \item $\hat{r}_{u, i}$ is the predicted score give by user $u$ to item $i$.
\end{itemize}
\hfill\break
\hfill\break
It is calculated as such for the entire system:
\hfill\break
\hfill\break
    \[
      \begin{gathered}
          RMSE_{sys} = \sum_{u \in T} \frac{RMSE_u}{|T|}
      \end{gathered}
    \]
\hfill\break
\hfill\break
    Where:
\begin{itemize}
    \item $T$ is the test set.
    \item $RMSE_u$ is the RMSE calculated for user $u$.
\end{itemize}
\hfill\break
\hfill\break
There may be cases in which some items of the test set of the user could not be predicted
\textit{e.g. a CBRS was chosen and items were not present locally, a methodology different from TestRatings was chosen}.
In those cases, the $RMSE_u$ formula becomes:
\hfill\break
\hfill\break
    \[
       \begin{gathered}
           RMSE_u = \sqrt{\sum_{i \in T_u} \frac{(r_{u,i} - \hat{r}_{u,i})^2}{|T_u| - unk}}
       \end{gathered}
    \]
\hfill\break
\hfill\break
    Where:
\begin{itemize}
    \item $unk$ (unknown) is the number of items of the user test set that could not be predicted.
\end{itemize}
\hfill\break
\hfill\break
If no items of the user test set have been predicted $|T_u| - unk = 0$, then:
\hfill\break
\hfill\break
    \[
      \begin{gathered}
          RMSE_u = NaN
      \end{gathered}
    \]
\hfill\break
\hfill\break
% rmse report end___
\BLOCK{endif}

\BLOCK{if my_dict['metrics'] is defined and
          my_dict['metrics']['MAE'] is defined}
% MAE REPORT start
\subsubsection{MAE}\label{subsubsec:mae}
The MAE abbreviation of Mean Absolute Error metric is calculated as such for the \textbf{single user}:
\hfill\break
\hfill\break
    \[
      \begin{gathered}
          MAE_u = \sum_{i \in T_u} \frac{|r_{u,i} - \hat{r}_{u,i}|}{|T_u|}
      \end{gathered}
    \]
\hfill\break
\hfill\break
    Where:
\begin{itemize}
    \item $T_u$ is the test set of the user $u$.
    \item $r_{u, i}$ is the actual score give by user $u$ to item $i$.
    \item $\hat{r}_{u, i}$ is the predicted score give by user $u$ to item $i$.
\end{itemize}
\hfill\break
\hfill\break
It is calculated as such for the entire system:
\hfill\break
\hfill\break
    \[
       \begin{gathered}
           MAE_{sys} = \sum_{u \in T} \frac{MAE_u}{|T|}
       \end{gathered}
    \]
\hfill\break
\hfill\break
    Where:
\begin{itemize}
    \item $T$ is the test set.
    \item $MAE_u$ is the MAE calculated for user $u$.
\end{itemize}
\hfill\break
\hfill\break
There may be cases in which some items of the test set of the user could not be predicted
\textit{e.g. a CBRS was chosen and items were not present locally, a methodology different from TestRatings was chosen}.
In those cases, the $MAE_u$ formula becomes:
\hfill\break
\hfill\break
    \[
      \begin{gathered}
          MAE_u = \sum_{i \in T_u} \frac{|r_{u,i} - \hat{r}_{u,i}|}{|T_u| - unk}
      \end{gathered}
    \]
\hfill\break
\hfill\break
    Where:
\begin{itemize}
    \item $unk$ (unknown) is the number of items of the user test set that could not be predicted.
\end{itemize}
\hfill\break
\hfill\break
If no items of the user test set have been predicted $|T_u| - unk = 0$, then:
\hfill\break
\hfill\break
    \[
        \begin{gathered}
            MAE_u = NaN
        \end{gathered}
    \]
\hfill\break
\hfill\break
% mae report end___
\BLOCK{endif}
% <------ error metrics ended -------->

% ------------------------------------------- RANKING METRICS STARTED -----------------------------------------------
\BLOCK{if my_dict['metrics'] is defined and
          my_dict['metrics']['NDCG'] is defined}
% Normalized Discounted Cumulative Gain REPORT start
\subsubsection{NDCG}\label{subsubsec:ndcg}
The NDCG abbreviation of Normalized Discounted Cumulative Gain metric is calculated for the \textbf{single user}
by first computing the DCG score using the following formula:
\hfill\break
\hfill\break
    \[
       \begin{gathered}
           DCG_{u}(scores_{u}) = \sum_{r\in scores_{u}}{\frac{f(r)}{log_x(2 + i)}}
       \end{gathered}
    \]
\hfill\break
\hfill\break
    Where:
\begin{itemize}
    \item $scores_{u}$ are the ground truth scores for predicted items, ordered according to the order of said items in the
        ranking for the user $u$.
    \item $f$ is a gain function \textit{linear or exponential, in particular}.
    \item $x$ is the base of the logarithm.
    \item $i$ is the index of the truth score $r$ in the list of scores $scores_{u}$.
\end{itemize}
\hfill\break
\hfill\break
If $f$ is "linear", then the truth score $r$ is returned as is.
Otherwise, in the "exponentia" case, the following formula is applied to $r$:
\hfill\break
\hfill\break
    \[
       \begin{gathered}
           f(r) = 2^{r} - 1
       \end{gathered}
    \]
\hfill\break
\hfill\break
The NDCG for a single user is then calculated using the following formula:
\hfill\break
\hfill\break
    \[
       \begin{gathered}
           NDCG_u(scores_{u}) = \frac{DCG_{u}(scores_{u})}{IDCG_{u}(scores_{u})}
       \end{gathered}
    \]
\hfill\break
\hfill\break
    Where:
\begin{itemize}
    \item $IDCG_{u}$ is the DCG of the ideal ranking for the truth scores.
\end{itemize}
\hfill\break
\hfill\break
So the basic idea is to compare the actual ranking with the ideal one.
Finally, the NDCG of the entire system is calculated instead as such:
\hfill\break
\hfill\break
    \[
       \begin{gathered}
           NDCG_{sys} = \frac{\sum_{u} NDCG_u}{|U|}
       \end{gathered}
    \]
\hfill\break
\hfill\break
    Where:
\begin{itemize}
    \item $NDCG_u$ is the NDCG calculated for user $u$.
    \item $U$ is the set of all users.
\end{itemize}
\hfill\break
\hfill\break
The system average excludes NaN values.
\hfill\break
\hfill\break
% ndcg report end___
\BLOCK{endif}

\BLOCK{if my_dict['metrics'] is defined and
          my_dict['metrics']['NDCGAtK'] is defined}
% NDCG@K REPORT start
\subsubsection{NDCG@k}\label{subsubsec:ndcg-k}
The NDCG@K abbreviation of Normalized Discounted Cumulative Gain at K metric is calculated for the \textbf{single user}
by using the [framework implementation of the NDCG][clayrs.evaluation.NDCG] but considering $scores_{u}$ cut at the
first $k$ predictions.
The K used for the experiment is \VAR{my_dict['metrics']['MRRAtK']['k']|safe_text}.
\hfill\break
\hfill\break
% ndcg@k report end___
\BLOCK{endif}

\BLOCK{if my_dict['metrics'] is defined and
          my_dict['metrics']['MRR'] is defined}
% MRR REPORT start
\subsubsection{MRR}\label{subsubsec:mrr}
The MRR abbreviation of Mean Reciprocal Rank metric is a system-wide metric, so only its result it will be returned
and not those of every user.
MRR is calculated as such:
\hfill\break
\hfill\break
    \[
       \begin{gathered}
           MRR_{sys} = \frac{1}{|Q|}\cdot\sum_{i=1}^{|Q|}\frac{1}{rank(i)}
       \end{gathered}
    \]
\hfill\break
\hfill\break
    Where:
\begin{itemize}
    \item $Q$ is the set of recommendation lists.
    \item $rank(i)$ is the position of the first relevant item in the i-th recommendation list.
\end{itemize}
\hfill\break
\hfill\break
% ATTENTION here we modified with \_ in case change
The MRR metric needs to discern relevant items from the not relevant ones.
To achieve this, one could pass a custom \texttt{relevant\_threshold} parameter that will be applied to every user.
If the rating of an item is $\geq$ \texttt{relevant\_threshold}, then it is considered relevant; otherwise, it is not.
If no \texttt{relevant\_threshold} parameter is passed, then for every user, its mean rating score will be used.
In this experiment, the relevant threshold used is
\VAR{my_dict['metrics']['MRR']['relevant_threshold']|default('no threshold has been setted')|safe_text}.
\hfill\break
\hfill\break
% mrr report end___
\BLOCK{endif}

\BLOCK{if my_dict['metrics'] is defined and
          my_dict['metrics']['MRRAtK'] is defined}
% MRR@K REPORT start
\subsubsection{MRR@K}\label{subsubsec:mrr-k}
The MRR@K abbreviation of Mean Reciprocal Rank at K metric is a system-wide metric, so only its result will be returned
and not those of every user.
MRR@K is calculated as such:
\hfill\break
\hfill\break
    \[
       \begin{gathered}
           MRR@K_{sys} = \frac{1}{|Q|}\cdot\sum_{i=1}^{K}\frac{1}{rank(i)}
       \end{gathered}
    \]
\hfill\break
\hfill\break
    Where:
\begin{itemize}
    \item $K$ is the cutoff parameter.
    \item $Q$ is the set of recommendation lists.
    \item $rank(i)$ is the position of the first relevant item in the i-th recommendation list.
\end{itemize}
\hfill\break
\hfill\break
In this experiment, the relevant threshold used is
\VAR{my_dict['metrics']['MRRAtK']['relevant_threshold']|default('no threshold has been setted')|safe_text}.
\hfill\break
\hfill\break
% mrr@k report end___
\BLOCK{endif}

\BLOCK{if my_dict['metrics'] is defined and
          my_dict['metrics']['MAP'] is defined}
% MAP REPORT start
\subsubsection{MAP}\label{subsubsec:map}
The MAP metric abbreviation of Mean average Precision is a ranking metric computed by first calculating the AP
abbreviation of Average Precision for each user and then taking its mean.
The AP is calculated as such for the \textbf{single user}:
\hfill\break
\hfill\break
    \[
       \begin{gathered}
           AP_u = \frac{1}{m_u}\sum_{i=1}^{N_u}P(i)\cdot rel(i)
       \end{gathered}
    \]
\hfill\break
\hfill\break
    Where:
\begin{itemize}
    \item $m_u$ is the number of relevant items for the user $u$.
    \item $N_u$ is the number of recommended items for the user $u$.
    \item $P(i)$ is the precision computed at cutoff $i$.
    \item $rel(i)$ is an indicator variable that says whether the i-th item is relevant ($rel(i)=1$) or not ($rel(i)=0$).
\end{itemize}
\hfill\break
\hfill\break
After computing the AP for each user, we can compute the MAP for the whole system:
\hfill\break
\hfill\break
    \[
       \begin{gathered}
           MAP_{sys} = \frac{1}{|U|}\sum_{u}AP_u
       \end{gathered}
    \]
\hfill\break
\hfill\break
This metric will return the AP computed for each user in the dataframe containing users results, and the MAP
computed for the whole system in the dataframe containing system results.
In this experiment the MAP has been calculated using a relevant threshold:
\VAR{my_dict['metrics']['MAP']['relevant_threshold']|default('no threshold has been setted')|safe_text}.
\hfill\break
\hfill\break
% map report end___
\BLOCK{endif}

\BLOCK{if my_dict['metrics'] is defined and
          my_dict['metrics']['MAPAtK'] is defined}
% MAP@K REPORT start
\subsubsection{MAP@K}\label{subsubsec:map-k}
The MAP@K metric abbreviation of Mean average Precision At K is a ranking metric computed by first calculating the
AP@K abbreviation of Average Precision At K for each user and then taking its mean.
The AP@K is calculated as such for the \textbf{single user}:
\hfill\break
\hfill\break
    \[
       \begin{gathered}
           AP@K_u = \frac{1}{m_u}\sum_{i=1}^{K}P(i)\cdot rel(i)
       \end{gathered}
    \]
\hfill\break
\hfill\break
    Where:
\begin{itemize}
    \item $m_u$ is the number of relevant items for the user $u$.
    \item $K$ is the cutoff value.
    \item $P(i)$ is the precision computed at cutoff $i$.
    \item $rel(i)$ is an indicator variable that says whether the i-th item is relevant ($rel(i)=1$) or not ($rel(i)=0$).
\end{itemize}
\hfill\break
\hfill\break
After computing the AP@K for each user, we can compute the MAP@K for the whole system:
\hfill\break
\hfill\break
    \[
       \begin{gathered}
           MAP@K_{sys} = \frac{1}{|U|}\sum_{u}AP@K_u
       \end{gathered}
    \]
\hfill\break
\hfill\break
This metric will return the AP@K computed for each user in the dataframe containing users results, and the MAP@K
computed for the whole system in the dataframe containing system results.
\hfill\break
\hfill\break
% map@k report end___
\BLOCK{endif}

\BLOCK{if my_dict['metrics'] is defined and
          my_dict['metrics']['Correlation'] is defined}
% CORRELATION REPORT start
\subsubsection{Correlation}\label{subsubsec:corr}
The Correlation metric calculates the correlation between the ranking of a user and its ideal ranking.
The current correlation methods implemented are:
\begin{itemize}
    \item `pearson`
    \item `kendall`
    \item `spearman`
\end{itemize}
\hfill\break
\hfill\break
Every correlation method is implemented by the pandas library, so refer to its
\href{https://pandas.pydata.org/docs/reference/api/pandas.Series.corr.html}{documentation} for more information.
\hfill\break
\hfill\break
The correlation metric is calculated as such for the \textbf{single user}:
\hfill\break
\hfill\break
    \[
      \begin{gathered}
          Corr_u = Corr(ranking_u, ideal_ranking_u)
      \end{gathered}
    \]
\hfill\break
\hfill\break
    Where:
\begin{itemize}
    \item $ranking_u$ is ranking of the user.
    \item $ideal_ranking_u$ is the ideal ranking for the user.
\end{itemize}
\hfill\break
\hfill\break
The ideal ranking is calculated based on the rating inside the ground truth of the user.
The Correlation metric calculated for the entire system is simply the average of every $Corr$:
\hfill\break
\hfill\break
    \[
      \begin{gathered}
          Corr_{sys} = \frac{\sum_{u} Corr_u}{|U|}
      \end{gathered}
    \]
\hfill\break
\hfill\break
    Where:
\begin{itemize}
    \item $Corr_u$ is the correlation of the user $u$.
    \item $U$ is the set of all users.
\end{itemize}
\hfill\break
\hfill\break
% ATTETNION here there are already \_
The system average excludes NaN values.
It's also possible to specify a cutoff parameter using the \texttt{top\_n} parameter.
If specified, only the first $n$ results of the recommendation list will be used to calculate the correlation.
For this experiment, the settings for the correlation metrics are:
\begin{itemize}
    \item method: \VAR{my_dict['metrics']['Correlation']['method']|default('no method for correlation specified')|safe_text}.
    \item top\_n: \VAR{my_dict['metrics']['Correlation']['top_n']|default('no cutoff setted')|safe_text}.
\end{itemize}
\hfill\break
\hfill\break
% correlation report end___
\BLOCK{endif}
% <------- ranking metrics ended ------->

% -------------------------------------------- FAIRNESS METRICS ------------------------------------------------------
\BLOCK{if my_dict['metrics'] is defined and
          my_dict['metrics']['GiniIndex'] is defined}
% Gini Index REPORT start
\subsubsection{Gini Index}\label{subsubsec:gini}
The Gini Index metric measures inequality in recommendation lists.
It's a system-wide metric, so only its result it will be returned and not those of every user.
The metric is calculated as such:
\hfill\break
\hfill\break
    \[
       \begin{gathered}
           Gini_{sys} = \frac{\sum_i(2i - n - 1)x_i}{n\cdot\sum_i x_i}
       \end{gathered}
    \]
\hfill\break
\hfill\break
    Where:
\begin{itemize}
    \item $n$ is the total number of distinct items that are being recommended.
    \item $x_i$ is the number of times that the item $i$ has been recommended.
\end{itemize}
\hfill\break
\hfill\break
A perfectly equal recommender system should recommend every item the same number of times, in which case the Gini
index would be equal to 0.
The more the recsys is "unequal", the more the Gini Index is closer to 1.
If the \texttt{'top\_n'} parameter is specified, then the Gini index will measure inequality considering only
the first $n$ items of every recommendation list of all users.
For this experiment, the \texttt{top\_n}:
\VAR{my_dict['metrics']['GiniIndex']['top_n']|default('no list of top n item has setted')|safe_text}.
\hfill\break
\hfill\break
% gini index report end___
\BLOCK{endif}

\BLOCK{if my_dict['metrics'] is defined and
          my_dict['metrics']['PredictionCoverage'] is defined}
% PREDICTION COVERAGE REPORT start
\subsubsection{Prediction Coverage}\label{subsubsec:pred_cov}
The Prediction Coverage metric measures in percentage how many distinct items are being recommended in relation
to all available items.
It's a system wise metric, so only its result it will be returned and not those of every user.
The metric is calculated as such:
\hfill\break
\hfill\break
    \[
       \begin{gathered}
           Prediction Coverage_{sys} = (\frac{|I_p|}{|I|})\cdot100
       \end{gathered}
    \]
\hfill\break
\hfill\break
    Where:
\begin{itemize}
    \item $I$ is the set of all available items.
    \item $I_p$ is the set of recommended items.
\end{itemize}
\hfill\break
\hfill\break
The $I$ must be specified through the 'catalog' parameter.
\hfill\break
\hfill\break
% prediction coverage report end___
\BLOCK{endif}

\BLOCK{if my_dict['metrics'] is defined and
          my_dict['metrics']['CatalogCoverage'] is defined}
% CATALOG COVERAGE REPORT start
\subsubsection{Catalog Coverage}\label{subsubsec:cat_cov}
The Catalog Coverage metric measures in percentage how many distinct items are being recommended in relation
to all available items.
It's a system-wide metric, so only its result it will be returned and not those of every user.
It differs from the Prediction Coverage since it allows for different parameters to come into play.
If no parameter is passed then it's a simple Prediction Coverage.
The metric is calculated as such:
\hfill\break
\hfill\break
    \[
        \begin{gathered}
            Catalog Coverage_{sys} = (\frac{|\bigcup_{j=1...N}reclist(u_j)|}{|I|})\cdot100
        \end{gathered}
    \]
\hfill\break
\hfill\break
    Where:
\begin{itemize}
    \item $N$ is the total number of users.
    \item $reclist(u_j)$ is the set of items contained in the recommendation list of user $j$.
    \item $I$ is the set of all available items.
\end{itemize}
\hfill\break
\hfill\break
The $I$ must be specified through the 'catalog' parameter.
The recommendation list of every user ($reclist(u_j)$) can be reduced to the first n parameter with the top-n parameter,
so that catalog coverage is measured considering only the highest ranked items.
With the 'k' parameter one could specify the number of users that will be used to
calculate catalog coverage: k users will be randomly sampled and their recommendation lists will be used.
The formula above becomes:
\hfill\break
\hfill\break
    \[
       \begin{gathered}
         Catalog Coverage_{\text{sys}} = \left(\frac{|\bigcup_{j=1\ldots k} \text{reclist}(u_j)|}{|I|}\right) \cdot 100
       \end{gathered}
    \]
\hfill\break
\hfill\break
    Where:
\begin{itemize}
    \item $k$ is the parameter specified.
\end{itemize}
\hfill\break
\hfill\break
Obviously 'k' $<$ N, else simply recommendation lists of all users will be used.
\hfill\break
\hfill\break
% catalog coverage report end___
\BLOCK{endif}

\BLOCK{if my_dict['metrics'] is defined and
           my_dict['metrics']['DeltaGap'] is defined}
% DELTA GAP REPORT start
\subsubsection{Delta GAP}\label{subsubsec:dg}
The Delta GAP abbreviation of Group Average popularity metric lets you compare the average popularity "requested" by
one or multiple groups of users and the average popularity "obtained" with the recommendation given by the recsys.
It's a system wise metric and results of every group will be returned.
It is calculated as such:
\hfill\break
\hfill\break
    \[
       \begin{gathered}
           \Delta GAP = \frac{recs_GAP - profile_GAP}{profile_GAP}
       \end{gathered}
    \]
\hfill\break
\hfill\break
Users are split into groups based on the user groups parameter, which contains names of the groups as keys,
and percentage of how many user must contain a group as values.
For example:
\begin{itemize}
    \item user groups = \{'popular\_users': 0.3, 'medium\_popular\_users': 0.2, 'low\_popular\_users': 0.5\}
\end{itemize}
\hfill\break
\hfill\break
Every user will be inserted in a group based on how many popular items the user has rated
\textit{in relation to the percentage of users we specified as value in the dictionary}:
\begin{itemize}
    \item users with many popular items will be inserted into the first group.
    \item users with niche items rated will be inserted into one of the last groups.
\end{itemize}
\hfill\break
\hfill\break
In general users are grouped by $Popularity_ratio$ in descending order. $Popularity_ratio$ for a single user $u$
is defined as:
\hfill\break
\hfill\break
    \[
      \begin{gathered}
          Popularity_ratio_u = n_most_popular_items_rated_u / n_items_rated_u
      \end{gathered}
    \]
\hfill\break
\hfill\break
The most popular items are the first \texttt{pop\_percentage} percent of all items ordered in descending order by
popularity.
The popularity of an item is defined as the number of times it is rated in the \texttt{original\_ratings} parameter
divided by the total number of users in the \texttt{original\_ratings}.
\hfill\break
\hfill\break
It can happen that for a particular user of a group no recommendation are available: in that case it will be skipped,
and it won't be considered in the $\Delta GAP$ computation of its group.
In case no user of a group has recs available, a warning will be printed and the whole group won't be considered.
If the 'top\_n' parameter is specified, then $\Delta GAP$ will be calculated considering only the first
n items of every recommendation list of all users
\hfill\break
\hfill\break
% delta gap report end___
\BLOCK{endif}
% <------- fairness metrics end ------>

% ------------------------------------------------ PLOT METRICS STARTED ----------------------------------------------
\BLOCK{if my_dict['metrics'] is defined and
          my_dict['metrics']['LongTailDistr'] is defined}
% LONG TAIL DISTIBUTION REPORT start
\subsubsection{Long Tail Distribution}\label{subsubsec:ltd}
This metric generates the Long Tail Distribution plot and saves it in the output directory with the file name
specified.
The plot can be generated both for the truth set or the predictions set based on the on parameter:
\begin{itemize}
    \item \textbf{on = 'truth'}: in this case the long tail distribution is useful to see which are the most popular items
       the most rated ones.
    \item \textbf{on = 'pred'}: in this case the long tail distribution is useful to see which are the most recommended
        items.
\end{itemize}
\hfill\break
\hfill\break
The plot file will be saved as \texttt{out\_dir/file\_name.format}.
Since multiple splits could be evaluated at once, the overwrite parameter comes into play:
if set to \texttt{False}, files with the same name will be saved as \texttt{file\_name\_(1).format}, \texttt{file\_name\_(2).format}, etc.
so that for every split a plot is generated without overwriting any previously generated files.
\hfill\break
\hfill\break
For this experiment the Long Tail Distribution has been used with the following settings:
\begin{itemize}
    \item on: \VAR{my_dict['metrics']['LongTailDistr']['on']|safe_text}.
    \item format: \VAR{my_dict['metrics']['LongTailDistr']['format']|safe_text}.
    \item overwrite: \VAR{my_dict['metrics']['LongTailDistr']['overwrite']|safe_text}.
\end{itemize}
\hfill\break
\hfill\break
% LTD report end___
\BLOCK{endif}

\BLOCK{if my_dict['metrics'] is defined and
          my_dict['metrics']['PopRatioProfileVsRecs'] is defined}
% PopRatioProfileVsRecs REPORT start
\subsubsection{Pop Ratio Profile Vs Recs}\label{subsubsec:popratio}
This metric generates a plot where users are split into groups and, for every group, a boxplot comparing
profile popularity ratio and recommendations popularity ratio is drawn.
Users are split into groups based on the user groups parameter, which contains names of the groups as keys,
and percentage of how many user must contain a group as values.
For example:
\begin{itemize}
       \item user groups = \{'popular\_users': 0.3, 'medium\_popular\_users': 0.2, 'low\_popular\_users': 0.5\}
\end{itemize}
\hfill\break
\hfill\break
Every user will be inserted in a group based on how many popular items the user has rated
\textit{in relation to the percentage of users we specified as value in the dictionary}:
\begin{itemize}
    \item users with many popular items will be inserted into the first group
    \item users with niche items rated will be inserted into one of the last groups.
\end{itemize}
\hfill\break
\hfill\break
In general users are grouped by $Popularity_ratio$ in descending order. $Popularity_ratio$ for a single user $u$
is defined as:
\hfill\break
\hfill\break
    \[
       \begin{gathered}
           Popularity_ratio_u = n_most_popular_items_rated_u / n_items_rated_u
       \end{gathered}
    \]
\hfill\break
\hfill\break
The most popular items are the first 'pop\_percentage', items of all items ordered in descending order by
popularity.
The popularity of an item is defined as the number of times it is rated in the 'original\_ratings' parameter
divided by the total number of users in the 'original\_ratings'.
It can happen that for a particular user of a group no recommendation are available: in that case it will be skipped,
and it won't be considered in the $Popularity\_ratio$ computation of its group.
In case no user of a group has recs available, a warning will be printed and the whole group won't be considered.
\begin{itemize}
    \item The plot file will be saved as `out\_dir/file\_name.format`
\end{itemize}
\hfill\break
\hfill\break
Since multiple splits could be evaluated at once, the \texttt{overwrite} parameter comes into play: if set to
\texttt{False}, files with the same name will be saved as \texttt{file\_name (1).format}, \texttt{file\_name (2).format},
etc., so that for every split, a plot is generated without overwriting any previously generated files.
Thanks to the \texttt{store\_frame} parameter, it's also possible to store a CSV containing the calculations done in
order to build every boxplot.
It will be saved in the same directory and with the same file name as the plot itself
(but with the .csv format):
\begin{itemize}
    \item The CSV will be saved as \texttt{out\_dir/file\_name.csv}
\end{itemize}
\hfill\break
\hfill\break
%PopRatioProfileVsRecs report end___
\BLOCK{endif}

\BLOCK{if my_dict['metrics'] is defined and
        my_dict['metrics']['PopRecsCorrelation'] is defined}
% PopRecsCorrelation REPORT start
\subsubsection{Pop Recs Correlation}\label{subsubsec:poprc}
This metric generates a plot which has as the X-axis the popularity of each item and as Y-axis the recommendation
frequency, so that it can be easily seen the correlation between popular niche items and how many times are being
recommended.
The popularity of an item is defined as the number of times it is rated in the 'original\_ratings' parameter
divided by the total number of users in the 'original\_ratings'.
\begin{itemize}
    \item The plot file will be saved as out\_dir/file\_name.format'
\end{itemize}
\hfill\break
\hfill\break
Since multiple splits could be evaluated at once, the overwrite parameter comes into play: if set to \texttt{False},
files with the same name will be saved as \texttt{'file\_name\_(1).format'}, \texttt{'file\_name\_(2).format'}, etc.,
so that for every split, a plot is generated without overwriting any previously generated files.
\hfill\break
\hfill\break
There exists cases in which some items are not recommended even once, so in the graph could appear
zero recommendations.
One could change this behaviour thanks to the 'mode' parameter:
\begin{itemize}
    \item \textbf{mode='both'}: Two graphs will be created, the first one containing eventual zero recommendations, the
      second one where zero recommendations are excluded. This additional graph will be stored as
      \texttt{out\_dir/file\_name\_no\_zeros.format} \textit{the string '\_no\_zeros' will be added to the file\_name chosen automatically}.
    \item \textbf{mode='w\_zeros'}: Only a graph containing eventual zero recommendations will be created.
    \item \textbf{mode='no\_zeros'}: Only a graph excluding eventual zero recommendations will be created. The graph will be
      saved as \texttt{out\_dir/file\_name\_no\_zeros.format} \textit{the string '\_no\_zeros' will be added to the file\_name chosen automatically}.
\end{itemize}
\hfill\break
\hfill\break
For this experiment the PopRecsCorrelation has been used with the following settings:
\begin{itemize}
    \item mode: \VAR{my_dict['metrics']['PopRecsCorrelation']['mode']|safe_text}.
    \item format: \VAR{my_dict['metrics']['PopRecsCorrelation']['format']|safe_text}.
    \item overwrite: \VAR{my_dict['metrics']['PopRecsCorrelation']['overwrite']|safe_text}.
\end{itemize}
\hfill\break
\hfill\break
% PopRecsCorrelation report end___
\BLOCK{endif}
% <------- plot metric end ----------->

% <---------------------- closing the only metric section ------------------------->
\BLOCK{endif}

###


\begin{comment}
Author = DIEGO MICCOLI
Alias = Kozen88
Organization = SWAP Research Group UniBa
Date = 27-12-2023

This mini template is not working by itself because there are latex command missing needed
to compile the file and give as output a pdf file, in addition it has been added jinja
statement in order to control the rendering of the latex file with the jinja library, for these
reasons it needs to be used with the other mini chunks in conjunction.
\end{comment}