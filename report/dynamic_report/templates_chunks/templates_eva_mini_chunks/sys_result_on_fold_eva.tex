%! Author = DIEGO MICCOLI
%! Date = 27/12/2023

\usepackage{comment}


###

\BLOCK{if my_dict['sys_results'] is defined}
% ------------------------------- RESULT OF PERFORMACE OF THE SYSTEM ON THE FOLD -------------------------------------
\subsection{X}\label{subsec:X}
In the following table, we present the results of the evaluation~\ref{tab:results_table_X}
\begin{table}[!hbp]\label{tab:results_table_X}
    \centering
  \begin{tabular}{|c|c|}
    \hline
    \textbf{Metric}& \textbf{Value} \\ \hline
    \BLOCK{if my_dict['sys_results']['X']['Precision - macro'] is defined}
    Precision - macro & \VAR{my_dict['sys_results']['X']['Precision - macro']|truncate|safe_text}\\ \hline
    \BLOCK{endif}
    \BLOCK{if my_dict['sys_results']['X']['Precision - micro'] is defined}
    Precision - micro & \VAR{my_dict['sys_results']['X']['Precision - micro']|truncate|safe_text}\\ \hline
    \BLOCK{endif}
    \BLOCK{if my_dict['sys_results']['X']['Recall - macro'] is defined}
    Recall - macro & \VAR{my_dict['sys_results']['X']['Recall - macro']|truncate|safe_text}\\ \hline
    \BLOCK{endif}
    \BLOCK{if my_dict['sys_results']['X']['Recall - micro'] is defined}
    Recall - micro & \VAR{my_dict['sys_results']['X']['Recall - micro']|truncate|safe_text}\\ \hline
    \BLOCK{endif}
    \BLOCK{if my_dict['sys_results']['X']['R-Precision - macro'] is defined}
    R-Precision - macro & \VAR{my_dict['sys_results']['X']['R-Precision - macro']|truncate|safe_text}\\ \hline
    \BLOCK{endif}
    \BLOCK{if my_dict['sys_results']['X']['R-Precision - micro'] is defined}
    R-Precision - micro & \VAR{my_dict['sys_results']['X']['R-Precision - micro']|truncate|safe_text}\\ \hline
    \BLOCK{endif}
    \BLOCK{if my_dict['sys_results']['X']['F1 - macro'] is defined}
    F1 - macro &  \VAR{my_dict['sys_results']['X']['F1 - macro']|truncate|safe_text}\\ \hline
    \BLOCK{endif}
    \BLOCK{if my_dict['sys_results']['X']['F1 - micro'] is defined}
    F1 - micro & \VAR{my_dict['sys_results']['X']['F1 - micro']|truncate|safe_text}\\ \hline
    \BLOCK{endif}
    \BLOCK{if my_dict['sys_results']['X']['NDCG'] is defined}
    NDCG  & \VAR{my_dict['sys_results']['X']['NDCG']|truncate|safe_text}\\ \hline
    \BLOCK{endif}
    \BLOCK{if my_dict['sys_results']['X']['MRR'] is defined}
    MRR  & \VAR{my_dict['sys_results']['X']['MRR']|truncate|safe_text}\\ \hline
    \BLOCK{endif}
    \BLOCK{if my_dict['sys_results']['X']['RMSE'] is defined}
    RMSE & \VAR{my_dict['sys_results']['X']['RMSE']|truncate|safe_text}\\ \hline
    \BLOCK{endif}
    \BLOCK{if my_dict['sys_results']['X']['MSE'] is defined}
    MSE & \VAR{my_dict['sys_results']['X']['MSE']|truncate|safe_text}\\ \hline
    \BLOCK{endif}
    \BLOCK{if my_dict['sys_results']['X']['MAE'] is defined}
    MAE & \VAR{my_dict['sys_results']['X']['MAE']|truncate|safe_text}\\ \hline
    \BLOCK{endif}
    \BLOCK{if my_dict['sys_results']['X']['MRR'] is defined}
    MRR & \VAR{my_dict['sys_results']['X']['MRR']|truncate|safe_text}\\ \hline
    \BLOCK{endif}
    \BLOCK{if my_dict['sys_results']['X']['MAP'] is defined}
    MAP  & \VAR{my_dict['sys_results']['X']['MAP']|truncate|safe_text}\\ \hline
    \BLOCK{endif}
    \BLOCK{if my_dict['sys_results']['X']['Gini'] is defined}
    Gini & \VAR{my_dict['sys_results']['X']['Gini']|truncate|safe_text}\\ \hline
    \BLOCK{endif}
    \BLOCK{if my_dict['sys_results']['X']['PredictionCoverage'] is defined}
    PredictionCoverage & \VAR{my_dict['sys_results']['X']['PredictionCoverage']|truncate|safe_text}\\ \hline
    \BLOCK{endif}
  \end{tabular}
  \caption{Table of the results}
\end{table}
\hfill\break
\hfill\break
% end table of performance on the fold___
\BLOCK{endif}

###


\begin{comment}
Author = DIEGO MICCOLI
Alias = Kozen88
Organization = SWAP Research Group UniBa
Date = 27-12-2023

This mini template is not working by itself because there are latex command missing needed
to compile the file and give as output a pdf file, in addition it has been added jinja
statement in order to control the rendering of the latex file with the jinja library, for these
reasons it needs to be used with the other mini chunks in conjunction.
\end{comment}