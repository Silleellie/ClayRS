%! Author = DIEGO
%! Date = 21/12/2023

% Document
%\begin{document}

 ###

 This exogenous technique expands each content by using as external source the DBPedia ontology.
 It needs the entity of the contents for which a mapping is required (e.g. entity_type=`dbo:Film`)
 and the field of the raw source that will be used for the actual mapping. This technique can be performend
 with four different modalities as follows:

\begin{itemize}
 \item \begin{minipage}
           \textbf{mode='only_retrieved_evaluated'}, all properties from DBPedia will be retrieved but discarding the
            ones with a blank value.
        \end{minipage}
 \item \begin{minipage}
           \textbf{mode='all'}, all properties in DBPedia più all properties in local raw source will be retrieved.
            Local properties will be overwritten by dbpedia values if there's a conflict
        \end{minipage}
 \item \begin{minipage}
           \textbf{mode='all_retrieved'}, all properties in DBPedia *only* will be retrieved
       \end{minipage}
 \item \begin{minipage}
           \textbf{mode='original_retrieved'}, all local properties with their DBPedia value will be retrieved
       \end{minipage}
\end{itemize}

\hfill\break

In this experiment the DBpedia Mapping Technique has been used with the mode:
\VAR{my_dict['exogenous_representations']['DBPediaMappingTechnique']['mode']|safe_text}, the label_field_used is
\VAR{my_dict['exogenous_representations']['DBPediaMappingTechnique']['label_field']|safe_text} and the timeout used
to expand the content with dbpedia is set to
\VAR{my_dict['exogenous_representations']['DBPediaMappingTechnique']['max_timeout']|safe_text}.
\BLOCK{if my_dict['exogenous_representations']['DBPediaMappingTechnique']['prop_as_uri'] is defined and
my_dict['exogenous_representations']['DBPediaMappingTechnique']['prop_as_uri'] == true}
The properties have been returned in their full uri form.
\BLOCK{else}
The properties have been returned in their rdfs:label form.
\BLOCK{endif}

\hfill\break

###

%\end{document}