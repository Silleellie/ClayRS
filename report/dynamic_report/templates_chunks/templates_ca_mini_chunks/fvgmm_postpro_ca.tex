%! Author = DIEGO MICCOLI
%! Date = 27/12/2023

\usepackage{comment}


###

\BLOCK{ if in my_dict['field_representations'] and 'FVGMM' in my_dict['field_representations']['X']['postprocessing']['1'] }
% FVGMM posttprocessing
FVGMM has been used to preprocess data with the following settings:
\begin{itemize}
    \item number of components: \VAR{my_dict['field_representations']['X']['postprocessing']['1']['FVGMM']['n_components']|default('no setting used')|safe_text}
    \item covariance type: \VAR{my_dict['field_representations']['X']['postprocessing']['1']['FVGMM']['covariance_type']|default('no setting used')|safe_text}
    \item tol: \VAR{my_dict['field_representations']['X']['postprocessing']['1']['FVGMM']['tol']|default('no setting used')|safe_text}
    \item reg covar: \VAR{my_dict['field_representations']['X']['postprocessing']['1']['FVGMM']['reg_covar']|default('no setting used')|safe_text}
    \item max iter: \VAR{my_dict['field_representations']['X']['postprocessing']['1']['FVGMM']['max_iter']|default('no setting used')|safe_text}
    \item n init: \VAR{my_dict['field_representations']['X']['postprocessing']['1']['FVGMM']['n_init']|default('no setting used')|safe_text}
    \item init params: \VAR{my_dict['field_representations']['X']['postprocessing']['1']['FVGMM']['init_params']|default('no setting used')|safe_text}
    \item weights init: \VAR{my_dict['field_representations']['X']['postprocessing']['1']['FVGMM']['weights_init']|default('no setting used')|safe_text}
    \item means init: \VAR{my_dict['field_representations']['X']['postprocessing']['1']['FVGMM']['means_init']|default('no setting used')|safe_text}
    \item precisions init: \VAR{my_dict['field_representations']['X']['postprocessing']['1']['FVGMM']['precisions_init']|default('no setting used')|safe_text}
    \item random state: \VAR{my_dict['field_representations']['X']['postprocessing']['1']['FVGMM']['random_state']|default('no setting used')|safe_text}
    \item warm start: \VAR{my_dict['field_representations']['X']['postprocessing']['1']['FVGMM']['warm_start']|default('no setting used')|safe_text}
    \item verbose: \VAR{my_dict['field_representations']['X']['postprocessing']['1']['FVGMM']['verbose']|default('no setting used')|safe_text}
    \item verbose interval: \VAR{my_dict['field_representations']['X']['postprocessing']['1']['FVGMM']['verbose_interval']|default('no setting used')|safe_text}
    \item improved: \VAR{my_dict['field_representations']['X']['postprocessing']['1']['FVGMM']['improved']|default('no setting used')|safe_text}
    \item alpha: \VAR{my_dict['field_representations']['X']['postprocessing']['1']['FVGMM']['alpha']|default('no setting used')|safe_text}
    \item with mean: \VAR{my_dict['field_representations']['X']['postprocessing']['1']['FVGMM']['with_mean']|default('no setting used')|safe_text}
    \item with std: \VAR{my_dict['field_representations']['X']['postprocessing']['1']['FVGMM']['with_std']|default('no setting used')|safe_text}
\end{itemize}
\hfill\break
\hfill\break
\BLOCK{endif}

###



\begin{comment}
Author = DIEGO MICCOLI
Alias = Kozen88
Organization = SWAP Research Group UniBa
Date = 27-12-2023

This mini template is not working by itself because there are latex command missing needed
to compile the file and give as output a pdf file, in addition it has been added jinja
statement in order to control the rendering of the latex file with the jinja library, for these
reasons it needs to be used with the other mini chunks in conjunction.
\end{comment}