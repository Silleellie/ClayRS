%! Author = DIEGO MICCOLI
%! Date = 27/12/2023

\usepackage{comment}


###

\BLOCK{if 'exogenous_representations' in my_dict and
          'DBPediaMappingTechnique' in my_dict['exogenous_representations']}
% DBpedia Mapping Technique start
This exogenous technique expands each content by using as external source the DBPedia ontology.
It needs the entity of the contents for which a mapping is required (e.g. entity\_type=`dbo:Film`)
and the field of the raw source that will be used for the actual mapping.
This technique can be performed with four different modalities as follows:
\begin{itemize}
 \item
       \textbf{mode='only\_retrieved\_evaluated'}, all properties from DBPedia will be retrieved but discarding the
        ones with a blank value.
 \item
       \textbf{mode='all'}, all properties in DBPedia più all properties in local raw source will be retrieved.
        Local properties will be overwritten by dbpedia values if there's a conflict.
 \item
       \textbf{mode='all\_retrieved'}, all properties in DBPedia *only* will be retrieved.
 \item
       \textbf{mode='original\_retrieved'}, all local properties with their DBPedia value will be retrieved.
\end{itemize}
\hfill\break
\hfill\break
In this experiment the DBpedia Mapping Technique has been used with the mode:
\VAR{my_dict['exogenous_representations']['DBPediaMappingTechnique']['mode']|safe_text}, the label_field_used is
\VAR{my_dict['exogenous_representations']['DBPediaMappingTechnique']['label_field']|safe_text} and the timeout used
to expand the content with dbpedia is set to
\VAR{my_dict['exogenous_representations']['DBPediaMappingTechnique']['max_timeout']|safe_text}.
\BLOCK{if my_dict['exogenous_representations']['DBPediaMappingTechnique']['prop_as_uri'] is defined and
my_dict['exogenous_representations']['DBPediaMappingTechnique']['prop_as_uri'] == true}
The properties have been returned in their full uri form.
\BLOCK{else}
The properties have been returned in their rdfs:label form.
\BLOCK{endif}
\hfill\break
\hfill\break
% DBpedia mapping technique end___
\BLOCK{endif}


###



\begin{comment}
Author = DIEGO MICCOLI
Alias = Kozen88
Organization = SWAP Research Group UniBa
Date = 27-12-2023

This mini template is not working by itself because there are latex command missing needed
to compile the file and give as output a pdf file, in addition it has been added jinja
statement in order to control the rendering of the latex file with the jinja library, for these
reasons it needs to be used with the other mini chunks in conjunction.
\end{comment}