
###

\BLOCK{if 'X' in my_dict['field_representations'] and 'Word2DocEmbedding' in my_dict['field_representations']['X'] }
% check Word2DocEmbedding used as representation
The Word2Doc Embedding Technique has been used and
\BLOCK{ set emb_src3 = my_dict.get('field_representations', {}).get('X', {}).get('Word2DocEmbedding', {}).get('embedding_source', None) }
\BLOCK{if emb_src3 is not none}
the embedding source used is
\VAR{my_dict['field_representations']['X']['Word2DocEmbedding']['embedding_source']|safe_text}.
\BLOCK{else}
no embedding source has been specified.
\BLOCK{endif}
\BLOCK{ set ct1 = my_dict.get('field_representations', {}).get('X', {}).get('Word2DocEmbedding', {}).get('combining_technique', None) }
\BLOCK{if ct1 is not none}
Combining technique used is
\VAR{my_dict['field_representations']['X']['Word2DocEmbedding']['combining_technique']|safe_text}.
\BLOCK{else}
No combining technique has been used.
\BLOCK{endif}
\hfill\break
\hfill\break
\BLOCK{endif}

###


\begin{comment}
Author = DIEGO MICCOLI
Alias = Kozen88
Organization = SWAP Research Group UniBa
Date = 27-12-2023

This mini template is not working by itself because there are latex command missing needed
to compile the file and give as output a pdf file, in addition it has been added jinja
statement in order to control the rendering of the latex file with the jinja library, for these
reasons it needs to be used with the other mini chunks in conjunction.
\end{comment}