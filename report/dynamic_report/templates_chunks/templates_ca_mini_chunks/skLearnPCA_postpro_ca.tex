%! Author = DIEGO MICCOLI
%! Date = 27/12/2023

\usepackage{comment}


###

\BLOCK{if in my_dict['field_representations'] and 'SkLearnPCA' in my_dict['field_representations']['X']['postprocessing']['3']}
% SkLearnPCA postprocessing
SkLearnPCA has been used to preprocess data with the following settings:
\begin{itemize}
    \item number of components: \VAR{my_dict['field_representations']['X']['postprocessing']['3']['SkLearnPCA']['n_components']|default('no setting used')|safe_text}
    \item copy: \VAR{my_dict['field_representations']['X']['postprocessing']['3']['SkLearnPCA']['copy']|default('no setting used')|safe_text}
    \item whiten: \VAR{my_dict['field_representations']['X']['postprocessing']['3']['SkLearnPCA']['whiten']|default('no setting used')|safe_text}
    \item svd solver: \VAR{my_dict['field_representations']['X']['postprocessing']['3']['SkLearnPCA']['svd_solver']|default('no setting used')|safe_text}
    \item tol: \VAR{my_dict['field_representations']['X']['postprocessing']['3']['SkLearnPCA']['tol']|default('no setting used')|safe_text}
    \item iterated power: \VAR{my_dict['field_representations']['X']['postprocessing']['3']['SkLearnPCA']['iterated_power']|default('no setting used')|safe_text}
    \item random state: \VAR{my_dict['field_representations']['X']['postprocessing']['3']['SkLearnPCA']['random_state']|default('no setting used')|safe_text}
\end{itemize}
\hfill\break
\hfill\break
\BLOCK{endif}

###



\begin{comment}
Author = DIEGO MICCOLI
Alias = Kozen88
Organization = SWAP Research Group UniBa
Date = 27-12-2023

This mini template is not working by itself because there are latex command missing needed
to compile the file and give as output a pdf file, in addition it has been added jinja
statement in order to control the rendering of the latex file with the jinja library, for these
reasons it needs to be used with the other mini chunks in conjunction.
\end{comment}