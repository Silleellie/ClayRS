%! Author = DIEGO
%! Date = 09/01/2024

\usepackage{comment}

###

% field: X representation subsection
\textbf{\lstinline[style=verbatim-text]| X |} has been represented with the following techniques:
\hfill\break
\hfill\break
% content representation on X
\BLOCK{if my_dict is defined and
      my_dict['field_representations'] is defined and
      'X' in my_dict['field_representations']}

\BLOCK{ set valid_keys = [] }
\begin{itemize}
    \BLOCK{ for key, value in my_dict['field_representations']['X'].items() }
        \BLOCK{ if key != 'preprocessing' and key != 'postprocessing' }
            \BLOCK{ set _ = valid_keys.append(key) }
        \BLOCK{ endif }
    \BLOCK{ endfor }

    \BLOCK{ for valid_key in valid_keys }
        \item
        \verb| \VAR{ valid_key } | used as content representation technique with following parameters:
        \begin{itemize}
            \BLOCK{ for k, v in my_dict['field_representations']['X'][valid_key].items() }
                \item
                \verb| \VAR{ k }: \VAR{ v }|
            \BLOCK{ endfor }
        \end{itemize}
    \BLOCK{ endfor }
\end{itemize}
\hfill\break
\hfill\break
% content_representation_field_ X ____
\BLOCK{endif}

###

\begin{comment}
Author = DIEGO MICCOLI
Alias = Kozen88
Organization = SWAP Research Group UniBa
Date = 27-12-2023

This mini template is not working by itself because there are latex command missing needed
to compile the file and give as output a pdf file, in addition it has been added jinja
statement in order to control the rendering of the latex file with the jinja library, for these
reasons it needs to be used with the other mini chunks in conjunction.
\end{comment}