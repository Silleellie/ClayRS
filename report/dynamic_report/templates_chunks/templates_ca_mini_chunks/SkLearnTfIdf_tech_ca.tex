
###

\BLOCK{if 'X' in my_dict['field_representations'] and 'SkLearnTfIdf' in my_dict['field_representations']['X'] }
% check SkLearnTfIdf used as representation
The SkLearnTfIdf technique has been used.
The technique has been settled with the following parameter:
\begin{itemize}
    \item max df: \VAR{my_dict['field_representations']['X']['SkLearnTfIdf']['max_df']|safe_text}
    \item min df: \VAR{my_dict['field_representations']['X']['SkLearnTfIdf']['min_df']|safe_text}
    \item max features: \VAR{my_dict['field_representations']['X']['SkLearnTfIdf']['max_features']|safe_text}
    \item vocabulary: \VAR{my_dict['field_representations']['X']['SkLearnTfIdf']['vocabulary']|safe_text}
    \item binary: \VAR{my_dict['field_representations']['X']['SkLearnTfIdf']['binary']|safe_text}
    \item norm: \VAR{my_dict['field_representations']['X']['SkLearnTfIdf']['norm']|safe_text}
    \item use idf: \VAR{my_dict['field_representations']['X']['SkLearnTfIdf']['use_idf']|safe_text}
    \item smooth idf: \VAR{my_dict['field_representations']['X']['SkLearnTfIdf']['smooth_idf']|safe_text}
    \item sublinear tf: \VAR{my_dict['field_representations']['X']['SkLearnTfIdf']['sublinear_tf']|safe_text}
\end{itemize}
\hfill\break
\hfill\break
\BLOCK{endif}

###


\begin{comment}
Author = DIEGO MICCOLI
Alias = Kozen88
Organization = SWAP Research Group UniBa
Date = 27-12-2023

This mini template is not working by itself because there are latex command missing needed
to compile the file and give as output a pdf file, in addition it has been added jinja
statement in order to control the rendering of the latex file with the jinja library, for these
reasons it needs to be used with the other mini chunks in conjunction.
\end{comment}