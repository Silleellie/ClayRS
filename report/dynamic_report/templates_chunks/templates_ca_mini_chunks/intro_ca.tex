
###

\BLOCK{if my_dict['source_file'] is defined}
% ---------OPENING CONTENT ANALYZER SECTION ----------------
\section{CONTENT ANALYZER MODULE}\label{sec:ca}
The content analyzer module will deal with raw source document or more in general data which could be
video or audio data and give a rappresentation of these data which will be used by the other two module.
The text data source could be rappresented with exogenous technique or with a specified rappresentation
and each field given could be treated with preprocessing techiniques and postprocessing technique. In
this experiment the following techinques have been used on specific field in order to achieve the
rappresentation wanted:
\hfill\break
\hfill\break
% ------ TECNIQUE USED TO REPPRESENT DATA FIELD ------------

###


\begin{comment}
Author = DIEGO MICCOLI
Alias = Kozen88
Organization = SWAP Research Group UniBa
Date = 27-12-2023

This mini template is not working by itself because there are latex command missing needed
to compile the file and give as output a pdf file, in addition it has been added jinja
statement in order to control the rendering of the latex file with the jinja library, for these
reasons it needs to be used with the other mini chunks in conjunction.
\end{comment}