%! Author = DIEGO MICCOLI
%! Date = 27/12/2023

\usepackage{comment}


###

\BLOCK{if my_dict is defined and
      'field_representations' in my_dict and
      'X' in my_dict['field_representations'] and
      my_dict['field_representations']['X'] is defined and
      'VGGISH' in my_dict['field_representations']['X']}
% VGGISH used as representation
VGGISH technique has been applied to process data with the following settings:
\begin{itemize}
    \item feature layer: \VAR{my_dict['field_representations']['X']['VGGISH']['feature_layer']|default('no setting used')|safe_text}
    \item flatten: \VAR{my_dict['field_representations']['X']['VGGISH']['flatten']|default('no setting used')|safe_text}
    \item device:  \VAR{my_dict['field_representations']['X']['VGGISH']['device']|default('no setting used')|safe_text}
    \item apply on output: \VAR{my_dict['field_representations']['X']['VGGISH']['apply_on_output']|default('no setting used')|safe_text}
    \item contents dirs: \VAR{my_dict['field_representations']['X']['VGGISH']['contents_dirs']|default('no setting used')|safe_text}
    \item time tuple:
    \item max timeout: \VAR{my_dict['field_representations']['X']['VGGISH']['max_timeout']|default('no setting used')|safe_text}
    \item max retries: \VAR{my_dict['field_representations']['X']['VGGISH']['max_retries']|default('no setting used')|safe_text}
    \item max workers: \VAR{my_dict['field_representations']['X']['VGGISH']['max_workers']|default('no setting used')|safe_text}
    \item batch size:  \VAR{my_dict['field_representations']['X']['VGGISH']['batch_size']|default('no setting used')|safe_text}
\end{itemize}
\hfill\break
\hfill\break
% VGGISH usege end___
\BLOCK{endif}

###



\begin{comment}
Author = DIEGO MICCOLI
Alias = Kozen88
Organization = SWAP Research Group UniBa
Date = 27-12-2023

This mini template is not working by itself because there are latex command missing needed
to compile the file and give as output a pdf file, in addition it has been added jinja
statement in order to control the rendering of the latex file with the jinja library, for these
reasons it needs to be used with the other mini chunks in conjunction.
\end{comment}