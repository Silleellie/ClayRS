
###

%! Author = DIEGO
%! Date = 21/12/2023

% Preamble
\documentclass[11pt]{article}

% Packages
\usepackage{amsmath}
\usepackage{hyperref}
\usepackage{comment}
\usepackage{graphicx}

\title{REPORT EXPERIMENT CLAYRS FRAMEWORK}
\author{SWAP research group UniBa}
%\date{27 December 2023}

% -------------------- DOCUMENT: REPORT ON THE EXPERIMENT WITH CLAYRS FRAMEWORK STARTED -------------------------------
\begin{document}

\maketitle
This \LaTeX{} document was generated automatically from yaml files for the purpose of replicability of experiments done with
\href{https://github.com/swapUniba/ClayRS}{ClayRS},
it contains information about the experiment that has been conducted and the results obtained.
The report is divided in 3 principal section dedicated each one for the 3 principal module of the ClayRS framework
and a conclusion section to highlights what have been achieved from the experiment.
\hfill\break
\hfill\break

###



\begin{comment}
Author = DIEGO MICCOLI
Alias = Kozen88
Organization = SWAP Research Group UniBa
Date = 27-12-2023

This mini template is not working by itself because there are latex command missing needed
to compile the file and give as output a pdf file, in addition it has been added jinja
statement in order to control the rendering of the latex file with the jinja library, for these
reasons it needs to be used with the other mini chunks in conjunction.
\end{comment}