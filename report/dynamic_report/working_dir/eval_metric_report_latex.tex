

\BLOCK{if my_dict['metrics'] is defined or my_dict['sys_results'] is defined}
% -------------------------------------- OPENING THE EVALUATION MODULE SECTION ---------------------------------------
\section{Evaluation Module}\label{sec:eva-module}
The \textbf{EvalModel} which is the abbreviation for Evaluation Model has the task of evaluating a recommender system,
using several state-of-the-art metrics, this allows to compare different recommender system and different algorithm of
recommendation and find out which are the strength points and which the weak ones.

\BLOCK{if my_dict['metrics'] is defined}
\subsection{Metrics}\label{subsec:metrics}
% --- Metrics ---
During the experiment a bunch of formal metrics have been performed on the recommendation produced in order to evaluate
the performance of the system.
The metrics used are the followings:
\hfill\break
\hfill\break

% ---------------------------------------- CLASSIFICATION METRICS STARTED --------------------------------------------
\BLOCK{if my_dict['metrics'] is defined and
          my_dict['metrics']['Precision'] is defined}
% Precision report start
\subsubsection{Precision}\label{subsubsec:precision}
In ClayRS, the Precision metric is calculated as such for the \textbf{single user}:
\hfill\break
\hfill\break
    \[
         Precision_u = \frac{tp_u}{tp_u + fp_u}
    \]

% precision report end___
\BLOCK{endif}

\BLOCK{if my_dict['metrics'] is defined and
          my_dict['metrics']['Recall'] is defined}
% Recall report start
\subsubsection{Recall}\label{subsubsec:recall}
The Recall metric is calculated as such for the \textbf{single user}:
\hfill\break
\hfill\break
    \[
        Recall_u = \frac{tp_u}{tp_u + fn_u}
    \]

% recall report end___
\BLOCK{endif}

\BLOCK{if my_dict['metrics'] is defined and
          my_dict['metrics']['FMeasure'] is defined}
%FMeasure report start
\subsubsection{FMeasure}\label{subsubsec:f-meas}
The FMeasure metric combines Precision and Recall into a single metric.
It is calculated as such for the \textbf{single user}:
\hfill\break
\hfill\break
    \[
        FMeasure_u = (1 + \beta^2) \cdot \frac{P_u \cdot R_u}{(\beta^2 \cdot P_u) + R_u}
    \]

% FMeasure report end___
\BLOCK{endif}

\BLOCK{if my_dict['metrics'] is defined and
          my_dict['metrics']['PrecisionAtK'] is defined}
% PRECISION @K REPORT start
\subsubsection{Precision@K}\label{subsubsec:prec-k}
The Precision@K metric is calculated as such for the \textbf{single user}:
\hfill\break
\hfill\break
    \[
        Precision@K_u = \frac{tp@K_u}{tp@K_u + fp@K_u}
    \]

% precision@k report end___
\BLOCK{endif}

\BLOCK{if my_dict['metrics'] is defined and
          my_dict['metrics']['RecallAtK'] is defined}
% RECALL @K REPORT start
\subsubsection{Recall@K}\label{subsubsec:rec-k}
The Recall@K metric is calculated as such for the \textbf{single user}:
\hfill\break
\hfill\break
    \[
        Recall@K_u = \frac{tp@K_u}{tp@K_u + fn@K_u}
    \]

% recall@k report end___
\BLOCK{endif}

\BLOCK{if my_dict['metrics'] is defined and
          my_dict['metrics']['FMeasureAtK'] is defined}
% FMEASURE @K REPORT start
\subsubsection{FMeasure@K}\label{subsubsec:f-meas-k}
The FMeasure@K metric combines Precision@K and Recall@K into a single metric.
It is calculated as such for the \textbf{single user}:
\hfill\break
\hfill\break
    \[
        FMeasure@K_u = (1 + \beta^2) \cdot \frac{P@K_u \cdot R@K_u}{(\beta^2 \cdot P@K_u) + R@K_u}
    \]

% fmeasure@k report end___
\BLOCK{endif}

\BLOCK{if my_dict['metrics'] is defined and
          my_dict['metrics']['RPrecision'] is defined}
% R-PRECISION REPORT start
\subsubsection{R-Precision}\label{subsubsec:r-prec}
The R-Precision metric is calculated as such for the \textbf{single user}:
\hfill\break
\hfill\break
    \[
        R-Precision_u = \frac{tp@R_u}{tp@R_u + fp@R_u}
    \]

% r-precision report end___
\BLOCK{endif}
% <----- classification metrics end ------->

% -------------------------------------------- ERROR METRICS STARTED ------------------------------------------------
\BLOCK{if my_dict['metrics'] is defined and
          my_dict['metrics']['MSE'] is defined}
% MSE REPORT start
\subsubsection{MSE}\label{subsubsec:mse}
The MSE abbreviation Mean Squared Error metric is calculated as such for the \textbf{single user}:
\hfill\break
\hfill\break
    \[
        MSE_u = \sum_{i \in T_u} \frac{(r_{u,i} - \hat{r}_{u,i})^2}{|T_u|}
    \]

% mse report end___
\BLOCK{endif}

\BLOCK{if my_dict['metrics'] is defined and
          my_dict['metrics']['RMSE'] is defined}
% RMSE REPORT start
\subsubsection{RMSE}\label{subsubsec:rmse}
The RMSE (Root Mean Squared Error) metric is calculated as such for the \textbf{single user}:
\hfill\break
\hfill\break
    \[
        RMSE_u = \sqrt{\sum_{i \in T_u} \frac{(r_{u,i} - \hat{r}_{u,i})^2}{|T_u|}}
    \]
\hfill\break

% rmse report end___
\BLOCK{endif}

\BLOCK{if my_dict['metrics'] is defined and
          my_dict['metrics']['MAE'] is defined}
% MAE REPORT start
\subsubsection{MAE}\label{subsubsec:mae}
The MAE abbreviation of Mean Absolute Error metric is calculated as such for the \textbf{single user}:
\hfill\break
\hfill\break
    \[
        MAE_u = \sum_{i \in T_u} \frac{|r_{u,i} - \hat{r}_{u,i}|}{|T_u|}
    \]
\hfill\break

% mae report end___
\BLOCK{endif}
% <------ error metrics ended -------->

% ------------------------------------------- RANKING METRICS STARTED -----------------------------------------------
\BLOCK{if my_dict['metrics'] is defined and
          my_dict['metrics']['NDCG'] is defined}
% Normalized Discounted Cumulative Gain REPORT start
\subsubsection{NDCG}\label{subsubsec:ndcg}
The NDCG abbreviation of Normalized Discounted Cumulative Gain metric is calculated for the \textbf{single user}
by first computing the DCG score using the following formula:
\hfill\break
\hfill\break
    \[
        DCG_{u}(scores_{u}) = \sum_{r\in scores_{u}}{\frac{f(r)}{log_x(2 + i)}}
    \]
\hfill\break
% ndcg report end___
\BLOCK{endif}

\BLOCK{if my_dict['metrics'] is defined and
          my_dict['metrics']['NDCGAtK'] is defined}
% NDCG@K REPORT start
\subsubsection{NDCG@k}\label{subsubsec:ndcg-k}
The NDCG@K abbreviation of Normalized Discounted Cumulative Gain at K metric is calculated for the \textbf{single user}
by using the [framework implementation of the NDCG][clayrs.evaluation.NDCG] but considering $scores_{u}$ cut at the
first $k$ predictions.
The K used for the experiment is \VAR{my_dict['metrics']['MRRAtK']['k']|safe_text}.
\hfill\break

% ndcg@k report end___
\BLOCK{endif}

\BLOCK{if my_dict['metrics'] is defined and
          my_dict['metrics']['MRR'] is defined}
% MRR REPORT start
\subsubsection{MRR}\label{subsubsec:mrr}
The MRR abbreviation of Mean Reciprocal Rank metric is a system-wide metric, so only its result it will be returned
and not those of every user.
MRR is calculated as such:
\hfill\break
\hfill\break
    \[
        MRR_{sys} = \frac{1}{|Q|}\cdot\sum_{i=1}^{|Q|}\frac{1}{rank(i)}
    \]
\hfill\break

% mrr report end___
\BLOCK{endif}

\BLOCK{if my_dict['metrics'] is defined and
          my_dict['metrics']['MRRAtK'] is defined}
% MRR@K REPORT start
\subsubsection{MRR@K}\label{subsubsec:mrr-k}
The MRR@K abbreviation of Mean Reciprocal Rank at K metric is a system-wide metric, so only its result will be returned
and not those of every user.
MRR@K is calculated as such:
\hfill\break
\hfill\break
    \[
        MRR@K_{sys} = \frac{1}{|Q|}\cdot\sum_{i=1}^{K}\frac{1}{rank(i)}
    \]
\hfill\break

% mrr@k report end___
\BLOCK{endif}

\BLOCK{if my_dict['metrics'] is defined and
          my_dict['metrics']['MAP'] is defined}
% MAP REPORT start
\subsubsection{MAP}\label{subsubsec:map}
The MAP metric abbreviation of Mean average Precision is a ranking metric computed by first calculating the AP
abbreviation of Average Precision for each user and then taking its mean.
The AP is calculated as such for the \textbf{single user}:
\hfill\break
\hfill\break
    \[
        AP_u = \frac{1}{m_u}\sum_{i=1}^{N_u}P(i)\cdot rel(i)
    \]
\hfill\break

% map report end___
\BLOCK{endif}

\BLOCK{if my_dict['metrics'] is defined and
          my_dict['metrics']['MAPAtK'] is defined}
% MAP@K REPORT start
\subsubsection{MAP@K}\label{subsubsec:map-k}
The MAP@K metric abbreviation of Mean average Precision At K is a ranking metric computed by first calculating the
AP@K abbreviation of Average Precision At K for each user and then taking its mean.
The AP@K is calculated as such for the \textbf{single user}:
\hfill\break
\hfill\break
    \[
        AP@K_u = \frac{1}{m_u}\sum_{i=1}^{K}P(i)\cdot rel(i)
    \]
\hfill\break

% map@k report end___
\BLOCK{endif}

\BLOCK{if my_dict['metrics'] is defined and
          my_dict['metrics']['Correlation'] is defined}
% CORRELATION REPORT start
\subsubsection{Correlation}\label{subsubsec:corr}
The Correlation metric calculates the correlation between the ranking of a user and its ideal ranking.
The current correlation methods implemented are:
\begin{itemize}
    \item `pearson`
    \item `kendall`
    \item `spearman`
\end{itemize}
\hfill\break
\hfill\break
Every correlation method is implemented by the pandas library, so refer to its
\href{https://pandas.pydata.org/docs/reference/api/pandas.Series.corr.html}{documentation} for more information.
\hfill\break
\hfill\break
The correlation metric is calculated as such for the \textbf{single user}:
\hfill\break
\hfill\break
    \[
        Corr_u = Corr(ranking_u, ideal_ranking_u)
    \]
\hfill\break

% correlation report end___
\BLOCK{endif}
% <------- ranking metrics ended ------->

% -------------------------------------------- FAIRNESS METRICS ------------------------------------------------------
\BLOCK{if my_dict['metrics'] is defined and
          my_dict['metrics']['GiniIndex'] is defined}
% Gini Index REPORT start
\subsubsection{Gini Index}\label{subsubsec:gini}
The Gini Index metric measures inequality in recommendation lists.
It's a system-wide metric, so only its result it will be returned and not those of every user.
The metric is calculated as such:
\hfill\break
\hfill\break
    \[
        Gini_{sys} = \frac{\sum_i(2i - n - 1)x_i}{n\cdot\sum_i x_i}
    \]
\hfill\break

% gini index report end___
\BLOCK{endif}

\BLOCK{if my_dict['metrics'] is defined and
          my_dict['metrics']['PredictionCoverage'] is defined}
% PREDICTION COVERAGE REPORT start
\subsubsection{Prediction Coverage}\label{subsubsec:pred_cov}
The Prediction Coverage metric measures in percentage how many distinct items are being recommended in relation
to all available items.
It's a system wise metric, so only its result it will be returned and not those of every user.
The metric is calculated as such:
\hfill\break
\hfill\break
    \[
         Prediction Coverage_{sys} = (\frac{|I_p|}{|I|})\cdot100
    \]
\hfill\break

% prediction coverage report end___
\BLOCK{endif}

\BLOCK{if my_dict['metrics'] is defined and
          my_dict['metrics']['CatalogCoverage'] is defined}
% CATALOG COVERAGE REPORT start
\subsubsection{Catalog Coverage}\label{subsubsec:cat_cov}
The Catalog Coverage metric measures in percentage how many distinct items are being recommended in relation
to all available items.
It's a system-wide metric, so only its result it will be returned and not those of every user.
It differs from the Prediction Coverage since it allows for different parameters to come into play.
If no parameter is passed then it's a simple Prediction Coverage.
The metric is calculated as such:
\hfill\break
\hfill\break
    \[
         Catalog Coverage_{sys} = (\frac{|\bigcup_{j=1...N}reclist(u_j)|}{|I|})\cdot100
    \]
\hfill\break

% catalog coverage report end___
\BLOCK{endif}

\BLOCK{if my_dict['metrics'] is defined and
           my_dict['metrics']['DeltaGap'] is defined}
% DELTA GAP REPORT start
\subsubsection{Delta GAP}\label{subsubsec:dg}
The Delta GAP abbreviation of Group Average popularity metric lets you compare the average popularity "requested" by
one or multiple groups of users and the average popularity "obtained" with the recommendation given by the recsys.
It's a system wise metric and results of every group will be returned.
It is calculated as such:
\hfill\break
\hfill\break
    \[
        \Delta GAP = \frac{recs_GAP - profile_GAP}{profile_GAP}
    \]
\hfill\break

% delta gap report end___
\BLOCK{endif}
% <------- fairness metrics end ------>

% ------------------------------------------------ PLOT METRICS STARTED ----------------------------------------------
\BLOCK{if my_dict['metrics'] is defined and
          my_dict['metrics']['LongTailDistr'] is defined}
% LONG TAIL DISTIBUTION REPORT start
\subsubsection{Long Tail Distribution}\label{subsubsec:ltd}
This metric generates the Long Tail Distribution plot and saves it in the output directory with the file name
specified.
The plot can be generated both for the truth set or the predictions set based on the on parameter:
\begin{itemize}
    \item \textbf{on = 'truth'}: in this case the long tail distribution is useful to see which are the most popular items
       the most rated ones.
    \item \textbf{on = 'pred'}: in this case the long tail distribution is useful to see which are the most recommended
        items.
\end{itemize}

\hfill\break
For this experiment the Long Tail Distribution has been used with the following settings:
\begin{itemize}
    \item on: \VAR{my_dict['metrics']['LongTailDistr']['on']|safe_text}.
    \item format: \VAR{my_dict['metrics']['LongTailDistr']['format']|safe_text}.
    \item overwrite: \VAR{my_dict['metrics']['LongTailDistr']['overwrite']|safe_text}.
\end{itemize}
\hfill\break

% LTD report end___
\BLOCK{endif}

\BLOCK{if my_dict['metrics'] is defined and
          my_dict['metrics']['PopRatioProfileVsRecs'] is defined}
% PopRatioProfileVsRecs REPORT start
\subsubsection{Pop Ratio Profile Vs Recs}\label{subsubsec:popratio}
This metric generates a plot where users are split into groups and, for every group, a boxplot comparing
profile popularity ratio and recommendations popularity ratio is drawn.
Users are split into groups based on the user groups parameter, which contains names of the groups as keys,
and percentage of how many user must contain a group as values.
For example:
\begin{itemize}
       \item user groups = \{'popular\_users': 0.3, 'medium\_popular\_users': 0.2, 'low\_popular\_users': 0.5\}
\end{itemize}
\hfill\break

%PopRatioProfileVsRecs report end___
\BLOCK{endif}

\BLOCK{if my_dict['metrics'] is defined and
        my_dict['metrics']['PopRecsCorrelation'] is defined}
% PopRecsCorrelation REPORT start
\subsubsection{Pop Recs Correlation}\label{subsubsec:poprc}
This metric generates a plot which has as the X-axis the popularity of each item and as Y-axis the recommendation
frequency, so that it can be easily seen the correlation between popular niche items and how many times are being
recommended. \\
For this experiment the PopRecsCorrelation has been used with the following settings:
\begin{itemize}
    \item mode: \VAR{my_dict['metrics']['PopRecsCorrelation']['mode']|safe_text}.
    \item format: \VAR{my_dict['metrics']['PopRecsCorrelation']['format']|safe_text}.
    \item overwrite: \VAR{my_dict['metrics']['PopRecsCorrelation']['overwrite']|safe_text}.
\end{itemize}
\hfill\break

% PopRecsCorrelation report end___
\BLOCK{endif}
% <------- plot metric end ----------->

% <---------------------- closing the only metric section ------------------------->
\BLOCK{endif}



% In case of eval module is not used
\BLOCK{else}
The module of the evaluation to control and compare the performance of the recommender
system has not been used during this experiment.
\hfill\break
\hfill\break
% <-------------------- closing evaluation module section --------------------------->
\BLOCK{endif}

