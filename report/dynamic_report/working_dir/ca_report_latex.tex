

\BLOCK{if my_dict['source_file'] is defined}
% ----------------------------------------- OPENING CONTENT ANALYZER SECTION -----------------------------------------
\section{Content Analyzer Module}\label{sec:ca}
The content analyzer module will deal with raw source document or more in general data which could be
video or audio data and give a representation of these data which will be used by the other two module.
The text data source could be represented with exogenous technique or with a specified representation
and each field given could be treated with preprocessing techniques and postprocessing technique.
In this experiment the following techniques have been used on specific field in order to achieve the
representation wanted:
\hfill\break
\hfill\break
% --- TECNIQUE USED TO REPPRESENT DATA FIELD ---


\begin{itemize}
    \item \textbf{fields used}:  idx0, plot0, plot1, plot10, plot11, plot2, plot3, plot4, plot5, plot6, plot7, plot8, plot9, video\_path0, video\_path1, video\_path2, video\_path3, video\_path4

    \item \textbf{fields representation}:  FromNPY, OriginalData, WhooshTfIdf, SkLearnTfIdf, WordEmbeddingTechnique, SentenceEmbeddingTechnique, DocumentEmbeddingTechnique, Word2SentenceEmbedding, Word2DocEmbedding, Sentence2DocEmbedding, PyWSDSynsetDocumentFrequency, SkImageHogDescriptor, MFCC, VGGISH, PytorchImageModels, TorchVisionVideoModels

    \item \textbf{fields preprocessing}:  Spacy, NLTK, Ekphrasis, TorchUniformTemporalSubSampler, ConvertToMono, TorchResample, Lambda, Normalize, CenterCrop, Resize, ClipSampler

    \item \textbf{fields postprocessing}:  FVGMM, VLADGMM, SkLearnPCA
\end{itemize}


\BLOCK{else}
% In case of the content analyzer is not used
For this experiment the module of the content analyzer has not been used.
\hfill\break
\hfill\break
% <----------  closing the controll block for the content analyzer section ----------->
\BLOCK{endif}



\hfill\break
% subsection of Dataset identification
\subsection{Dataset used and its statistics}
The dataset used for the experiment is  \lstinline[style=verbatim-text]| 1000K data video movie |,
\BLOCK{ if my_dict is defined and my_dict['source_file'] is defined }
its source file is \lstinline[style=verbatim-text]| \VAR{ my_dict['source_file'] } | ,\BLOCK{endif}
and the content used during the experiment:
\BLOCK{ if my_dict is defined and my_dict['id_each_content'] is defined and my_dict['id_each_content'] }
\BLOCK{ for value in my_dict['id_each_content'] }
     \VAR{ value|safe_text },
\BLOCK{ endfor }.
\BLOCK{endif}


\BLOCK{ if my_dict is defined and my_dict['interactions'] is defined }
% --- DATA STATS ---
The statistics of the dataset used are reported in the following table:~\ref{tab:dataset_table}:
\begin{table}[ht]
    \centering
  \begin{tabular}{|c|c|}
    \hline
    \textbf{Parameter}& \textbf{Value} \\ \hline
    n\_users  & \VAR{my_dict['interactions']['n_users']|default('no users')|safe_text}\\ \hline
    n\_items  & \VAR{my_dict['interactions']['n_items']|default('no items')|safe_text}\\ \hline
    total\_interactions  & \VAR{my_dict['interactions']['total_interactions']|safe_text}\\ \hline
    min\_score  & \VAR{my_dict['interactions']['min_score']|truncate|safe_text}\\ \hline
    max\_score  & \VAR{my_dict['interactions']['max_score']|truncate|safe_text}\\ \hline
    mean\_score  & \VAR{my_dict['interactions']['mean_score']|truncate|safe_text}\\ \hline
    sparsity  & \VAR{my_dict['interactions']['sparsity']|truncate|safe_text}\\ \hline
  \end{tabular}
   \caption{Stats on the dataset}\label{tab:dataset_table}
\end{table}
\BLOCK{ else }
There are no statistics on the dataset to show.
\BLOCK{ endif }

\hfill\break



% ------------------------------ START SUBSECTION OF PARTITIONING OF RECSYS --------------------------------------------
% subsection of the splitting technique used, referred to partition protocol
\subsection{Data splitting technique}\label{subsec:partitioning}
\BLOCK{if my_dict['partitioning'] is defined and
        my_dict['partitioning']['KFoldPartitioning'] is defined}
% KFOLD PARTITIONING TECNIQUE
K-fold cross-validation is a technique used in machine learning to assess the performance of a predictive model.
The basic idea is to divide the dataset into K subsets, or folds.
The model is then trained K times, each time using K-1 folds for training and the remaining fold for validation.
This process is repeated K times, with a different fold used as the validation set in each iteration.
\hfill\break
The KFoldPartitioning has been used with the following setting:
\hfill\break
\BLOCK{if my_dict.get('partitioning', {}).get('KFoldPartitioning', {}).get('shuffle') == True}
The data has been shuffled before being split into batches.
\BLOCK{endif}
The partitioning technique has been executed with the following settings:
\begin{itemize}
    \item number of splits: \VAR{my_dict['partitioning']['KFoldPartitioning']['n_splits']}
    \item shuffle: \VAR{my_dict['partitioning']['KFoldPartitioning']['shuffle']}
    \item random state: \VAR{my_dict['partitioning']['KFoldPartitioning']['random_state']|default('no random state applied')}
    \item skip user error: \VAR{my_dict['partitioning']['KFoldPartitioning']['skip_user_error']|default('no setted')}
\end{itemize}
\hfill\break
% KFOLD PARTITIONING TECNIQUE ended
\BLOCK{endif}

\BLOCK{if my_dict['partitioning'] is defined and
        my_dict['partitioning']['HoldOutPartitioning'] is defined}
%  HOLD-OUT PARTIONING TECNIQUE
The partitioning used is the Hold-Out Partitioning.
This approach splits the dataset in use into a train set and a test set.
The training set is what the model is trained on, and the test set is used to see how
well the model will perform on new, unseen data.
\hfill\break
The train set size of this experiment is the \VAR{my_dict['partitioning']['HoldOutPartitioning']['train_set_size'] * 100}\%
of the original dataset, while the test set is the remaining \VAR{(100 - (my_dict['partitioning']['HoldOutPartitioning']['train_set_size'] * 100))}\%.
\hfill\break
\BLOCK{ if my_dict.get('partitioning', {}).get('HoldOutPartitioning', {}).get('shuffle') == True }
The data has been shuffled before being split into batches.
\BLOCK{endif}
\hfill\break
%  HOLD-OUT PARTIONING TECNIQUE ended
\BLOCK{endif}
% end partitioning section___________
