

\BLOCK{if my_dict['source_file'] is defined}
% ----------------------------------------- OPENING CONTENT ANALYZER SECTION -----------------------------------------
\section{Content Analyzer Module}\label{sec:ca}
The content analyzer module will deal with raw source document or more in general data which could be
video or audio data and give a representation of these data which will be used by the other two module.
The text data source could be represented with exogenous technique or with a specified representation
and each field given could be treated with preprocessing techniques and postprocessing technique.
In this experiment the following techniques have been used on specific field in order to achieve the
representation wanted:
\hfill\break
\hfill\break
% --- TECNIQUE USED TO REPPRESENT DATA FIELD ---



% field: idx0 representation subsection
\textbf{\lstinline[style=verbatim-text]| idx0 |} has been represented with the following techniques:
% content representation on idx0
\BLOCK{if my_dict is defined and
      my_dict['field_representations'] is defined and
      'idx0' in my_dict['field_representations']}

\BLOCK{ set valid_keys = [] }
\begin{itemize}
    \BLOCK{ for key, value in my_dict['field_representations']['idx0'].items() }
        \BLOCK{ if key != 'preprocessing' and key != 'postprocessing' }
            \BLOCK{ set _ = valid_keys.append(key) }
        \BLOCK{ endif }
    \BLOCK{ endfor }

    \BLOCK{ for valid_key in valid_keys }
        \item
        \verb| \VAR{ valid_key } | used as content representation technique.
    \BLOCK{ endfor }
\end{itemize}

% content_representation_field_ idx0 ____
\BLOCK{endif}



% preprocessing on idx0
\BLOCK{ if my_dict is defined and
      my_dict['field_representations'] is defined and
      'idx0' in my_dict['field_representations'] and
      'preprocessing' in my_dict['field_representations']['idx0'] and
      my_dict['field_representations']['idx0']['preprocessing'] is defined and
      my_dict['field_representations']['idx0']['preprocessing'] is not none }
    % dictionary 'preprocessing' not empty
\begin{itemize}
\BLOCK{ for key, value in my_dict['field_representations']['idx0']['preprocessing'].items() }
    \item
     \verb| \VAR{ key } | used as preprocessing technique.
\BLOCK{ endfor }
\end{itemize}
\BLOCK{else}
No preprocessing techniques have been used to preprocess data \lstinline[style=verbatim-text]| idx0 | has been represented with the following techniques:
 during the experiment.
\BLOCK{endif}
% _preprocessing on idx0 ____




\BLOCK{if my_dict is defined and
      my_dict['field_representations'] is defined and
      'idx0' in my_dict['field_representations'] and
      'postprocessing' in my_dict['field_representations']['idx0'] and
      my_dict['field_representations']['idx0']['postprocessing'] is defined and
      my_dict['field_representations']['idx0']['postprocessing'] is not none}
    % dictionary 'preprocessing' not empty
On th field \verb| idx0 | have been applied the following postprocessing techniques:
\BLOCK{ for key, value in my_dict['field_representations']['idx0']['postprocessing'].items() }
      \BLOCK{ for k, v in value.items() }
            \verb| \VAR{ k } | ,
      \BLOCK{ endfor }
\BLOCK{ endfor }
\BLOCK{else}
% This block is rendered if no post processing technique has been applied
No postprocessing techniques have been used on the data \lstinline[style=verbatim-text]| idx0 | in this experiment.
\BLOCK{endif}
\hfill\break

% postprocessing_on_ idx0 ___



% field: plot0 representation subsection
\textbf{\lstinline[style=verbatim-text]| plot0 |} has been represented with the following techniques:
% content representation on plot0
\BLOCK{if my_dict is defined and
      my_dict['field_representations'] is defined and
      'plot0' in my_dict['field_representations']}

\BLOCK{ set valid_keys = [] }
\begin{itemize}
    \BLOCK{ for key, value in my_dict['field_representations']['plot0'].items() }
        \BLOCK{ if key != 'preprocessing' and key != 'postprocessing' }
            \BLOCK{ set _ = valid_keys.append(key) }
        \BLOCK{ endif }
    \BLOCK{ endfor }

    \BLOCK{ for valid_key in valid_keys }
        \item
        \verb| \VAR{ valid_key } | used as content representation technique.
    \BLOCK{ endfor }
\end{itemize}

% content_representation_field_ plot0 ____
\BLOCK{endif}



% preprocessing on plot0
\BLOCK{ if my_dict is defined and
      my_dict['field_representations'] is defined and
      'plot0' in my_dict['field_representations'] and
      'preprocessing' in my_dict['field_representations']['plot0'] and
      my_dict['field_representations']['plot0']['preprocessing'] is defined and
      my_dict['field_representations']['plot0']['preprocessing'] is not none }
    % dictionary 'preprocessing' not empty
\begin{itemize}
\BLOCK{ for key, value in my_dict['field_representations']['plot0']['preprocessing'].items() }
    \item
     \verb| \VAR{ key } | used as preprocessing technique.
\BLOCK{ endfor }
\end{itemize}
\BLOCK{else}
No preprocessing techniques have been used to preprocess data \lstinline[style=verbatim-text]| plot0 | has been represented with the following techniques:
 during the experiment.
\BLOCK{endif}
% _preprocessing on plot0 ____




\BLOCK{if my_dict is defined and
      my_dict['field_representations'] is defined and
      'plot0' in my_dict['field_representations'] and
      'postprocessing' in my_dict['field_representations']['plot0'] and
      my_dict['field_representations']['plot0']['postprocessing'] is defined and
      my_dict['field_representations']['plot0']['postprocessing'] is not none}
    % dictionary 'preprocessing' not empty
On th field \verb| plot0 | have been applied the following postprocessing techniques:
\BLOCK{ for key, value in my_dict['field_representations']['plot0']['postprocessing'].items() }
      \BLOCK{ for k, v in value.items() }
            \verb| \VAR{ k } | ,
      \BLOCK{ endfor }
\BLOCK{ endfor }
\BLOCK{else}
% This block is rendered if no post processing technique has been applied
No postprocessing techniques have been used on the data \lstinline[style=verbatim-text]| plot0 | in this experiment.
\BLOCK{endif}
\hfill\break

% postprocessing_on_ plot0 ___



% field: plot1 representation subsection
\textbf{\lstinline[style=verbatim-text]| plot1 |} has been represented with the following techniques:
% content representation on plot1
\BLOCK{if my_dict is defined and
      my_dict['field_representations'] is defined and
      'plot1' in my_dict['field_representations']}

\BLOCK{ set valid_keys = [] }
\begin{itemize}
    \BLOCK{ for key, value in my_dict['field_representations']['plot1'].items() }
        \BLOCK{ if key != 'preprocessing' and key != 'postprocessing' }
            \BLOCK{ set _ = valid_keys.append(key) }
        \BLOCK{ endif }
    \BLOCK{ endfor }

    \BLOCK{ for valid_key in valid_keys }
        \item
        \verb| \VAR{ valid_key } | used as content representation technique.
    \BLOCK{ endfor }
\end{itemize}

% content_representation_field_ plot1 ____
\BLOCK{endif}



% preprocessing on plot1
\BLOCK{ if my_dict is defined and
      my_dict['field_representations'] is defined and
      'plot1' in my_dict['field_representations'] and
      'preprocessing' in my_dict['field_representations']['plot1'] and
      my_dict['field_representations']['plot1']['preprocessing'] is defined and
      my_dict['field_representations']['plot1']['preprocessing'] is not none }
    % dictionary 'preprocessing' not empty
\begin{itemize}
\BLOCK{ for key, value in my_dict['field_representations']['plot1']['preprocessing'].items() }
    \item
     \verb| \VAR{ key } | used as preprocessing technique.
\BLOCK{ endfor }
\end{itemize}
\BLOCK{else}
No preprocessing techniques have been used to preprocess data \lstinline[style=verbatim-text]| plot1 | has been represented with the following techniques:
 during the experiment.
\BLOCK{endif}
% _preprocessing on plot1 ____




\BLOCK{if my_dict is defined and
      my_dict['field_representations'] is defined and
      'plot1' in my_dict['field_representations'] and
      'postprocessing' in my_dict['field_representations']['plot1'] and
      my_dict['field_representations']['plot1']['postprocessing'] is defined and
      my_dict['field_representations']['plot1']['postprocessing'] is not none}
    % dictionary 'preprocessing' not empty
On th field \verb| plot1 | have been applied the following postprocessing techniques:
\BLOCK{ for key, value in my_dict['field_representations']['plot1']['postprocessing'].items() }
      \BLOCK{ for k, v in value.items() }
            \verb| \VAR{ k } | ,
      \BLOCK{ endfor }
\BLOCK{ endfor }
\BLOCK{else}
% This block is rendered if no post processing technique has been applied
No postprocessing techniques have been used on the data \lstinline[style=verbatim-text]| plot1 | in this experiment.
\BLOCK{endif}
\hfill\break

% postprocessing_on_ plot1 ___



% field: plot10 representation subsection
\textbf{\lstinline[style=verbatim-text]| plot10 |} has been represented with the following techniques:
% content representation on plot10
\BLOCK{if my_dict is defined and
      my_dict['field_representations'] is defined and
      'plot10' in my_dict['field_representations']}

\BLOCK{ set valid_keys = [] }
\begin{itemize}
    \BLOCK{ for key, value in my_dict['field_representations']['plot10'].items() }
        \BLOCK{ if key != 'preprocessing' and key != 'postprocessing' }
            \BLOCK{ set _ = valid_keys.append(key) }
        \BLOCK{ endif }
    \BLOCK{ endfor }

    \BLOCK{ for valid_key in valid_keys }
        \item
        \verb| \VAR{ valid_key } | used as content representation technique.
    \BLOCK{ endfor }
\end{itemize}

% content_representation_field_ plot10 ____
\BLOCK{endif}



% preprocessing on plot10
\BLOCK{ if my_dict is defined and
      my_dict['field_representations'] is defined and
      'plot10' in my_dict['field_representations'] and
      'preprocessing' in my_dict['field_representations']['plot10'] and
      my_dict['field_representations']['plot10']['preprocessing'] is defined and
      my_dict['field_representations']['plot10']['preprocessing'] is not none }
    % dictionary 'preprocessing' not empty
\begin{itemize}
\BLOCK{ for key, value in my_dict['field_representations']['plot10']['preprocessing'].items() }
    \item
     \verb| \VAR{ key } | used as preprocessing technique.
\BLOCK{ endfor }
\end{itemize}
\BLOCK{else}
No preprocessing techniques have been used to preprocess data \lstinline[style=verbatim-text]| plot10 | has been represented with the following techniques:
 during the experiment.
\BLOCK{endif}
% _preprocessing on plot10 ____




\BLOCK{if my_dict is defined and
      my_dict['field_representations'] is defined and
      'plot10' in my_dict['field_representations'] and
      'postprocessing' in my_dict['field_representations']['plot10'] and
      my_dict['field_representations']['plot10']['postprocessing'] is defined and
      my_dict['field_representations']['plot10']['postprocessing'] is not none}
    % dictionary 'preprocessing' not empty
On th field \verb| plot10 | have been applied the following postprocessing techniques:
\BLOCK{ for key, value in my_dict['field_representations']['plot10']['postprocessing'].items() }
      \BLOCK{ for k, v in value.items() }
            \verb| \VAR{ k } | ,
      \BLOCK{ endfor }
\BLOCK{ endfor }
\BLOCK{else}
% This block is rendered if no post processing technique has been applied
No postprocessing techniques have been used on the data \lstinline[style=verbatim-text]| plot10 | in this experiment.
\BLOCK{endif}
\hfill\break

% postprocessing_on_ plot10 ___



% field: plot11 representation subsection
\textbf{\lstinline[style=verbatim-text]| plot11 |} has been represented with the following techniques:
% content representation on plot11
\BLOCK{if my_dict is defined and
      my_dict['field_representations'] is defined and
      'plot11' in my_dict['field_representations']}

\BLOCK{ set valid_keys = [] }
\begin{itemize}
    \BLOCK{ for key, value in my_dict['field_representations']['plot11'].items() }
        \BLOCK{ if key != 'preprocessing' and key != 'postprocessing' }
            \BLOCK{ set _ = valid_keys.append(key) }
        \BLOCK{ endif }
    \BLOCK{ endfor }

    \BLOCK{ for valid_key in valid_keys }
        \item
        \verb| \VAR{ valid_key } | used as content representation technique.
    \BLOCK{ endfor }
\end{itemize}

% content_representation_field_ plot11 ____
\BLOCK{endif}



% preprocessing on plot11
\BLOCK{ if my_dict is defined and
      my_dict['field_representations'] is defined and
      'plot11' in my_dict['field_representations'] and
      'preprocessing' in my_dict['field_representations']['plot11'] and
      my_dict['field_representations']['plot11']['preprocessing'] is defined and
      my_dict['field_representations']['plot11']['preprocessing'] is not none }
    % dictionary 'preprocessing' not empty
\begin{itemize}
\BLOCK{ for key, value in my_dict['field_representations']['plot11']['preprocessing'].items() }
    \item
     \verb| \VAR{ key } | used as preprocessing technique.
\BLOCK{ endfor }
\end{itemize}
\BLOCK{else}
No preprocessing techniques have been used to preprocess data \lstinline[style=verbatim-text]| plot11 | has been represented with the following techniques:
 during the experiment.
\BLOCK{endif}
% _preprocessing on plot11 ____




\BLOCK{if my_dict is defined and
      my_dict['field_representations'] is defined and
      'plot11' in my_dict['field_representations'] and
      'postprocessing' in my_dict['field_representations']['plot11'] and
      my_dict['field_representations']['plot11']['postprocessing'] is defined and
      my_dict['field_representations']['plot11']['postprocessing'] is not none}
    % dictionary 'preprocessing' not empty
On th field \verb| plot11 | have been applied the following postprocessing techniques:
\BLOCK{ for key, value in my_dict['field_representations']['plot11']['postprocessing'].items() }
      \BLOCK{ for k, v in value.items() }
            \verb| \VAR{ k } | ,
      \BLOCK{ endfor }
\BLOCK{ endfor }
\BLOCK{else}
% This block is rendered if no post processing technique has been applied
No postprocessing techniques have been used on the data \lstinline[style=verbatim-text]| plot11 | in this experiment.
\BLOCK{endif}
\hfill\break

% postprocessing_on_ plot11 ___



% field: plot2 representation subsection
\textbf{\lstinline[style=verbatim-text]| plot2 |} has been represented with the following techniques:
% content representation on plot2
\BLOCK{if my_dict is defined and
      my_dict['field_representations'] is defined and
      'plot2' in my_dict['field_representations']}

\BLOCK{ set valid_keys = [] }
\begin{itemize}
    \BLOCK{ for key, value in my_dict['field_representations']['plot2'].items() }
        \BLOCK{ if key != 'preprocessing' and key != 'postprocessing' }
            \BLOCK{ set _ = valid_keys.append(key) }
        \BLOCK{ endif }
    \BLOCK{ endfor }

    \BLOCK{ for valid_key in valid_keys }
        \item
        \verb| \VAR{ valid_key } | used as content representation technique.
    \BLOCK{ endfor }
\end{itemize}

% content_representation_field_ plot2 ____
\BLOCK{endif}



% preprocessing on plot2
\BLOCK{ if my_dict is defined and
      my_dict['field_representations'] is defined and
      'plot2' in my_dict['field_representations'] and
      'preprocessing' in my_dict['field_representations']['plot2'] and
      my_dict['field_representations']['plot2']['preprocessing'] is defined and
      my_dict['field_representations']['plot2']['preprocessing'] is not none }
    % dictionary 'preprocessing' not empty
\begin{itemize}
\BLOCK{ for key, value in my_dict['field_representations']['plot2']['preprocessing'].items() }
    \item
     \verb| \VAR{ key } | used as preprocessing technique.
\BLOCK{ endfor }
\end{itemize}
\BLOCK{else}
No preprocessing techniques have been used to preprocess data \lstinline[style=verbatim-text]| plot2 | has been represented with the following techniques:
 during the experiment.
\BLOCK{endif}
% _preprocessing on plot2 ____




\BLOCK{if my_dict is defined and
      my_dict['field_representations'] is defined and
      'plot2' in my_dict['field_representations'] and
      'postprocessing' in my_dict['field_representations']['plot2'] and
      my_dict['field_representations']['plot2']['postprocessing'] is defined and
      my_dict['field_representations']['plot2']['postprocessing'] is not none}
    % dictionary 'preprocessing' not empty
On th field \verb| plot2 | have been applied the following postprocessing techniques:
\BLOCK{ for key, value in my_dict['field_representations']['plot2']['postprocessing'].items() }
      \BLOCK{ for k, v in value.items() }
            \verb| \VAR{ k } | ,
      \BLOCK{ endfor }
\BLOCK{ endfor }
\BLOCK{else}
% This block is rendered if no post processing technique has been applied
No postprocessing techniques have been used on the data \lstinline[style=verbatim-text]| plot2 | in this experiment.
\BLOCK{endif}
\hfill\break

% postprocessing_on_ plot2 ___



% field: plot3 representation subsection
\textbf{\lstinline[style=verbatim-text]| plot3 |} has been represented with the following techniques:
% content representation on plot3
\BLOCK{if my_dict is defined and
      my_dict['field_representations'] is defined and
      'plot3' in my_dict['field_representations']}

\BLOCK{ set valid_keys = [] }
\begin{itemize}
    \BLOCK{ for key, value in my_dict['field_representations']['plot3'].items() }
        \BLOCK{ if key != 'preprocessing' and key != 'postprocessing' }
            \BLOCK{ set _ = valid_keys.append(key) }
        \BLOCK{ endif }
    \BLOCK{ endfor }

    \BLOCK{ for valid_key in valid_keys }
        \item
        \verb| \VAR{ valid_key } | used as content representation technique.
    \BLOCK{ endfor }
\end{itemize}

% content_representation_field_ plot3 ____
\BLOCK{endif}



% preprocessing on plot3
\BLOCK{ if my_dict is defined and
      my_dict['field_representations'] is defined and
      'plot3' in my_dict['field_representations'] and
      'preprocessing' in my_dict['field_representations']['plot3'] and
      my_dict['field_representations']['plot3']['preprocessing'] is defined and
      my_dict['field_representations']['plot3']['preprocessing'] is not none }
    % dictionary 'preprocessing' not empty
\begin{itemize}
\BLOCK{ for key, value in my_dict['field_representations']['plot3']['preprocessing'].items() }
    \item
     \verb| \VAR{ key } | used as preprocessing technique.
\BLOCK{ endfor }
\end{itemize}
\BLOCK{else}
No preprocessing techniques have been used to preprocess data \lstinline[style=verbatim-text]| plot3 | has been represented with the following techniques:
 during the experiment.
\BLOCK{endif}
% _preprocessing on plot3 ____




\BLOCK{if my_dict is defined and
      my_dict['field_representations'] is defined and
      'plot3' in my_dict['field_representations'] and
      'postprocessing' in my_dict['field_representations']['plot3'] and
      my_dict['field_representations']['plot3']['postprocessing'] is defined and
      my_dict['field_representations']['plot3']['postprocessing'] is not none}
    % dictionary 'preprocessing' not empty
On th field \verb| plot3 | have been applied the following postprocessing techniques:
\BLOCK{ for key, value in my_dict['field_representations']['plot3']['postprocessing'].items() }
      \BLOCK{ for k, v in value.items() }
            \verb| \VAR{ k } | ,
      \BLOCK{ endfor }
\BLOCK{ endfor }
\BLOCK{else}
% This block is rendered if no post processing technique has been applied
No postprocessing techniques have been used on the data \lstinline[style=verbatim-text]| plot3 | in this experiment.
\BLOCK{endif}
\hfill\break

% postprocessing_on_ plot3 ___



% field: plot4 representation subsection
\textbf{\lstinline[style=verbatim-text]| plot4 |} has been represented with the following techniques:
% content representation on plot4
\BLOCK{if my_dict is defined and
      my_dict['field_representations'] is defined and
      'plot4' in my_dict['field_representations']}

\BLOCK{ set valid_keys = [] }
\begin{itemize}
    \BLOCK{ for key, value in my_dict['field_representations']['plot4'].items() }
        \BLOCK{ if key != 'preprocessing' and key != 'postprocessing' }
            \BLOCK{ set _ = valid_keys.append(key) }
        \BLOCK{ endif }
    \BLOCK{ endfor }

    \BLOCK{ for valid_key in valid_keys }
        \item
        \verb| \VAR{ valid_key } | used as content representation technique.
    \BLOCK{ endfor }
\end{itemize}

% content_representation_field_ plot4 ____
\BLOCK{endif}



% preprocessing on plot4
\BLOCK{ if my_dict is defined and
      my_dict['field_representations'] is defined and
      'plot4' in my_dict['field_representations'] and
      'preprocessing' in my_dict['field_representations']['plot4'] and
      my_dict['field_representations']['plot4']['preprocessing'] is defined and
      my_dict['field_representations']['plot4']['preprocessing'] is not none }
    % dictionary 'preprocessing' not empty
\begin{itemize}
\BLOCK{ for key, value in my_dict['field_representations']['plot4']['preprocessing'].items() }
    \item
     \verb| \VAR{ key } | used as preprocessing technique.
\BLOCK{ endfor }
\end{itemize}
\BLOCK{else}
No preprocessing techniques have been used to preprocess data \lstinline[style=verbatim-text]| plot4 | has been represented with the following techniques:
 during the experiment.
\BLOCK{endif}
% _preprocessing on plot4 ____




\BLOCK{if my_dict is defined and
      my_dict['field_representations'] is defined and
      'plot4' in my_dict['field_representations'] and
      'postprocessing' in my_dict['field_representations']['plot4'] and
      my_dict['field_representations']['plot4']['postprocessing'] is defined and
      my_dict['field_representations']['plot4']['postprocessing'] is not none}
    % dictionary 'preprocessing' not empty
On th field \verb| plot4 | have been applied the following postprocessing techniques:
\BLOCK{ for key, value in my_dict['field_representations']['plot4']['postprocessing'].items() }
      \BLOCK{ for k, v in value.items() }
            \verb| \VAR{ k } | ,
      \BLOCK{ endfor }
\BLOCK{ endfor }
\BLOCK{else}
% This block is rendered if no post processing technique has been applied
No postprocessing techniques have been used on the data \lstinline[style=verbatim-text]| plot4 | in this experiment.
\BLOCK{endif}
\hfill\break

% postprocessing_on_ plot4 ___



% field: plot5 representation subsection
\textbf{\lstinline[style=verbatim-text]| plot5 |} has been represented with the following techniques:
% content representation on plot5
\BLOCK{if my_dict is defined and
      my_dict['field_representations'] is defined and
      'plot5' in my_dict['field_representations']}

\BLOCK{ set valid_keys = [] }
\begin{itemize}
    \BLOCK{ for key, value in my_dict['field_representations']['plot5'].items() }
        \BLOCK{ if key != 'preprocessing' and key != 'postprocessing' }
            \BLOCK{ set _ = valid_keys.append(key) }
        \BLOCK{ endif }
    \BLOCK{ endfor }

    \BLOCK{ for valid_key in valid_keys }
        \item
        \verb| \VAR{ valid_key } | used as content representation technique.
    \BLOCK{ endfor }
\end{itemize}

% content_representation_field_ plot5 ____
\BLOCK{endif}



% preprocessing on plot5
\BLOCK{ if my_dict is defined and
      my_dict['field_representations'] is defined and
      'plot5' in my_dict['field_representations'] and
      'preprocessing' in my_dict['field_representations']['plot5'] and
      my_dict['field_representations']['plot5']['preprocessing'] is defined and
      my_dict['field_representations']['plot5']['preprocessing'] is not none }
    % dictionary 'preprocessing' not empty
\begin{itemize}
\BLOCK{ for key, value in my_dict['field_representations']['plot5']['preprocessing'].items() }
    \item
     \verb| \VAR{ key } | used as preprocessing technique.
\BLOCK{ endfor }
\end{itemize}
\BLOCK{else}
No preprocessing techniques have been used to preprocess data \lstinline[style=verbatim-text]| plot5 | has been represented with the following techniques:
 during the experiment.
\BLOCK{endif}
% _preprocessing on plot5 ____




\BLOCK{if my_dict is defined and
      my_dict['field_representations'] is defined and
      'plot5' in my_dict['field_representations'] and
      'postprocessing' in my_dict['field_representations']['plot5'] and
      my_dict['field_representations']['plot5']['postprocessing'] is defined and
      my_dict['field_representations']['plot5']['postprocessing'] is not none}
    % dictionary 'preprocessing' not empty
On th field \verb| plot5 | have been applied the following postprocessing techniques:
\BLOCK{ for key, value in my_dict['field_representations']['plot5']['postprocessing'].items() }
      \BLOCK{ for k, v in value.items() }
            \verb| \VAR{ k } | ,
      \BLOCK{ endfor }
\BLOCK{ endfor }
\BLOCK{else}
% This block is rendered if no post processing technique has been applied
No postprocessing techniques have been used on the data \lstinline[style=verbatim-text]| plot5 | in this experiment.
\BLOCK{endif}
\hfill\break

% postprocessing_on_ plot5 ___



% field: plot6 representation subsection
\textbf{\lstinline[style=verbatim-text]| plot6 |} has been represented with the following techniques:
% content representation on plot6
\BLOCK{if my_dict is defined and
      my_dict['field_representations'] is defined and
      'plot6' in my_dict['field_representations']}

\BLOCK{ set valid_keys = [] }
\begin{itemize}
    \BLOCK{ for key, value in my_dict['field_representations']['plot6'].items() }
        \BLOCK{ if key != 'preprocessing' and key != 'postprocessing' }
            \BLOCK{ set _ = valid_keys.append(key) }
        \BLOCK{ endif }
    \BLOCK{ endfor }

    \BLOCK{ for valid_key in valid_keys }
        \item
        \verb| \VAR{ valid_key } | used as content representation technique.
    \BLOCK{ endfor }
\end{itemize}

% content_representation_field_ plot6 ____
\BLOCK{endif}



% preprocessing on plot6
\BLOCK{ if my_dict is defined and
      my_dict['field_representations'] is defined and
      'plot6' in my_dict['field_representations'] and
      'preprocessing' in my_dict['field_representations']['plot6'] and
      my_dict['field_representations']['plot6']['preprocessing'] is defined and
      my_dict['field_representations']['plot6']['preprocessing'] is not none }
    % dictionary 'preprocessing' not empty
\begin{itemize}
\BLOCK{ for key, value in my_dict['field_representations']['plot6']['preprocessing'].items() }
    \item
     \verb| \VAR{ key } | used as preprocessing technique.
\BLOCK{ endfor }
\end{itemize}
\BLOCK{else}
No preprocessing techniques have been used to preprocess data \lstinline[style=verbatim-text]| plot6 | has been represented with the following techniques:
 during the experiment.
\BLOCK{endif}
% _preprocessing on plot6 ____




\BLOCK{if my_dict is defined and
      my_dict['field_representations'] is defined and
      'plot6' in my_dict['field_representations'] and
      'postprocessing' in my_dict['field_representations']['plot6'] and
      my_dict['field_representations']['plot6']['postprocessing'] is defined and
      my_dict['field_representations']['plot6']['postprocessing'] is not none}
    % dictionary 'preprocessing' not empty
On th field \verb| plot6 | have been applied the following postprocessing techniques:
\BLOCK{ for key, value in my_dict['field_representations']['plot6']['postprocessing'].items() }
      \BLOCK{ for k, v in value.items() }
            \verb| \VAR{ k } | ,
      \BLOCK{ endfor }
\BLOCK{ endfor }
\BLOCK{else}
% This block is rendered if no post processing technique has been applied
No postprocessing techniques have been used on the data \lstinline[style=verbatim-text]| plot6 | in this experiment.
\BLOCK{endif}
\hfill\break

% postprocessing_on_ plot6 ___



% field: plot7 representation subsection
\textbf{\lstinline[style=verbatim-text]| plot7 |} has been represented with the following techniques:
% content representation on plot7
\BLOCK{if my_dict is defined and
      my_dict['field_representations'] is defined and
      'plot7' in my_dict['field_representations']}

\BLOCK{ set valid_keys = [] }
\begin{itemize}
    \BLOCK{ for key, value in my_dict['field_representations']['plot7'].items() }
        \BLOCK{ if key != 'preprocessing' and key != 'postprocessing' }
            \BLOCK{ set _ = valid_keys.append(key) }
        \BLOCK{ endif }
    \BLOCK{ endfor }

    \BLOCK{ for valid_key in valid_keys }
        \item
        \verb| \VAR{ valid_key } | used as content representation technique.
    \BLOCK{ endfor }
\end{itemize}

% content_representation_field_ plot7 ____
\BLOCK{endif}



% preprocessing on plot7
\BLOCK{ if my_dict is defined and
      my_dict['field_representations'] is defined and
      'plot7' in my_dict['field_representations'] and
      'preprocessing' in my_dict['field_representations']['plot7'] and
      my_dict['field_representations']['plot7']['preprocessing'] is defined and
      my_dict['field_representations']['plot7']['preprocessing'] is not none }
    % dictionary 'preprocessing' not empty
\begin{itemize}
\BLOCK{ for key, value in my_dict['field_representations']['plot7']['preprocessing'].items() }
    \item
     \verb| \VAR{ key } | used as preprocessing technique.
\BLOCK{ endfor }
\end{itemize}
\BLOCK{else}
No preprocessing techniques have been used to preprocess data \lstinline[style=verbatim-text]| plot7 | has been represented with the following techniques:
 during the experiment.
\BLOCK{endif}
% _preprocessing on plot7 ____




\BLOCK{if my_dict is defined and
      my_dict['field_representations'] is defined and
      'plot7' in my_dict['field_representations'] and
      'postprocessing' in my_dict['field_representations']['plot7'] and
      my_dict['field_representations']['plot7']['postprocessing'] is defined and
      my_dict['field_representations']['plot7']['postprocessing'] is not none}
    % dictionary 'preprocessing' not empty
On th field \verb| plot7 | have been applied the following postprocessing techniques:
\BLOCK{ for key, value in my_dict['field_representations']['plot7']['postprocessing'].items() }
      \BLOCK{ for k, v in value.items() }
            \verb| \VAR{ k } | ,
      \BLOCK{ endfor }
\BLOCK{ endfor }
\BLOCK{else}
% This block is rendered if no post processing technique has been applied
No postprocessing techniques have been used on the data \lstinline[style=verbatim-text]| plot7 | in this experiment.
\BLOCK{endif}
\hfill\break

% postprocessing_on_ plot7 ___



% field: plot8 representation subsection
\textbf{\lstinline[style=verbatim-text]| plot8 |} has been represented with the following techniques:
% content representation on plot8
\BLOCK{if my_dict is defined and
      my_dict['field_representations'] is defined and
      'plot8' in my_dict['field_representations']}

\BLOCK{ set valid_keys = [] }
\begin{itemize}
    \BLOCK{ for key, value in my_dict['field_representations']['plot8'].items() }
        \BLOCK{ if key != 'preprocessing' and key != 'postprocessing' }
            \BLOCK{ set _ = valid_keys.append(key) }
        \BLOCK{ endif }
    \BLOCK{ endfor }

    \BLOCK{ for valid_key in valid_keys }
        \item
        \verb| \VAR{ valid_key } | used as content representation technique.
    \BLOCK{ endfor }
\end{itemize}

% content_representation_field_ plot8 ____
\BLOCK{endif}



% preprocessing on plot8
\BLOCK{ if my_dict is defined and
      my_dict['field_representations'] is defined and
      'plot8' in my_dict['field_representations'] and
      'preprocessing' in my_dict['field_representations']['plot8'] and
      my_dict['field_representations']['plot8']['preprocessing'] is defined and
      my_dict['field_representations']['plot8']['preprocessing'] is not none }
    % dictionary 'preprocessing' not empty
\begin{itemize}
\BLOCK{ for key, value in my_dict['field_representations']['plot8']['preprocessing'].items() }
    \item
     \verb| \VAR{ key } | used as preprocessing technique.
\BLOCK{ endfor }
\end{itemize}
\BLOCK{else}
No preprocessing techniques have been used to preprocess data \lstinline[style=verbatim-text]| plot8 | has been represented with the following techniques:
 during the experiment.
\BLOCK{endif}
% _preprocessing on plot8 ____




\BLOCK{if my_dict is defined and
      my_dict['field_representations'] is defined and
      'plot8' in my_dict['field_representations'] and
      'postprocessing' in my_dict['field_representations']['plot8'] and
      my_dict['field_representations']['plot8']['postprocessing'] is defined and
      my_dict['field_representations']['plot8']['postprocessing'] is not none}
    % dictionary 'preprocessing' not empty
On th field \verb| plot8 | have been applied the following postprocessing techniques:
\BLOCK{ for key, value in my_dict['field_representations']['plot8']['postprocessing'].items() }
      \BLOCK{ for k, v in value.items() }
            \verb| \VAR{ k } | ,
      \BLOCK{ endfor }
\BLOCK{ endfor }
\BLOCK{else}
% This block is rendered if no post processing technique has been applied
No postprocessing techniques have been used on the data \lstinline[style=verbatim-text]| plot8 | in this experiment.
\BLOCK{endif}
\hfill\break

% postprocessing_on_ plot8 ___



% field: plot9 representation subsection
\textbf{\lstinline[style=verbatim-text]| plot9 |} has been represented with the following techniques:
% content representation on plot9
\BLOCK{if my_dict is defined and
      my_dict['field_representations'] is defined and
      'plot9' in my_dict['field_representations']}

\BLOCK{ set valid_keys = [] }
\begin{itemize}
    \BLOCK{ for key, value in my_dict['field_representations']['plot9'].items() }
        \BLOCK{ if key != 'preprocessing' and key != 'postprocessing' }
            \BLOCK{ set _ = valid_keys.append(key) }
        \BLOCK{ endif }
    \BLOCK{ endfor }

    \BLOCK{ for valid_key in valid_keys }
        \item
        \verb| \VAR{ valid_key } | used as content representation technique.
    \BLOCK{ endfor }
\end{itemize}

% content_representation_field_ plot9 ____
\BLOCK{endif}



% preprocessing on plot9
\BLOCK{ if my_dict is defined and
      my_dict['field_representations'] is defined and
      'plot9' in my_dict['field_representations'] and
      'preprocessing' in my_dict['field_representations']['plot9'] and
      my_dict['field_representations']['plot9']['preprocessing'] is defined and
      my_dict['field_representations']['plot9']['preprocessing'] is not none }
    % dictionary 'preprocessing' not empty
\begin{itemize}
\BLOCK{ for key, value in my_dict['field_representations']['plot9']['preprocessing'].items() }
    \item
     \verb| \VAR{ key } | used as preprocessing technique.
\BLOCK{ endfor }
\end{itemize}
\BLOCK{else}
No preprocessing techniques have been used to preprocess data \lstinline[style=verbatim-text]| plot9 | has been represented with the following techniques:
 during the experiment.
\BLOCK{endif}
% _preprocessing on plot9 ____




\BLOCK{if my_dict is defined and
      my_dict['field_representations'] is defined and
      'plot9' in my_dict['field_representations'] and
      'postprocessing' in my_dict['field_representations']['plot9'] and
      my_dict['field_representations']['plot9']['postprocessing'] is defined and
      my_dict['field_representations']['plot9']['postprocessing'] is not none}
    % dictionary 'preprocessing' not empty
On th field \verb| plot9 | have been applied the following postprocessing techniques:
\BLOCK{ for key, value in my_dict['field_representations']['plot9']['postprocessing'].items() }
      \BLOCK{ for k, v in value.items() }
            \verb| \VAR{ k } | ,
      \BLOCK{ endfor }
\BLOCK{ endfor }
\BLOCK{else}
% This block is rendered if no post processing technique has been applied
No postprocessing techniques have been used on the data \lstinline[style=verbatim-text]| plot9 | in this experiment.
\BLOCK{endif}
\hfill\break

% postprocessing_on_ plot9 ___



% field: video_path0 representation subsection
\textbf{\lstinline[style=verbatim-text]| video_path0 |} has been represented with the following techniques:
% content representation on video_path0
\BLOCK{if my_dict is defined and
      my_dict['field_representations'] is defined and
      'video_path0' in my_dict['field_representations']}

\BLOCK{ set valid_keys = [] }
\begin{itemize}
    \BLOCK{ for key, value in my_dict['field_representations']['video_path0'].items() }
        \BLOCK{ if key != 'preprocessing' and key != 'postprocessing' }
            \BLOCK{ set _ = valid_keys.append(key) }
        \BLOCK{ endif }
    \BLOCK{ endfor }

    \BLOCK{ for valid_key in valid_keys }
        \item
        \verb| \VAR{ valid_key } | used as content representation technique.
    \BLOCK{ endfor }
\end{itemize}

% content_representation_field_ video_path0 ____
\BLOCK{endif}



% preprocessing on video_path0
\BLOCK{ if my_dict is defined and
      my_dict['field_representations'] is defined and
      'video_path0' in my_dict['field_representations'] and
      'preprocessing' in my_dict['field_representations']['video_path0'] and
      my_dict['field_representations']['video_path0']['preprocessing'] is defined and
      my_dict['field_representations']['video_path0']['preprocessing'] is not none }
    % dictionary 'preprocessing' not empty
\begin{itemize}
\BLOCK{ for key, value in my_dict['field_representations']['video_path0']['preprocessing'].items() }
    \item
     \verb| \VAR{ key } | used as preprocessing technique.
\BLOCK{ endfor }
\end{itemize}
\BLOCK{else}
No preprocessing techniques have been used to preprocess data \lstinline[style=verbatim-text]| video_path0 | has been represented with the following techniques:
 during the experiment.
\BLOCK{endif}
% _preprocessing on video_path0 ____




\BLOCK{if my_dict is defined and
      my_dict['field_representations'] is defined and
      'video_path0' in my_dict['field_representations'] and
      'postprocessing' in my_dict['field_representations']['video_path0'] and
      my_dict['field_representations']['video_path0']['postprocessing'] is defined and
      my_dict['field_representations']['video_path0']['postprocessing'] is not none}
    % dictionary 'preprocessing' not empty
On th field \verb| video_path0 | have been applied the following postprocessing techniques:
\BLOCK{ for key, value in my_dict['field_representations']['video_path0']['postprocessing'].items() }
      \BLOCK{ for k, v in value.items() }
            \verb| \VAR{ k } | ,
      \BLOCK{ endfor }
\BLOCK{ endfor }
\BLOCK{else}
% This block is rendered if no post processing technique has been applied
No postprocessing techniques have been used on the data \lstinline[style=verbatim-text]| video_path0 | in this experiment.
\BLOCK{endif}
\hfill\break

% postprocessing_on_ video_path0 ___



% field: video_path1 representation subsection
\textbf{\lstinline[style=verbatim-text]| video_path1 |} has been represented with the following techniques:
% content representation on video_path1
\BLOCK{if my_dict is defined and
      my_dict['field_representations'] is defined and
      'video_path1' in my_dict['field_representations']}

\BLOCK{ set valid_keys = [] }
\begin{itemize}
    \BLOCK{ for key, value in my_dict['field_representations']['video_path1'].items() }
        \BLOCK{ if key != 'preprocessing' and key != 'postprocessing' }
            \BLOCK{ set _ = valid_keys.append(key) }
        \BLOCK{ endif }
    \BLOCK{ endfor }

    \BLOCK{ for valid_key in valid_keys }
        \item
        \verb| \VAR{ valid_key } | used as content representation technique.
    \BLOCK{ endfor }
\end{itemize}

% content_representation_field_ video_path1 ____
\BLOCK{endif}



% preprocessing on video_path1
\BLOCK{ if my_dict is defined and
      my_dict['field_representations'] is defined and
      'video_path1' in my_dict['field_representations'] and
      'preprocessing' in my_dict['field_representations']['video_path1'] and
      my_dict['field_representations']['video_path1']['preprocessing'] is defined and
      my_dict['field_representations']['video_path1']['preprocessing'] is not none }
    % dictionary 'preprocessing' not empty
\begin{itemize}
\BLOCK{ for key, value in my_dict['field_representations']['video_path1']['preprocessing'].items() }
    \item
     \verb| \VAR{ key } | used as preprocessing technique.
\BLOCK{ endfor }
\end{itemize}
\BLOCK{else}
No preprocessing techniques have been used to preprocess data \lstinline[style=verbatim-text]| video_path1 | has been represented with the following techniques:
 during the experiment.
\BLOCK{endif}
% _preprocessing on video_path1 ____




\BLOCK{if my_dict is defined and
      my_dict['field_representations'] is defined and
      'video_path1' in my_dict['field_representations'] and
      'postprocessing' in my_dict['field_representations']['video_path1'] and
      my_dict['field_representations']['video_path1']['postprocessing'] is defined and
      my_dict['field_representations']['video_path1']['postprocessing'] is not none}
    % dictionary 'preprocessing' not empty
On th field \verb| video_path1 | have been applied the following postprocessing techniques:
\BLOCK{ for key, value in my_dict['field_representations']['video_path1']['postprocessing'].items() }
      \BLOCK{ for k, v in value.items() }
            \verb| \VAR{ k } | ,
      \BLOCK{ endfor }
\BLOCK{ endfor }
\BLOCK{else}
% This block is rendered if no post processing technique has been applied
No postprocessing techniques have been used on the data \lstinline[style=verbatim-text]| video_path1 | in this experiment.
\BLOCK{endif}
\hfill\break

% postprocessing_on_ video_path1 ___



% field: video_path2 representation subsection
\textbf{\lstinline[style=verbatim-text]| video_path2 |} has been represented with the following techniques:
% content representation on video_path2
\BLOCK{if my_dict is defined and
      my_dict['field_representations'] is defined and
      'video_path2' in my_dict['field_representations']}

\BLOCK{ set valid_keys = [] }
\begin{itemize}
    \BLOCK{ for key, value in my_dict['field_representations']['video_path2'].items() }
        \BLOCK{ if key != 'preprocessing' and key != 'postprocessing' }
            \BLOCK{ set _ = valid_keys.append(key) }
        \BLOCK{ endif }
    \BLOCK{ endfor }

    \BLOCK{ for valid_key in valid_keys }
        \item
        \verb| \VAR{ valid_key } | used as content representation technique.
    \BLOCK{ endfor }
\end{itemize}

% content_representation_field_ video_path2 ____
\BLOCK{endif}



% preprocessing on video_path2
\BLOCK{ if my_dict is defined and
      my_dict['field_representations'] is defined and
      'video_path2' in my_dict['field_representations'] and
      'preprocessing' in my_dict['field_representations']['video_path2'] and
      my_dict['field_representations']['video_path2']['preprocessing'] is defined and
      my_dict['field_representations']['video_path2']['preprocessing'] is not none }
    % dictionary 'preprocessing' not empty
\begin{itemize}
\BLOCK{ for key, value in my_dict['field_representations']['video_path2']['preprocessing'].items() }
    \item
     \verb| \VAR{ key } | used as preprocessing technique.
\BLOCK{ endfor }
\end{itemize}
\BLOCK{else}
No preprocessing techniques have been used to preprocess data \lstinline[style=verbatim-text]| video_path2 | has been represented with the following techniques:
 during the experiment.
\BLOCK{endif}
% _preprocessing on video_path2 ____




\BLOCK{if my_dict is defined and
      my_dict['field_representations'] is defined and
      'video_path2' in my_dict['field_representations'] and
      'postprocessing' in my_dict['field_representations']['video_path2'] and
      my_dict['field_representations']['video_path2']['postprocessing'] is defined and
      my_dict['field_representations']['video_path2']['postprocessing'] is not none}
    % dictionary 'preprocessing' not empty
On th field \verb| video_path2 | have been applied the following postprocessing techniques:
\BLOCK{ for key, value in my_dict['field_representations']['video_path2']['postprocessing'].items() }
      \BLOCK{ for k, v in value.items() }
            \verb| \VAR{ k } | ,
      \BLOCK{ endfor }
\BLOCK{ endfor }
\BLOCK{else}
% This block is rendered if no post processing technique has been applied
No postprocessing techniques have been used on the data \lstinline[style=verbatim-text]| video_path2 | in this experiment.
\BLOCK{endif}
\hfill\break

% postprocessing_on_ video_path2 ___



% field: video_path3 representation subsection
\textbf{\lstinline[style=verbatim-text]| video_path3 |} has been represented with the following techniques:
% content representation on video_path3
\BLOCK{if my_dict is defined and
      my_dict['field_representations'] is defined and
      'video_path3' in my_dict['field_representations']}

\BLOCK{ set valid_keys = [] }
\begin{itemize}
    \BLOCK{ for key, value in my_dict['field_representations']['video_path3'].items() }
        \BLOCK{ if key != 'preprocessing' and key != 'postprocessing' }
            \BLOCK{ set _ = valid_keys.append(key) }
        \BLOCK{ endif }
    \BLOCK{ endfor }

    \BLOCK{ for valid_key in valid_keys }
        \item
        \verb| \VAR{ valid_key } | used as content representation technique.
    \BLOCK{ endfor }
\end{itemize}

% content_representation_field_ video_path3 ____
\BLOCK{endif}



% preprocessing on video_path3
\BLOCK{ if my_dict is defined and
      my_dict['field_representations'] is defined and
      'video_path3' in my_dict['field_representations'] and
      'preprocessing' in my_dict['field_representations']['video_path3'] and
      my_dict['field_representations']['video_path3']['preprocessing'] is defined and
      my_dict['field_representations']['video_path3']['preprocessing'] is not none }
    % dictionary 'preprocessing' not empty
\begin{itemize}
\BLOCK{ for key, value in my_dict['field_representations']['video_path3']['preprocessing'].items() }
    \item
     \verb| \VAR{ key } | used as preprocessing technique.
\BLOCK{ endfor }
\end{itemize}
\BLOCK{else}
No preprocessing techniques have been used to preprocess data \lstinline[style=verbatim-text]| video_path3 | has been represented with the following techniques:
 during the experiment.
\BLOCK{endif}
% _preprocessing on video_path3 ____




\BLOCK{if my_dict is defined and
      my_dict['field_representations'] is defined and
      'video_path3' in my_dict['field_representations'] and
      'postprocessing' in my_dict['field_representations']['video_path3'] and
      my_dict['field_representations']['video_path3']['postprocessing'] is defined and
      my_dict['field_representations']['video_path3']['postprocessing'] is not none}
    % dictionary 'preprocessing' not empty
On th field \verb| video_path3 | have been applied the following postprocessing techniques:
\BLOCK{ for key, value in my_dict['field_representations']['video_path3']['postprocessing'].items() }
      \BLOCK{ for k, v in value.items() }
            \verb| \VAR{ k } | ,
      \BLOCK{ endfor }
\BLOCK{ endfor }
\BLOCK{else}
% This block is rendered if no post processing technique has been applied
No postprocessing techniques have been used on the data \lstinline[style=verbatim-text]| video_path3 | in this experiment.
\BLOCK{endif}
\hfill\break

% postprocessing_on_ video_path3 ___



% field: video_path4 representation subsection
\textbf{\lstinline[style=verbatim-text]| video_path4 |} has been represented with the following techniques:
% content representation on video_path4
\BLOCK{if my_dict is defined and
      my_dict['field_representations'] is defined and
      'video_path4' in my_dict['field_representations']}

\BLOCK{ set valid_keys = [] }
\begin{itemize}
    \BLOCK{ for key, value in my_dict['field_representations']['video_path4'].items() }
        \BLOCK{ if key != 'preprocessing' and key != 'postprocessing' }
            \BLOCK{ set _ = valid_keys.append(key) }
        \BLOCK{ endif }
    \BLOCK{ endfor }

    \BLOCK{ for valid_key in valid_keys }
        \item
        \verb| \VAR{ valid_key } | used as content representation technique.
    \BLOCK{ endfor }
\end{itemize}

% content_representation_field_ video_path4 ____
\BLOCK{endif}



% preprocessing on video_path4
\BLOCK{ if my_dict is defined and
      my_dict['field_representations'] is defined and
      'video_path4' in my_dict['field_representations'] and
      'preprocessing' in my_dict['field_representations']['video_path4'] and
      my_dict['field_representations']['video_path4']['preprocessing'] is defined and
      my_dict['field_representations']['video_path4']['preprocessing'] is not none }
    % dictionary 'preprocessing' not empty
\begin{itemize}
\BLOCK{ for key, value in my_dict['field_representations']['video_path4']['preprocessing'].items() }
    \item
     \verb| \VAR{ key } | used as preprocessing technique.
\BLOCK{ endfor }
\end{itemize}
\BLOCK{else}
No preprocessing techniques have been used to preprocess data \lstinline[style=verbatim-text]| video_path4 | has been represented with the following techniques:
 during the experiment.
\BLOCK{endif}
% _preprocessing on video_path4 ____




\BLOCK{if my_dict is defined and
      my_dict['field_representations'] is defined and
      'video_path4' in my_dict['field_representations'] and
      'postprocessing' in my_dict['field_representations']['video_path4'] and
      my_dict['field_representations']['video_path4']['postprocessing'] is defined and
      my_dict['field_representations']['video_path4']['postprocessing'] is not none}
    % dictionary 'preprocessing' not empty
On th field \verb| video_path4 | have been applied the following postprocessing techniques:
\BLOCK{ for key, value in my_dict['field_representations']['video_path4']['postprocessing'].items() }
      \BLOCK{ for k, v in value.items() }
            \verb| \VAR{ k } | ,
      \BLOCK{ endfor }
\BLOCK{ endfor }
\BLOCK{else}
% This block is rendered if no post processing technique has been applied
No postprocessing techniques have been used on the data \lstinline[style=verbatim-text]| video_path4 | in this experiment.
\BLOCK{endif}
\hfill\break

% postprocessing_on_ video_path4 ___


\BLOCK{if my_dict['exogenous_representations'] is defined and
        my_dict['exogenous_representations'] is not none}
% exogenous techniques start
In this experiment the data used have been represented with exogenous representation, in particular the
following techniques have been used:
\begin{itemize}
\BLOCK{ for key, value in (my_dict['exogenous_representations'] | default({})).items() }
    \item
     \verb| \VAR{ key } | used as exogenous technique with following settings:
     \begin{itemize}
      \BLOCK{ for k, v in (value | default({})).items() }
        \BLOCK{ if k not in ["mode", "prop_as_uri", "max_timeout"] }
             \item
             \verb| \VAR{ k }: \VAR{ v }|
        \BLOCK{ endif }
      \BLOCK{ endfor }
     \end{itemize}
\BLOCK{ endfor }
\end{itemize}
\BLOCK{else}
No exogenous techniques have been used in this experiment to represent the data used.
\BLOCK{endif}
% exogenous_techniques_end___




\BLOCK{else}
% In case of the content analyzer is not used
For this experiment the module of the content analyzer has not been used.
\hfill\break
\hfill\break
% <----------  closing the controll block for the content analyzer section ----------->
\BLOCK{endif}



\hfill\break
% subsection of Dataset identification
\subsection{Dataset used and its statistics}
The dataset used for the experiment is  \lstinline[style=verbatim-text]| 1000K data video movie |,
\BLOCK{ if my_dict is defined and my_dict['source_file'] is defined }
its source file is \lstinline[style=verbatim-text]| \VAR{ my_dict['source_file'] } | ,\BLOCK{endif}
and the content used during the experiment:
\BLOCK{ if my_dict is defined and my_dict['id_each_content'] is defined and my_dict['id_each_content'] }
\BLOCK{ for value in my_dict['id_each_content'] }
     \verb| \VAR{ value }, |
\BLOCK{ endfor }.
\BLOCK{endif}


\BLOCK{ if my_dict is defined and my_dict['interactions'] is defined }
% --- DATA STATS ---
The statistics of the dataset used are reported in the following table:~\ref{tab:dataset_table}:
\begin{table}[ht]
    \centering
  \begin{tabular}{|c|c|}
    \hline
    \textbf{Parameter}& \textbf{Value} \\ \hline
    n\_users  & \VAR{my_dict['interactions']['n_users']|default('no users')|safe_text}\\ \hline
    n\_items  & \VAR{my_dict['interactions']['n_items']|default('no items')|safe_text}\\ \hline
    total\_interactions  & \VAR{my_dict['interactions']['total_interactions']|safe_text}\\ \hline
    min\_score  & \VAR{my_dict['interactions']['min_score']|truncate|safe_text}\\ \hline
    max\_score  & \VAR{my_dict['interactions']['max_score']|truncate|safe_text}\\ \hline
    mean\_score  & \VAR{my_dict['interactions']['mean_score']|truncate|safe_text}\\ \hline
    sparsity  & \VAR{my_dict['interactions']['sparsity']|truncate|safe_text}\\ \hline
  \end{tabular}
   \caption{Stats on the dataset}\label{tab:dataset_table}
\end{table}
\BLOCK{ else }
There are no statistics on the dataset to show.
\BLOCK{ endif }

\hfill\break



% ------------------------------ START SUBSECTION OF PARTITIONING OF RECSYS --------------------------------------------
% subsection of the splitting technique used, referred to partition protocol
\subsection{Data splitting technique}\label{subsec:partitioning}
\BLOCK{if my_dict['partitioning'] is defined and
        my_dict['partitioning']['KFoldPartitioning'] is defined}
% KFOLD PARTITIONING TECNIQUE
K-fold cross-validation is a technique used in machine learning to assess the performance of a predictive model.
The basic idea is to divide the dataset into K subsets, or folds.
The model is then trained K times, each time using K-1 folds for training and the remaining fold for validation.
This process is repeated K times, with a different fold used as the validation set in each iteration.
\hfill\break
The KFoldPartitioning has been used with the following setting:
\hfill\break
\BLOCK{if my_dict.get('partitioning', {}).get('KFoldPartitioning', {}).get('shuffle') == True}
The data has been shuffled before being split into batches.
\BLOCK{endif}
The partitioning technique has been executed with the following settings:
\begin{itemize}
    \item number of splits: \VAR{my_dict['partitioning']['KFoldPartitioning']['n_splits']}
    \item shuffle: \VAR{my_dict['partitioning']['KFoldPartitioning']['shuffle']}
    \item random state: \VAR{my_dict['partitioning']['KFoldPartitioning']['random_state']|default('no random state applied')}
    \item skip user error: \VAR{my_dict['partitioning']['KFoldPartitioning']['skip_user_error']|default('no setted')}
\end{itemize}
\hfill\break
% KFOLD PARTITIONING TECNIQUE ended
\BLOCK{endif}

\BLOCK{if my_dict['partitioning'] is defined and
        my_dict['partitioning']['HoldOutPartitioning'] is defined}
%  HOLD-OUT PARTIONING TECNIQUE
The partitioning used is the Hold-Out Partitioning.
This approach splits the dataset in use into a train set and a test set.
The training set is what the model is trained on, and the test set is used to see how
well the model will perform on new, unseen data.
\hfill\break
The train set size of this experiment is the \VAR{my_dict['partitioning']['HoldOutPartitioning']['train_set_size'] * 100}\%
of the original dataset, while the test set is the remaining \VAR{(100 - (my_dict['partitioning']['HoldOutPartitioning']['train_set_size'] * 100))}\%.
\hfill\break
\BLOCK{ if my_dict.get('partitioning', {}).get('HoldOutPartitioning', {}).get('shuffle') == True }
The data has been shuffled before being split into batches.
\BLOCK{endif}
\hfill\break
%  HOLD-OUT PARTIONING TECNIQUE ended
\BLOCK{endif}
% end partitioning section___________



\BLOCK{if my_dict['source_file'] is defined}
% ----------------------------------------- OPENING CONTENT ANALYZER SECTION -----------------------------------------
\section{Content Analyzer Module}\label{sec:ca}
The content analyzer module will deal with raw source document or more in general data which could be
video or audio data and give a representation of these data which will be used by the other two module.
The text data source could be represented with exogenous technique or with a specified representation
and each field given could be treated with preprocessing techniques and postprocessing technique.
In this experiment the following techniques have been used on specific field in order to achieve the
representation wanted:
\hfill\break
\hfill\break
% --- TECNIQUE USED TO REPPRESENT DATA FIELD ---



% field: idx0 representation subsection
\textbf{\lstinline[style=verbatim-text]| idx0 |} has been represented with the following techniques:
% content representation on idx0
\BLOCK{if my_dict is defined and
      my_dict['field_representations'] is defined and
      'idx0' in my_dict['field_representations']}

\BLOCK{ set valid_keys = [] }
\begin{itemize}
    \BLOCK{ for key, value in my_dict['field_representations']['idx0'].items() }
        \BLOCK{ if key != 'preprocessing' and key != 'postprocessing' }
            \BLOCK{ set _ = valid_keys.append(key) }
        \BLOCK{ endif }
    \BLOCK{ endfor }

    \BLOCK{ for valid_key in valid_keys }
        \item
        \verb| \VAR{ valid_key } | used as content representation technique.
    \BLOCK{ endfor }
\end{itemize}

% content_representation_field_ idx0 ____
\BLOCK{endif}



% preprocessing on idx0
\BLOCK{ if my_dict is defined and
      my_dict['field_representations'] is defined and
      'idx0' in my_dict['field_representations'] and
      'preprocessing' in my_dict['field_representations']['idx0'] and
      my_dict['field_representations']['idx0']['preprocessing'] is defined and
      my_dict['field_representations']['idx0']['preprocessing'] is not none }
    % dictionary 'preprocessing' not empty
\begin{itemize}
\BLOCK{ for key, value in my_dict['field_representations']['idx0']['preprocessing'].items() }
    \item
     \verb| \VAR{ key } | used as preprocessing technique.
\BLOCK{ endfor }
\end{itemize}
\BLOCK{else}
No preprocessing techniques have been used to preprocess data \lstinline[style=verbatim-text]| idx0 | has been represented with the following techniques:
 during the experiment.
\BLOCK{endif}
% _preprocessing on idx0 ____




\BLOCK{if my_dict is defined and
      my_dict['field_representations'] is defined and
      'idx0' in my_dict['field_representations'] and
      'postprocessing' in my_dict['field_representations']['idx0'] and
      my_dict['field_representations']['idx0']['postprocessing'] is defined and
      my_dict['field_representations']['idx0']['postprocessing'] is not none}
    % dictionary 'preprocessing' not empty
On th field \verb| idx0 | have been applied the following postprocessing techniques:
\BLOCK{ for key, value in my_dict['field_representations']['idx0']['postprocessing'].items() }
      \BLOCK{ for k, v in value.items() }
            \verb| \VAR{ k } | ,
      \BLOCK{ endfor }
\BLOCK{ endfor }
\BLOCK{else}
% This block is rendered if no post processing technique has been applied
No postprocessing techniques have been used on the data \lstinline[style=verbatim-text]| idx0 | in this experiment.
\BLOCK{endif}
\hfill\break

% postprocessing_on_ idx0 ___



% field: plot0 representation subsection
\textbf{\lstinline[style=verbatim-text]| plot0 |} has been represented with the following techniques:
% content representation on plot0
\BLOCK{if my_dict is defined and
      my_dict['field_representations'] is defined and
      'plot0' in my_dict['field_representations']}

\BLOCK{ set valid_keys = [] }
\begin{itemize}
    \BLOCK{ for key, value in my_dict['field_representations']['plot0'].items() }
        \BLOCK{ if key != 'preprocessing' and key != 'postprocessing' }
            \BLOCK{ set _ = valid_keys.append(key) }
        \BLOCK{ endif }
    \BLOCK{ endfor }

    \BLOCK{ for valid_key in valid_keys }
        \item
        \verb| \VAR{ valid_key } | used as content representation technique.
    \BLOCK{ endfor }
\end{itemize}

% content_representation_field_ plot0 ____
\BLOCK{endif}



% preprocessing on plot0
\BLOCK{ if my_dict is defined and
      my_dict['field_representations'] is defined and
      'plot0' in my_dict['field_representations'] and
      'preprocessing' in my_dict['field_representations']['plot0'] and
      my_dict['field_representations']['plot0']['preprocessing'] is defined and
      my_dict['field_representations']['plot0']['preprocessing'] is not none }
    % dictionary 'preprocessing' not empty
\begin{itemize}
\BLOCK{ for key, value in my_dict['field_representations']['plot0']['preprocessing'].items() }
    \item
     \verb| \VAR{ key } | used as preprocessing technique.
\BLOCK{ endfor }
\end{itemize}
\BLOCK{else}
No preprocessing techniques have been used to preprocess data \lstinline[style=verbatim-text]| plot0 | has been represented with the following techniques:
 during the experiment.
\BLOCK{endif}
% _preprocessing on plot0 ____




\BLOCK{if my_dict is defined and
      my_dict['field_representations'] is defined and
      'plot0' in my_dict['field_representations'] and
      'postprocessing' in my_dict['field_representations']['plot0'] and
      my_dict['field_representations']['plot0']['postprocessing'] is defined and
      my_dict['field_representations']['plot0']['postprocessing'] is not none}
    % dictionary 'preprocessing' not empty
On th field \verb| plot0 | have been applied the following postprocessing techniques:
\BLOCK{ for key, value in my_dict['field_representations']['plot0']['postprocessing'].items() }
      \BLOCK{ for k, v in value.items() }
            \verb| \VAR{ k } | ,
      \BLOCK{ endfor }
\BLOCK{ endfor }
\BLOCK{else}
% This block is rendered if no post processing technique has been applied
No postprocessing techniques have been used on the data \lstinline[style=verbatim-text]| plot0 | in this experiment.
\BLOCK{endif}
\hfill\break

% postprocessing_on_ plot0 ___



% field: plot1 representation subsection
\textbf{\lstinline[style=verbatim-text]| plot1 |} has been represented with the following techniques:
% content representation on plot1
\BLOCK{if my_dict is defined and
      my_dict['field_representations'] is defined and
      'plot1' in my_dict['field_representations']}

\BLOCK{ set valid_keys = [] }
\begin{itemize}
    \BLOCK{ for key, value in my_dict['field_representations']['plot1'].items() }
        \BLOCK{ if key != 'preprocessing' and key != 'postprocessing' }
            \BLOCK{ set _ = valid_keys.append(key) }
        \BLOCK{ endif }
    \BLOCK{ endfor }

    \BLOCK{ for valid_key in valid_keys }
        \item
        \verb| \VAR{ valid_key } | used as content representation technique.
    \BLOCK{ endfor }
\end{itemize}

% content_representation_field_ plot1 ____
\BLOCK{endif}



% preprocessing on plot1
\BLOCK{ if my_dict is defined and
      my_dict['field_representations'] is defined and
      'plot1' in my_dict['field_representations'] and
      'preprocessing' in my_dict['field_representations']['plot1'] and
      my_dict['field_representations']['plot1']['preprocessing'] is defined and
      my_dict['field_representations']['plot1']['preprocessing'] is not none }
    % dictionary 'preprocessing' not empty
\begin{itemize}
\BLOCK{ for key, value in my_dict['field_representations']['plot1']['preprocessing'].items() }
    \item
     \verb| \VAR{ key } | used as preprocessing technique.
\BLOCK{ endfor }
\end{itemize}
\BLOCK{else}
No preprocessing techniques have been used to preprocess data \lstinline[style=verbatim-text]| plot1 | has been represented with the following techniques:
 during the experiment.
\BLOCK{endif}
% _preprocessing on plot1 ____




\BLOCK{if my_dict is defined and
      my_dict['field_representations'] is defined and
      'plot1' in my_dict['field_representations'] and
      'postprocessing' in my_dict['field_representations']['plot1'] and
      my_dict['field_representations']['plot1']['postprocessing'] is defined and
      my_dict['field_representations']['plot1']['postprocessing'] is not none}
    % dictionary 'preprocessing' not empty
On th field \verb| plot1 | have been applied the following postprocessing techniques:
\BLOCK{ for key, value in my_dict['field_representations']['plot1']['postprocessing'].items() }
      \BLOCK{ for k, v in value.items() }
            \verb| \VAR{ k } | ,
      \BLOCK{ endfor }
\BLOCK{ endfor }
\BLOCK{else}
% This block is rendered if no post processing technique has been applied
No postprocessing techniques have been used on the data \lstinline[style=verbatim-text]| plot1 | in this experiment.
\BLOCK{endif}
\hfill\break

% postprocessing_on_ plot1 ___



% field: plot10 representation subsection
\textbf{\lstinline[style=verbatim-text]| plot10 |} has been represented with the following techniques:
% content representation on plot10
\BLOCK{if my_dict is defined and
      my_dict['field_representations'] is defined and
      'plot10' in my_dict['field_representations']}

\BLOCK{ set valid_keys = [] }
\begin{itemize}
    \BLOCK{ for key, value in my_dict['field_representations']['plot10'].items() }
        \BLOCK{ if key != 'preprocessing' and key != 'postprocessing' }
            \BLOCK{ set _ = valid_keys.append(key) }
        \BLOCK{ endif }
    \BLOCK{ endfor }

    \BLOCK{ for valid_key in valid_keys }
        \item
        \verb| \VAR{ valid_key } | used as content representation technique.
    \BLOCK{ endfor }
\end{itemize}

% content_representation_field_ plot10 ____
\BLOCK{endif}



% preprocessing on plot10
\BLOCK{ if my_dict is defined and
      my_dict['field_representations'] is defined and
      'plot10' in my_dict['field_representations'] and
      'preprocessing' in my_dict['field_representations']['plot10'] and
      my_dict['field_representations']['plot10']['preprocessing'] is defined and
      my_dict['field_representations']['plot10']['preprocessing'] is not none }
    % dictionary 'preprocessing' not empty
\begin{itemize}
\BLOCK{ for key, value in my_dict['field_representations']['plot10']['preprocessing'].items() }
    \item
     \verb| \VAR{ key } | used as preprocessing technique.
\BLOCK{ endfor }
\end{itemize}
\BLOCK{else}
No preprocessing techniques have been used to preprocess data \lstinline[style=verbatim-text]| plot10 | has been represented with the following techniques:
 during the experiment.
\BLOCK{endif}
% _preprocessing on plot10 ____




\BLOCK{if my_dict is defined and
      my_dict['field_representations'] is defined and
      'plot10' in my_dict['field_representations'] and
      'postprocessing' in my_dict['field_representations']['plot10'] and
      my_dict['field_representations']['plot10']['postprocessing'] is defined and
      my_dict['field_representations']['plot10']['postprocessing'] is not none}
    % dictionary 'preprocessing' not empty
On th field \verb| plot10 | have been applied the following postprocessing techniques:
\BLOCK{ for key, value in my_dict['field_representations']['plot10']['postprocessing'].items() }
      \BLOCK{ for k, v in value.items() }
            \verb| \VAR{ k } | ,
      \BLOCK{ endfor }
\BLOCK{ endfor }
\BLOCK{else}
% This block is rendered if no post processing technique has been applied
No postprocessing techniques have been used on the data \lstinline[style=verbatim-text]| plot10 | in this experiment.
\BLOCK{endif}
\hfill\break

% postprocessing_on_ plot10 ___



% field: plot11 representation subsection
\textbf{\lstinline[style=verbatim-text]| plot11 |} has been represented with the following techniques:
% content representation on plot11
\BLOCK{if my_dict is defined and
      my_dict['field_representations'] is defined and
      'plot11' in my_dict['field_representations']}

\BLOCK{ set valid_keys = [] }
\begin{itemize}
    \BLOCK{ for key, value in my_dict['field_representations']['plot11'].items() }
        \BLOCK{ if key != 'preprocessing' and key != 'postprocessing' }
            \BLOCK{ set _ = valid_keys.append(key) }
        \BLOCK{ endif }
    \BLOCK{ endfor }

    \BLOCK{ for valid_key in valid_keys }
        \item
        \verb| \VAR{ valid_key } | used as content representation technique.
    \BLOCK{ endfor }
\end{itemize}

% content_representation_field_ plot11 ____
\BLOCK{endif}



% preprocessing on plot11
\BLOCK{ if my_dict is defined and
      my_dict['field_representations'] is defined and
      'plot11' in my_dict['field_representations'] and
      'preprocessing' in my_dict['field_representations']['plot11'] and
      my_dict['field_representations']['plot11']['preprocessing'] is defined and
      my_dict['field_representations']['plot11']['preprocessing'] is not none }
    % dictionary 'preprocessing' not empty
\begin{itemize}
\BLOCK{ for key, value in my_dict['field_representations']['plot11']['preprocessing'].items() }
    \item
     \verb| \VAR{ key } | used as preprocessing technique.
\BLOCK{ endfor }
\end{itemize}
\BLOCK{else}
No preprocessing techniques have been used to preprocess data \lstinline[style=verbatim-text]| plot11 | has been represented with the following techniques:
 during the experiment.
\BLOCK{endif}
% _preprocessing on plot11 ____




\BLOCK{if my_dict is defined and
      my_dict['field_representations'] is defined and
      'plot11' in my_dict['field_representations'] and
      'postprocessing' in my_dict['field_representations']['plot11'] and
      my_dict['field_representations']['plot11']['postprocessing'] is defined and
      my_dict['field_representations']['plot11']['postprocessing'] is not none}
    % dictionary 'preprocessing' not empty
On th field \verb| plot11 | have been applied the following postprocessing techniques:
\BLOCK{ for key, value in my_dict['field_representations']['plot11']['postprocessing'].items() }
      \BLOCK{ for k, v in value.items() }
            \verb| \VAR{ k } | ,
      \BLOCK{ endfor }
\BLOCK{ endfor }
\BLOCK{else}
% This block is rendered if no post processing technique has been applied
No postprocessing techniques have been used on the data \lstinline[style=verbatim-text]| plot11 | in this experiment.
\BLOCK{endif}
\hfill\break

% postprocessing_on_ plot11 ___



% field: plot2 representation subsection
\textbf{\lstinline[style=verbatim-text]| plot2 |} has been represented with the following techniques:
% content representation on plot2
\BLOCK{if my_dict is defined and
      my_dict['field_representations'] is defined and
      'plot2' in my_dict['field_representations']}

\BLOCK{ set valid_keys = [] }
\begin{itemize}
    \BLOCK{ for key, value in my_dict['field_representations']['plot2'].items() }
        \BLOCK{ if key != 'preprocessing' and key != 'postprocessing' }
            \BLOCK{ set _ = valid_keys.append(key) }
        \BLOCK{ endif }
    \BLOCK{ endfor }

    \BLOCK{ for valid_key in valid_keys }
        \item
        \verb| \VAR{ valid_key } | used as content representation technique.
    \BLOCK{ endfor }
\end{itemize}

% content_representation_field_ plot2 ____
\BLOCK{endif}



% preprocessing on plot2
\BLOCK{ if my_dict is defined and
      my_dict['field_representations'] is defined and
      'plot2' in my_dict['field_representations'] and
      'preprocessing' in my_dict['field_representations']['plot2'] and
      my_dict['field_representations']['plot2']['preprocessing'] is defined and
      my_dict['field_representations']['plot2']['preprocessing'] is not none }
    % dictionary 'preprocessing' not empty
\begin{itemize}
\BLOCK{ for key, value in my_dict['field_representations']['plot2']['preprocessing'].items() }
    \item
     \verb| \VAR{ key } | used as preprocessing technique.
\BLOCK{ endfor }
\end{itemize}
\BLOCK{else}
No preprocessing techniques have been used to preprocess data \lstinline[style=verbatim-text]| plot2 | has been represented with the following techniques:
 during the experiment.
\BLOCK{endif}
% _preprocessing on plot2 ____




\BLOCK{if my_dict is defined and
      my_dict['field_representations'] is defined and
      'plot2' in my_dict['field_representations'] and
      'postprocessing' in my_dict['field_representations']['plot2'] and
      my_dict['field_representations']['plot2']['postprocessing'] is defined and
      my_dict['field_representations']['plot2']['postprocessing'] is not none}
    % dictionary 'preprocessing' not empty
On th field \verb| plot2 | have been applied the following postprocessing techniques:
\BLOCK{ for key, value in my_dict['field_representations']['plot2']['postprocessing'].items() }
      \BLOCK{ for k, v in value.items() }
            \verb| \VAR{ k } | ,
      \BLOCK{ endfor }
\BLOCK{ endfor }
\BLOCK{else}
% This block is rendered if no post processing technique has been applied
No postprocessing techniques have been used on the data \lstinline[style=verbatim-text]| plot2 | in this experiment.
\BLOCK{endif}
\hfill\break

% postprocessing_on_ plot2 ___



% field: plot3 representation subsection
\textbf{\lstinline[style=verbatim-text]| plot3 |} has been represented with the following techniques:
% content representation on plot3
\BLOCK{if my_dict is defined and
      my_dict['field_representations'] is defined and
      'plot3' in my_dict['field_representations']}

\BLOCK{ set valid_keys = [] }
\begin{itemize}
    \BLOCK{ for key, value in my_dict['field_representations']['plot3'].items() }
        \BLOCK{ if key != 'preprocessing' and key != 'postprocessing' }
            \BLOCK{ set _ = valid_keys.append(key) }
        \BLOCK{ endif }
    \BLOCK{ endfor }

    \BLOCK{ for valid_key in valid_keys }
        \item
        \verb| \VAR{ valid_key } | used as content representation technique.
    \BLOCK{ endfor }
\end{itemize}

% content_representation_field_ plot3 ____
\BLOCK{endif}



% preprocessing on plot3
\BLOCK{ if my_dict is defined and
      my_dict['field_representations'] is defined and
      'plot3' in my_dict['field_representations'] and
      'preprocessing' in my_dict['field_representations']['plot3'] and
      my_dict['field_representations']['plot3']['preprocessing'] is defined and
      my_dict['field_representations']['plot3']['preprocessing'] is not none }
    % dictionary 'preprocessing' not empty
\begin{itemize}
\BLOCK{ for key, value in my_dict['field_representations']['plot3']['preprocessing'].items() }
    \item
     \verb| \VAR{ key } | used as preprocessing technique.
\BLOCK{ endfor }
\end{itemize}
\BLOCK{else}
No preprocessing techniques have been used to preprocess data \lstinline[style=verbatim-text]| plot3 | has been represented with the following techniques:
 during the experiment.
\BLOCK{endif}
% _preprocessing on plot3 ____




\BLOCK{if my_dict is defined and
      my_dict['field_representations'] is defined and
      'plot3' in my_dict['field_representations'] and
      'postprocessing' in my_dict['field_representations']['plot3'] and
      my_dict['field_representations']['plot3']['postprocessing'] is defined and
      my_dict['field_representations']['plot3']['postprocessing'] is not none}
    % dictionary 'preprocessing' not empty
On th field \verb| plot3 | have been applied the following postprocessing techniques:
\BLOCK{ for key, value in my_dict['field_representations']['plot3']['postprocessing'].items() }
      \BLOCK{ for k, v in value.items() }
            \verb| \VAR{ k } | ,
      \BLOCK{ endfor }
\BLOCK{ endfor }
\BLOCK{else}
% This block is rendered if no post processing technique has been applied
No postprocessing techniques have been used on the data \lstinline[style=verbatim-text]| plot3 | in this experiment.
\BLOCK{endif}
\hfill\break

% postprocessing_on_ plot3 ___



% field: plot4 representation subsection
\textbf{\lstinline[style=verbatim-text]| plot4 |} has been represented with the following techniques:
% content representation on plot4
\BLOCK{if my_dict is defined and
      my_dict['field_representations'] is defined and
      'plot4' in my_dict['field_representations']}

\BLOCK{ set valid_keys = [] }
\begin{itemize}
    \BLOCK{ for key, value in my_dict['field_representations']['plot4'].items() }
        \BLOCK{ if key != 'preprocessing' and key != 'postprocessing' }
            \BLOCK{ set _ = valid_keys.append(key) }
        \BLOCK{ endif }
    \BLOCK{ endfor }

    \BLOCK{ for valid_key in valid_keys }
        \item
        \verb| \VAR{ valid_key } | used as content representation technique.
    \BLOCK{ endfor }
\end{itemize}

% content_representation_field_ plot4 ____
\BLOCK{endif}



% preprocessing on plot4
\BLOCK{ if my_dict is defined and
      my_dict['field_representations'] is defined and
      'plot4' in my_dict['field_representations'] and
      'preprocessing' in my_dict['field_representations']['plot4'] and
      my_dict['field_representations']['plot4']['preprocessing'] is defined and
      my_dict['field_representations']['plot4']['preprocessing'] is not none }
    % dictionary 'preprocessing' not empty
\begin{itemize}
\BLOCK{ for key, value in my_dict['field_representations']['plot4']['preprocessing'].items() }
    \item
     \verb| \VAR{ key } | used as preprocessing technique.
\BLOCK{ endfor }
\end{itemize}
\BLOCK{else}
No preprocessing techniques have been used to preprocess data \lstinline[style=verbatim-text]| plot4 | has been represented with the following techniques:
 during the experiment.
\BLOCK{endif}
% _preprocessing on plot4 ____




\BLOCK{if my_dict is defined and
      my_dict['field_representations'] is defined and
      'plot4' in my_dict['field_representations'] and
      'postprocessing' in my_dict['field_representations']['plot4'] and
      my_dict['field_representations']['plot4']['postprocessing'] is defined and
      my_dict['field_representations']['plot4']['postprocessing'] is not none}
    % dictionary 'preprocessing' not empty
On th field \verb| plot4 | have been applied the following postprocessing techniques:
\BLOCK{ for key, value in my_dict['field_representations']['plot4']['postprocessing'].items() }
      \BLOCK{ for k, v in value.items() }
            \verb| \VAR{ k } | ,
      \BLOCK{ endfor }
\BLOCK{ endfor }
\BLOCK{else}
% This block is rendered if no post processing technique has been applied
No postprocessing techniques have been used on the data \lstinline[style=verbatim-text]| plot4 | in this experiment.
\BLOCK{endif}
\hfill\break

% postprocessing_on_ plot4 ___



% field: plot5 representation subsection
\textbf{\lstinline[style=verbatim-text]| plot5 |} has been represented with the following techniques:
% content representation on plot5
\BLOCK{if my_dict is defined and
      my_dict['field_representations'] is defined and
      'plot5' in my_dict['field_representations']}

\BLOCK{ set valid_keys = [] }
\begin{itemize}
    \BLOCK{ for key, value in my_dict['field_representations']['plot5'].items() }
        \BLOCK{ if key != 'preprocessing' and key != 'postprocessing' }
            \BLOCK{ set _ = valid_keys.append(key) }
        \BLOCK{ endif }
    \BLOCK{ endfor }

    \BLOCK{ for valid_key in valid_keys }
        \item
        \verb| \VAR{ valid_key } | used as content representation technique.
    \BLOCK{ endfor }
\end{itemize}

% content_representation_field_ plot5 ____
\BLOCK{endif}



% preprocessing on plot5
\BLOCK{ if my_dict is defined and
      my_dict['field_representations'] is defined and
      'plot5' in my_dict['field_representations'] and
      'preprocessing' in my_dict['field_representations']['plot5'] and
      my_dict['field_representations']['plot5']['preprocessing'] is defined and
      my_dict['field_representations']['plot5']['preprocessing'] is not none }
    % dictionary 'preprocessing' not empty
\begin{itemize}
\BLOCK{ for key, value in my_dict['field_representations']['plot5']['preprocessing'].items() }
    \item
     \verb| \VAR{ key } | used as preprocessing technique.
\BLOCK{ endfor }
\end{itemize}
\BLOCK{else}
No preprocessing techniques have been used to preprocess data \lstinline[style=verbatim-text]| plot5 | has been represented with the following techniques:
 during the experiment.
\BLOCK{endif}
% _preprocessing on plot5 ____




\BLOCK{if my_dict is defined and
      my_dict['field_representations'] is defined and
      'plot5' in my_dict['field_representations'] and
      'postprocessing' in my_dict['field_representations']['plot5'] and
      my_dict['field_representations']['plot5']['postprocessing'] is defined and
      my_dict['field_representations']['plot5']['postprocessing'] is not none}
    % dictionary 'preprocessing' not empty
On th field \verb| plot5 | have been applied the following postprocessing techniques:
\BLOCK{ for key, value in my_dict['field_representations']['plot5']['postprocessing'].items() }
      \BLOCK{ for k, v in value.items() }
            \verb| \VAR{ k } | ,
      \BLOCK{ endfor }
\BLOCK{ endfor }
\BLOCK{else}
% This block is rendered if no post processing technique has been applied
No postprocessing techniques have been used on the data \lstinline[style=verbatim-text]| plot5 | in this experiment.
\BLOCK{endif}
\hfill\break

% postprocessing_on_ plot5 ___



% field: plot6 representation subsection
\textbf{\lstinline[style=verbatim-text]| plot6 |} has been represented with the following techniques:
% content representation on plot6
\BLOCK{if my_dict is defined and
      my_dict['field_representations'] is defined and
      'plot6' in my_dict['field_representations']}

\BLOCK{ set valid_keys = [] }
\begin{itemize}
    \BLOCK{ for key, value in my_dict['field_representations']['plot6'].items() }
        \BLOCK{ if key != 'preprocessing' and key != 'postprocessing' }
            \BLOCK{ set _ = valid_keys.append(key) }
        \BLOCK{ endif }
    \BLOCK{ endfor }

    \BLOCK{ for valid_key in valid_keys }
        \item
        \verb| \VAR{ valid_key } | used as content representation technique.
    \BLOCK{ endfor }
\end{itemize}

% content_representation_field_ plot6 ____
\BLOCK{endif}



% preprocessing on plot6
\BLOCK{ if my_dict is defined and
      my_dict['field_representations'] is defined and
      'plot6' in my_dict['field_representations'] and
      'preprocessing' in my_dict['field_representations']['plot6'] and
      my_dict['field_representations']['plot6']['preprocessing'] is defined and
      my_dict['field_representations']['plot6']['preprocessing'] is not none }
    % dictionary 'preprocessing' not empty
\begin{itemize}
\BLOCK{ for key, value in my_dict['field_representations']['plot6']['preprocessing'].items() }
    \item
     \verb| \VAR{ key } | used as preprocessing technique.
\BLOCK{ endfor }
\end{itemize}
\BLOCK{else}
No preprocessing techniques have been used to preprocess data \lstinline[style=verbatim-text]| plot6 | has been represented with the following techniques:
 during the experiment.
\BLOCK{endif}
% _preprocessing on plot6 ____




\BLOCK{if my_dict is defined and
      my_dict['field_representations'] is defined and
      'plot6' in my_dict['field_representations'] and
      'postprocessing' in my_dict['field_representations']['plot6'] and
      my_dict['field_representations']['plot6']['postprocessing'] is defined and
      my_dict['field_representations']['plot6']['postprocessing'] is not none}
    % dictionary 'preprocessing' not empty
On th field \verb| plot6 | have been applied the following postprocessing techniques:
\BLOCK{ for key, value in my_dict['field_representations']['plot6']['postprocessing'].items() }
      \BLOCK{ for k, v in value.items() }
            \verb| \VAR{ k } | ,
      \BLOCK{ endfor }
\BLOCK{ endfor }
\BLOCK{else}
% This block is rendered if no post processing technique has been applied
No postprocessing techniques have been used on the data \lstinline[style=verbatim-text]| plot6 | in this experiment.
\BLOCK{endif}
\hfill\break

% postprocessing_on_ plot6 ___



% field: plot7 representation subsection
\textbf{\lstinline[style=verbatim-text]| plot7 |} has been represented with the following techniques:
% content representation on plot7
\BLOCK{if my_dict is defined and
      my_dict['field_representations'] is defined and
      'plot7' in my_dict['field_representations']}

\BLOCK{ set valid_keys = [] }
\begin{itemize}
    \BLOCK{ for key, value in my_dict['field_representations']['plot7'].items() }
        \BLOCK{ if key != 'preprocessing' and key != 'postprocessing' }
            \BLOCK{ set _ = valid_keys.append(key) }
        \BLOCK{ endif }
    \BLOCK{ endfor }

    \BLOCK{ for valid_key in valid_keys }
        \item
        \verb| \VAR{ valid_key } | used as content representation technique.
    \BLOCK{ endfor }
\end{itemize}

% content_representation_field_ plot7 ____
\BLOCK{endif}



% preprocessing on plot7
\BLOCK{ if my_dict is defined and
      my_dict['field_representations'] is defined and
      'plot7' in my_dict['field_representations'] and
      'preprocessing' in my_dict['field_representations']['plot7'] and
      my_dict['field_representations']['plot7']['preprocessing'] is defined and
      my_dict['field_representations']['plot7']['preprocessing'] is not none }
    % dictionary 'preprocessing' not empty
\begin{itemize}
\BLOCK{ for key, value in my_dict['field_representations']['plot7']['preprocessing'].items() }
    \item
     \verb| \VAR{ key } | used as preprocessing technique.
\BLOCK{ endfor }
\end{itemize}
\BLOCK{else}
No preprocessing techniques have been used to preprocess data \lstinline[style=verbatim-text]| plot7 | has been represented with the following techniques:
 during the experiment.
\BLOCK{endif}
% _preprocessing on plot7 ____




\BLOCK{if my_dict is defined and
      my_dict['field_representations'] is defined and
      'plot7' in my_dict['field_representations'] and
      'postprocessing' in my_dict['field_representations']['plot7'] and
      my_dict['field_representations']['plot7']['postprocessing'] is defined and
      my_dict['field_representations']['plot7']['postprocessing'] is not none}
    % dictionary 'preprocessing' not empty
On th field \verb| plot7 | have been applied the following postprocessing techniques:
\BLOCK{ for key, value in my_dict['field_representations']['plot7']['postprocessing'].items() }
      \BLOCK{ for k, v in value.items() }
            \verb| \VAR{ k } | ,
      \BLOCK{ endfor }
\BLOCK{ endfor }
\BLOCK{else}
% This block is rendered if no post processing technique has been applied
No postprocessing techniques have been used on the data \lstinline[style=verbatim-text]| plot7 | in this experiment.
\BLOCK{endif}
\hfill\break

% postprocessing_on_ plot7 ___



% field: plot8 representation subsection
\textbf{\lstinline[style=verbatim-text]| plot8 |} has been represented with the following techniques:
% content representation on plot8
\BLOCK{if my_dict is defined and
      my_dict['field_representations'] is defined and
      'plot8' in my_dict['field_representations']}

\BLOCK{ set valid_keys = [] }
\begin{itemize}
    \BLOCK{ for key, value in my_dict['field_representations']['plot8'].items() }
        \BLOCK{ if key != 'preprocessing' and key != 'postprocessing' }
            \BLOCK{ set _ = valid_keys.append(key) }
        \BLOCK{ endif }
    \BLOCK{ endfor }

    \BLOCK{ for valid_key in valid_keys }
        \item
        \verb| \VAR{ valid_key } | used as content representation technique.
    \BLOCK{ endfor }
\end{itemize}

% content_representation_field_ plot8 ____
\BLOCK{endif}



% preprocessing on plot8
\BLOCK{ if my_dict is defined and
      my_dict['field_representations'] is defined and
      'plot8' in my_dict['field_representations'] and
      'preprocessing' in my_dict['field_representations']['plot8'] and
      my_dict['field_representations']['plot8']['preprocessing'] is defined and
      my_dict['field_representations']['plot8']['preprocessing'] is not none }
    % dictionary 'preprocessing' not empty
\begin{itemize}
\BLOCK{ for key, value in my_dict['field_representations']['plot8']['preprocessing'].items() }
    \item
     \verb| \VAR{ key } | used as preprocessing technique.
\BLOCK{ endfor }
\end{itemize}
\BLOCK{else}
No preprocessing techniques have been used to preprocess data \lstinline[style=verbatim-text]| plot8 | has been represented with the following techniques:
 during the experiment.
\BLOCK{endif}
% _preprocessing on plot8 ____




\BLOCK{if my_dict is defined and
      my_dict['field_representations'] is defined and
      'plot8' in my_dict['field_representations'] and
      'postprocessing' in my_dict['field_representations']['plot8'] and
      my_dict['field_representations']['plot8']['postprocessing'] is defined and
      my_dict['field_representations']['plot8']['postprocessing'] is not none}
    % dictionary 'preprocessing' not empty
On th field \verb| plot8 | have been applied the following postprocessing techniques:
\BLOCK{ for key, value in my_dict['field_representations']['plot8']['postprocessing'].items() }
      \BLOCK{ for k, v in value.items() }
            \verb| \VAR{ k } | ,
      \BLOCK{ endfor }
\BLOCK{ endfor }
\BLOCK{else}
% This block is rendered if no post processing technique has been applied
No postprocessing techniques have been used on the data \lstinline[style=verbatim-text]| plot8 | in this experiment.
\BLOCK{endif}
\hfill\break

% postprocessing_on_ plot8 ___



% field: plot9 representation subsection
\textbf{\lstinline[style=verbatim-text]| plot9 |} has been represented with the following techniques:
% content representation on plot9
\BLOCK{if my_dict is defined and
      my_dict['field_representations'] is defined and
      'plot9' in my_dict['field_representations']}

\BLOCK{ set valid_keys = [] }
\begin{itemize}
    \BLOCK{ for key, value in my_dict['field_representations']['plot9'].items() }
        \BLOCK{ if key != 'preprocessing' and key != 'postprocessing' }
            \BLOCK{ set _ = valid_keys.append(key) }
        \BLOCK{ endif }
    \BLOCK{ endfor }

    \BLOCK{ for valid_key in valid_keys }
        \item
        \verb| \VAR{ valid_key } | used as content representation technique.
    \BLOCK{ endfor }
\end{itemize}

% content_representation_field_ plot9 ____
\BLOCK{endif}



% preprocessing on plot9
\BLOCK{ if my_dict is defined and
      my_dict['field_representations'] is defined and
      'plot9' in my_dict['field_representations'] and
      'preprocessing' in my_dict['field_representations']['plot9'] and
      my_dict['field_representations']['plot9']['preprocessing'] is defined and
      my_dict['field_representations']['plot9']['preprocessing'] is not none }
    % dictionary 'preprocessing' not empty
\begin{itemize}
\BLOCK{ for key, value in my_dict['field_representations']['plot9']['preprocessing'].items() }
    \item
     \verb| \VAR{ key } | used as preprocessing technique.
\BLOCK{ endfor }
\end{itemize}
\BLOCK{else}
No preprocessing techniques have been used to preprocess data \lstinline[style=verbatim-text]| plot9 | has been represented with the following techniques:
 during the experiment.
\BLOCK{endif}
% _preprocessing on plot9 ____




\BLOCK{if my_dict is defined and
      my_dict['field_representations'] is defined and
      'plot9' in my_dict['field_representations'] and
      'postprocessing' in my_dict['field_representations']['plot9'] and
      my_dict['field_representations']['plot9']['postprocessing'] is defined and
      my_dict['field_representations']['plot9']['postprocessing'] is not none}
    % dictionary 'preprocessing' not empty
On th field \verb| plot9 | have been applied the following postprocessing techniques:
\BLOCK{ for key, value in my_dict['field_representations']['plot9']['postprocessing'].items() }
      \BLOCK{ for k, v in value.items() }
            \verb| \VAR{ k } | ,
      \BLOCK{ endfor }
\BLOCK{ endfor }
\BLOCK{else}
% This block is rendered if no post processing technique has been applied
No postprocessing techniques have been used on the data \lstinline[style=verbatim-text]| plot9 | in this experiment.
\BLOCK{endif}
\hfill\break

% postprocessing_on_ plot9 ___



% field: video_path0 representation subsection
\textbf{\lstinline[style=verbatim-text]| video_path0 |} has been represented with the following techniques:
% content representation on video_path0
\BLOCK{if my_dict is defined and
      my_dict['field_representations'] is defined and
      'video_path0' in my_dict['field_representations']}

\BLOCK{ set valid_keys = [] }
\begin{itemize}
    \BLOCK{ for key, value in my_dict['field_representations']['video_path0'].items() }
        \BLOCK{ if key != 'preprocessing' and key != 'postprocessing' }
            \BLOCK{ set _ = valid_keys.append(key) }
        \BLOCK{ endif }
    \BLOCK{ endfor }

    \BLOCK{ for valid_key in valid_keys }
        \item
        \verb| \VAR{ valid_key } | used as content representation technique.
    \BLOCK{ endfor }
\end{itemize}

% content_representation_field_ video_path0 ____
\BLOCK{endif}



% preprocessing on video_path0
\BLOCK{ if my_dict is defined and
      my_dict['field_representations'] is defined and
      'video_path0' in my_dict['field_representations'] and
      'preprocessing' in my_dict['field_representations']['video_path0'] and
      my_dict['field_representations']['video_path0']['preprocessing'] is defined and
      my_dict['field_representations']['video_path0']['preprocessing'] is not none }
    % dictionary 'preprocessing' not empty
\begin{itemize}
\BLOCK{ for key, value in my_dict['field_representations']['video_path0']['preprocessing'].items() }
    \item
     \verb| \VAR{ key } | used as preprocessing technique.
\BLOCK{ endfor }
\end{itemize}
\BLOCK{else}
No preprocessing techniques have been used to preprocess data \lstinline[style=verbatim-text]| video_path0 | has been represented with the following techniques:
 during the experiment.
\BLOCK{endif}
% _preprocessing on video_path0 ____




\BLOCK{if my_dict is defined and
      my_dict['field_representations'] is defined and
      'video_path0' in my_dict['field_representations'] and
      'postprocessing' in my_dict['field_representations']['video_path0'] and
      my_dict['field_representations']['video_path0']['postprocessing'] is defined and
      my_dict['field_representations']['video_path0']['postprocessing'] is not none}
    % dictionary 'preprocessing' not empty
On th field \verb| video_path0 | have been applied the following postprocessing techniques:
\BLOCK{ for key, value in my_dict['field_representations']['video_path0']['postprocessing'].items() }
      \BLOCK{ for k, v in value.items() }
            \verb| \VAR{ k } | ,
      \BLOCK{ endfor }
\BLOCK{ endfor }
\BLOCK{else}
% This block is rendered if no post processing technique has been applied
No postprocessing techniques have been used on the data \lstinline[style=verbatim-text]| video_path0 | in this experiment.
\BLOCK{endif}
\hfill\break

% postprocessing_on_ video_path0 ___



% field: video_path1 representation subsection
\textbf{\lstinline[style=verbatim-text]| video_path1 |} has been represented with the following techniques:
% content representation on video_path1
\BLOCK{if my_dict is defined and
      my_dict['field_representations'] is defined and
      'video_path1' in my_dict['field_representations']}

\BLOCK{ set valid_keys = [] }
\begin{itemize}
    \BLOCK{ for key, value in my_dict['field_representations']['video_path1'].items() }
        \BLOCK{ if key != 'preprocessing' and key != 'postprocessing' }
            \BLOCK{ set _ = valid_keys.append(key) }
        \BLOCK{ endif }
    \BLOCK{ endfor }

    \BLOCK{ for valid_key in valid_keys }
        \item
        \verb| \VAR{ valid_key } | used as content representation technique.
    \BLOCK{ endfor }
\end{itemize}

% content_representation_field_ video_path1 ____
\BLOCK{endif}



% preprocessing on video_path1
\BLOCK{ if my_dict is defined and
      my_dict['field_representations'] is defined and
      'video_path1' in my_dict['field_representations'] and
      'preprocessing' in my_dict['field_representations']['video_path1'] and
      my_dict['field_representations']['video_path1']['preprocessing'] is defined and
      my_dict['field_representations']['video_path1']['preprocessing'] is not none }
    % dictionary 'preprocessing' not empty
\begin{itemize}
\BLOCK{ for key, value in my_dict['field_representations']['video_path1']['preprocessing'].items() }
    \item
     \verb| \VAR{ key } | used as preprocessing technique.
\BLOCK{ endfor }
\end{itemize}
\BLOCK{else}
No preprocessing techniques have been used to preprocess data \lstinline[style=verbatim-text]| video_path1 | has been represented with the following techniques:
 during the experiment.
\BLOCK{endif}
% _preprocessing on video_path1 ____




\BLOCK{if my_dict is defined and
      my_dict['field_representations'] is defined and
      'video_path1' in my_dict['field_representations'] and
      'postprocessing' in my_dict['field_representations']['video_path1'] and
      my_dict['field_representations']['video_path1']['postprocessing'] is defined and
      my_dict['field_representations']['video_path1']['postprocessing'] is not none}
    % dictionary 'preprocessing' not empty
On th field \verb| video_path1 | have been applied the following postprocessing techniques:
\BLOCK{ for key, value in my_dict['field_representations']['video_path1']['postprocessing'].items() }
      \BLOCK{ for k, v in value.items() }
            \verb| \VAR{ k } | ,
      \BLOCK{ endfor }
\BLOCK{ endfor }
\BLOCK{else}
% This block is rendered if no post processing technique has been applied
No postprocessing techniques have been used on the data \lstinline[style=verbatim-text]| video_path1 | in this experiment.
\BLOCK{endif}
\hfill\break

% postprocessing_on_ video_path1 ___



% field: video_path2 representation subsection
\textbf{\lstinline[style=verbatim-text]| video_path2 |} has been represented with the following techniques:
% content representation on video_path2
\BLOCK{if my_dict is defined and
      my_dict['field_representations'] is defined and
      'video_path2' in my_dict['field_representations']}

\BLOCK{ set valid_keys = [] }
\begin{itemize}
    \BLOCK{ for key, value in my_dict['field_representations']['video_path2'].items() }
        \BLOCK{ if key != 'preprocessing' and key != 'postprocessing' }
            \BLOCK{ set _ = valid_keys.append(key) }
        \BLOCK{ endif }
    \BLOCK{ endfor }

    \BLOCK{ for valid_key in valid_keys }
        \item
        \verb| \VAR{ valid_key } | used as content representation technique.
    \BLOCK{ endfor }
\end{itemize}

% content_representation_field_ video_path2 ____
\BLOCK{endif}



% preprocessing on video_path2
\BLOCK{ if my_dict is defined and
      my_dict['field_representations'] is defined and
      'video_path2' in my_dict['field_representations'] and
      'preprocessing' in my_dict['field_representations']['video_path2'] and
      my_dict['field_representations']['video_path2']['preprocessing'] is defined and
      my_dict['field_representations']['video_path2']['preprocessing'] is not none }
    % dictionary 'preprocessing' not empty
\begin{itemize}
\BLOCK{ for key, value in my_dict['field_representations']['video_path2']['preprocessing'].items() }
    \item
     \verb| \VAR{ key } | used as preprocessing technique.
\BLOCK{ endfor }
\end{itemize}
\BLOCK{else}
No preprocessing techniques have been used to preprocess data \lstinline[style=verbatim-text]| video_path2 | has been represented with the following techniques:
 during the experiment.
\BLOCK{endif}
% _preprocessing on video_path2 ____




\BLOCK{if my_dict is defined and
      my_dict['field_representations'] is defined and
      'video_path2' in my_dict['field_representations'] and
      'postprocessing' in my_dict['field_representations']['video_path2'] and
      my_dict['field_representations']['video_path2']['postprocessing'] is defined and
      my_dict['field_representations']['video_path2']['postprocessing'] is not none}
    % dictionary 'preprocessing' not empty
On th field \verb| video_path2 | have been applied the following postprocessing techniques:
\BLOCK{ for key, value in my_dict['field_representations']['video_path2']['postprocessing'].items() }
      \BLOCK{ for k, v in value.items() }
            \verb| \VAR{ k } | ,
      \BLOCK{ endfor }
\BLOCK{ endfor }
\BLOCK{else}
% This block is rendered if no post processing technique has been applied
No postprocessing techniques have been used on the data \lstinline[style=verbatim-text]| video_path2 | in this experiment.
\BLOCK{endif}
\hfill\break

% postprocessing_on_ video_path2 ___



% field: video_path3 representation subsection
\textbf{\lstinline[style=verbatim-text]| video_path3 |} has been represented with the following techniques:
% content representation on video_path3
\BLOCK{if my_dict is defined and
      my_dict['field_representations'] is defined and
      'video_path3' in my_dict['field_representations']}

\BLOCK{ set valid_keys = [] }
\begin{itemize}
    \BLOCK{ for key, value in my_dict['field_representations']['video_path3'].items() }
        \BLOCK{ if key != 'preprocessing' and key != 'postprocessing' }
            \BLOCK{ set _ = valid_keys.append(key) }
        \BLOCK{ endif }
    \BLOCK{ endfor }

    \BLOCK{ for valid_key in valid_keys }
        \item
        \verb| \VAR{ valid_key } | used as content representation technique.
    \BLOCK{ endfor }
\end{itemize}

% content_representation_field_ video_path3 ____
\BLOCK{endif}



% preprocessing on video_path3
\BLOCK{ if my_dict is defined and
      my_dict['field_representations'] is defined and
      'video_path3' in my_dict['field_representations'] and
      'preprocessing' in my_dict['field_representations']['video_path3'] and
      my_dict['field_representations']['video_path3']['preprocessing'] is defined and
      my_dict['field_representations']['video_path3']['preprocessing'] is not none }
    % dictionary 'preprocessing' not empty
\begin{itemize}
\BLOCK{ for key, value in my_dict['field_representations']['video_path3']['preprocessing'].items() }
    \item
     \verb| \VAR{ key } | used as preprocessing technique.
\BLOCK{ endfor }
\end{itemize}
\BLOCK{else}
No preprocessing techniques have been used to preprocess data \lstinline[style=verbatim-text]| video_path3 | has been represented with the following techniques:
 during the experiment.
\BLOCK{endif}
% _preprocessing on video_path3 ____




\BLOCK{if my_dict is defined and
      my_dict['field_representations'] is defined and
      'video_path3' in my_dict['field_representations'] and
      'postprocessing' in my_dict['field_representations']['video_path3'] and
      my_dict['field_representations']['video_path3']['postprocessing'] is defined and
      my_dict['field_representations']['video_path3']['postprocessing'] is not none}
    % dictionary 'preprocessing' not empty
On th field \verb| video_path3 | have been applied the following postprocessing techniques:
\BLOCK{ for key, value in my_dict['field_representations']['video_path3']['postprocessing'].items() }
      \BLOCK{ for k, v in value.items() }
            \verb| \VAR{ k } | ,
      \BLOCK{ endfor }
\BLOCK{ endfor }
\BLOCK{else}
% This block is rendered if no post processing technique has been applied
No postprocessing techniques have been used on the data \lstinline[style=verbatim-text]| video_path3 | in this experiment.
\BLOCK{endif}
\hfill\break

% postprocessing_on_ video_path3 ___



% field: video_path4 representation subsection
\textbf{\lstinline[style=verbatim-text]| video_path4 |} has been represented with the following techniques:
% content representation on video_path4
\BLOCK{if my_dict is defined and
      my_dict['field_representations'] is defined and
      'video_path4' in my_dict['field_representations']}

\BLOCK{ set valid_keys = [] }
\begin{itemize}
    \BLOCK{ for key, value in my_dict['field_representations']['video_path4'].items() }
        \BLOCK{ if key != 'preprocessing' and key != 'postprocessing' }
            \BLOCK{ set _ = valid_keys.append(key) }
        \BLOCK{ endif }
    \BLOCK{ endfor }

    \BLOCK{ for valid_key in valid_keys }
        \item
        \verb| \VAR{ valid_key } | used as content representation technique.
    \BLOCK{ endfor }
\end{itemize}

% content_representation_field_ video_path4 ____
\BLOCK{endif}



% preprocessing on video_path4
\BLOCK{ if my_dict is defined and
      my_dict['field_representations'] is defined and
      'video_path4' in my_dict['field_representations'] and
      'preprocessing' in my_dict['field_representations']['video_path4'] and
      my_dict['field_representations']['video_path4']['preprocessing'] is defined and
      my_dict['field_representations']['video_path4']['preprocessing'] is not none }
    % dictionary 'preprocessing' not empty
\begin{itemize}
\BLOCK{ for key, value in my_dict['field_representations']['video_path4']['preprocessing'].items() }
    \item
     \verb| \VAR{ key } | used as preprocessing technique.
\BLOCK{ endfor }
\end{itemize}
\BLOCK{else}
No preprocessing techniques have been used to preprocess data \lstinline[style=verbatim-text]| video_path4 | has been represented with the following techniques:
 during the experiment.
\BLOCK{endif}
% _preprocessing on video_path4 ____




\BLOCK{if my_dict is defined and
      my_dict['field_representations'] is defined and
      'video_path4' in my_dict['field_representations'] and
      'postprocessing' in my_dict['field_representations']['video_path4'] and
      my_dict['field_representations']['video_path4']['postprocessing'] is defined and
      my_dict['field_representations']['video_path4']['postprocessing'] is not none}
    % dictionary 'preprocessing' not empty
On th field \verb| video_path4 | have been applied the following postprocessing techniques:
\BLOCK{ for key, value in my_dict['field_representations']['video_path4']['postprocessing'].items() }
      \BLOCK{ for k, v in value.items() }
            \verb| \VAR{ k } | ,
      \BLOCK{ endfor }
\BLOCK{ endfor }
\BLOCK{else}
% This block is rendered if no post processing technique has been applied
No postprocessing techniques have been used on the data \lstinline[style=verbatim-text]| video_path4 | in this experiment.
\BLOCK{endif}
\hfill\break

% postprocessing_on_ video_path4 ___


\BLOCK{if my_dict['exogenous_representations'] is defined and
        my_dict['exogenous_representations'] is not none}
% exogenous techniques start
In this experiment the data used have been represented with exogenous representation, in particular the
following techniques have been used:
\begin{itemize}
\BLOCK{ for key, value in (my_dict['exogenous_representations'] | default({})).items() }
    \item
     \verb| \VAR{ key } | used as exogenous technique with following settings:
     \begin{itemize}
      \BLOCK{ for k, v in (value | default({})).items() }
        \BLOCK{ if k not in ["mode", "prop_as_uri", "max_timeout"] }
             \item
             \verb| \VAR{ k }: \VAR{ v }|
        \BLOCK{ endif }
      \BLOCK{ endfor }
     \end{itemize}
\BLOCK{ endfor }
\end{itemize}
\BLOCK{else}
No exogenous techniques have been used in this experiment to represent the data used.
\BLOCK{endif}
% exogenous_techniques_end___




\BLOCK{else}
% In case of the content analyzer is not used
For this experiment the module of the content analyzer has not been used.
\hfill\break
\hfill\break
% <----------  closing the controll block for the content analyzer section ----------->
\BLOCK{endif}



\hfill\break
% subsection of Dataset identification
\subsection{Dataset used and its statistics}
The dataset used for the experiment is  \lstinline[style=verbatim-text]| 1000K data video movie |,
\BLOCK{ if my_dict is defined and my_dict['source_file'] is defined }
its source file is \lstinline[style=verbatim-text]| \VAR{ my_dict['source_file'] } | ,\BLOCK{endif}
and the content used during the experiment:
\BLOCK{ if my_dict is defined and my_dict['id_each_content'] is defined and my_dict['id_each_content'] }
\BLOCK{ for value in my_dict['id_each_content'] }
     \verb| \VAR{ value }, |
\BLOCK{ endfor }.
\BLOCK{endif}


\BLOCK{ if my_dict is defined and my_dict['interactions'] is defined }
% --- DATA STATS ---
The statistics of the dataset used are reported in the following table:~\ref{tab:dataset_table}:
\begin{table}[ht]
    \centering
  \begin{tabular}{|c|c|}
    \hline
    \textbf{Parameter}& \textbf{Value} \\ \hline
    n\_users  & \VAR{my_dict['interactions']['n_users']|default('no users')|safe_text}\\ \hline
    n\_items  & \VAR{my_dict['interactions']['n_items']|default('no items')|safe_text}\\ \hline
    total\_interactions  & \VAR{my_dict['interactions']['total_interactions']|safe_text}\\ \hline
    min\_score  & \VAR{my_dict['interactions']['min_score']|truncate|safe_text}\\ \hline
    max\_score  & \VAR{my_dict['interactions']['max_score']|truncate|safe_text}\\ \hline
    mean\_score  & \VAR{my_dict['interactions']['mean_score']|truncate|safe_text}\\ \hline
    sparsity  & \VAR{my_dict['interactions']['sparsity']|truncate|safe_text}\\ \hline
  \end{tabular}
   \caption{Stats on the dataset}\label{tab:dataset_table}
\end{table}
\BLOCK{ else }
There are no statistics on the dataset to show.
\BLOCK{ endif }

\hfill\break



% ------------------------------ START SUBSECTION OF PARTITIONING OF RECSYS --------------------------------------------
% subsection of the splitting technique used, referred to partition protocol
\subsection{Data splitting technique}\label{subsec:partitioning}
\BLOCK{if my_dict['partitioning'] is defined and
        my_dict['partitioning']['KFoldPartitioning'] is defined}
% KFOLD PARTITIONING TECNIQUE
K-fold cross-validation is a technique used in machine learning to assess the performance of a predictive model.
The basic idea is to divide the dataset into K subsets, or folds.
The model is then trained K times, each time using K-1 folds for training and the remaining fold for validation.
This process is repeated K times, with a different fold used as the validation set in each iteration.
\hfill\break
The KFoldPartitioning has been used with the following setting:
\hfill\break
\BLOCK{if my_dict.get('partitioning', {}).get('KFoldPartitioning', {}).get('shuffle') == True}
The data has been shuffled before being split into batches.
\BLOCK{endif}
The partitioning technique has been executed with the following settings:
\begin{itemize}
    \item number of splits: \VAR{my_dict['partitioning']['KFoldPartitioning']['n_splits']}
    \item shuffle: \VAR{my_dict['partitioning']['KFoldPartitioning']['shuffle']}
    \item random state: \VAR{my_dict['partitioning']['KFoldPartitioning']['random_state']|default('no random state applied')}
    \item skip user error: \VAR{my_dict['partitioning']['KFoldPartitioning']['skip_user_error']|default('no setted')}
\end{itemize}
\hfill\break
% KFOLD PARTITIONING TECNIQUE ended
\BLOCK{endif}

\BLOCK{if my_dict['partitioning'] is defined and
        my_dict['partitioning']['HoldOutPartitioning'] is defined}
%  HOLD-OUT PARTIONING TECNIQUE
The partitioning used is the Hold-Out Partitioning.
This approach splits the dataset in use into a train set and a test set.
The training set is what the model is trained on, and the test set is used to see how
well the model will perform on new, unseen data.
\hfill\break
The train set size of this experiment is the \VAR{my_dict['partitioning']['HoldOutPartitioning']['train_set_size'] * 100}\%
of the original dataset, while the test set is the remaining \VAR{(100 - (my_dict['partitioning']['HoldOutPartitioning']['train_set_size'] * 100))}\%.
\hfill\break
\BLOCK{ if my_dict.get('partitioning', {}).get('HoldOutPartitioning', {}).get('shuffle') == True }
The data has been shuffled before being split into batches.
\BLOCK{endif}
\hfill\break
%  HOLD-OUT PARTIONING TECNIQUE ended
\BLOCK{endif}
% end partitioning section___________
