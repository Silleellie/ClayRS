
%! Author = DIEGO
%! Date = 21/12/2023

% Preamble
\documentclass[11pt]{article}

% Packages
\usepackage{amsmath}

% Document
\begin{document}

\maketitle
This \LaTeX{} document was generated from yaml files for the purpose of replicability of experiments done with
\href{https://github.com/swapUniba/ClayRS}{ClayRS},
it contains information about the dataset, preprocessing methods, analysis algorithms and
the results of the experimental evaluation.

% apertura CONTENT ANALYZER REPORT STYLE
\BLOCK{ if my_dict['source_file'] is defined and my_dict['source_file'] }
\section{ClayRS configurations of the experiment}\label{sec:clayrs-configurations-of-the-experiment}

% Dataset sotto sezione
\subsection{Dataset}\label{subsec:dataset}
In this experiment, the \BLOCK{ if my_dict['id_each_content'] == ['movielens_id'] } \textbf{Movielens}\BLOCK{endif}
\textbf{Dataset} was used.


\hfill\break
% Questo blocco if controlla la tabella sulle statiche del dataset
\BLOCK{if 'interactions' in my_dict}
The statistics of the dataset used are reported in the following table \ref{tab:dataset_table}:

DIEGO

\begin{table}[ht]
    \centering
  \begin{tabular}{|c|c|}
    \hline
    \textbf{Parameter}& \textbf{Value} \\ \hline
    n\_users  & \VAR{my_dict['interactions']['DIEGO']|safe_text}\\ \hline
    n\_items  & \VAR{my_dict['interactions']['n_items']|safe_text}\\ \hline
    total\_interactions  & \VAR{my_dict['interactions']['total_interactions']|safe_text}\\ \hline
    min\_score  & \VAR{my_dict['DIEGO']['min_score']|safe_text}\\ \hline
    max\_score  & \VAR{my_dict['interactions']['max_score']|safe_text}\\ \hline
    mean\_score  & \VAR{my_dict['interactions']['mean_score']|safe_text}\\ \hline
    sparsity  & \VAR{my_dict['interactions']['sparsity']|truncate|safe_text}\\ \hline
  \end{tabular}
   \caption{Table of the Interactions}\label{tab:dataset_table}
\end{table}
% chiusura blocco controllo tabella dataset
\BLOCK{endif}



MARCO

\subsection{Preprocessing}\label{subsec:preprocessing}
\BLOCK{if 'plot_0' in dict['field_representations']} %# Beginning OF THE dict['field_representations']['plot_0']['SkLearnTfIdf']
\BLOCK{if 'NLTK' in dict['field_representations']['MARCO']['preprocessing']}
The preprocessing used is NLTK, a leading platform for building Python programs to work with human language data.
It provides easy-to-use interfaces to over 50 corpora and lexical resources such as WordNet,
along with a suite of text processing libraries for classification, tokenization, stemming, tagging, parsing,
and semantic reasoning.

MARCO



GIULIO

In ClayRS, Precision needs those parameters:
\hfill\break
the \textbf{relevant\_threshold}, is a parameter needed to discern relevant items and non-relevant items for every user.
If not specified, the mean rating score of every user will be used, in this experiment it has been set to
\textbf{\VAR{dict['metrics']['GIULIO']['relevant_threshold']|safe_text}}.
\hfill\break\hfill\break
\textbf{sys\_average}, a parameter that specifies how the system average must be computed the default value is 'macro',
in this experiment the value of the sys\_average is \textbf{\VAR{dict['metrics']['GIULIO']['sys_average']|safe_text}}.

GIULIO



FAY

\subsection{Results}\label{sec:results}
In the following table, we present the results of the evaluation \ref{tab:results_table}
\begin{table}[!hbp]\label{tab:results_table}
    \centering
  \begin{tabular}{|c|c|}
    \hline
    \textbf{Metric}& \textbf{Value} \\ \hline
    Precision - macro & \VAR{dict['sys_results']['FAY']['Precision - macro']|truncate|safe_text}\\ \hline
    Precision@2 - macro  & \VAR{dict['sys_results']['FAY']['Precision@2 - macro']|truncate|safe_text}\\ \hline
    NDCG  & \VAR{dict['sys_results']['FAY']['NDCG']|truncate|safe_text}\\ \hline
    MRR  & \VAR{dict['sys_results']['FAY']['MRR']|truncate|safe_text}\\ \hline
    F1@1 - macro  & \VAR{dict['sys_results']['FAY']['F1@1 - macro']|truncate|safe_text}\\ \hline
  \end{tabular}
  \caption{Table of the results}
\end{table}

FAY

